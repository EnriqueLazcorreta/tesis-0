% !TEX root = ../Lazcorreta.Tesis.tex

\usepackage[utf8]{inputenc}
\usepackage[T1]{fontenc}
\usepackage{times}
%\usepackage{lmodern}


%Diseño de contenido


%Formato de la clase amsart.cls, menos restrictivo que book en la colocación de flotantes
\setcounter{topnumber}{4}
\setcounter{bottomnumber}{4}
\setcounter{totalnumber}{4}
\setcounter{dbltopnumber}{4}
\renewcommand{\topfraction}{.97}
\renewcommand{\bottomfraction}{.97}
\renewcommand{\textfraction}{.03}
\renewcommand{\floatpagefraction}{.9}
\renewcommand{\dbltopfraction}{.97}
\renewcommand{\dblfloatpagefraction}{.9}
\setlength{\floatsep}{12pt plus 6pt minus 4pt}
\setlength{\textfloatsep}{15pt plus 8pt minus 5pt}
\setlength{\intextsep}{12pt plus 6pt minus 4pt}
\setlength{\dblfloatsep}{12pt plus 6pt minus 4pt}
\setlength{\dbltextfloatsep}{15pt plus 8pt minus 5pt}


\usepackage[clearempty]{titlesec}
%\titleformat{name=\chapter}

%\titleformat
%{\chapter} % command
%[display] % shape
%{\bfseries\Large\itshape} % format
%{Story No. \ \thechapter} % label
%{0.5ex} % sep
%{
    %\rule{\textwidth}{1pt}
    %\vspace{1ex}
    %\centering
%} % before-code
%[
%\vspace{-0.5ex}%
%\rule{\textwidth}{0.3pt}
%] % after-code
 %
 %
%\titleformat{\section}[wrap]
%{\normalfont\bfseries}
%{\thesection.}{0.5em}{}
 %
%\titlespacing{\section}{12pc}{1.5ex plus .1ex minus .2ex}{1pc}




% %TODO: Probando makeidx, que tiene que cargar después que hyperref
% \usepackage{makeidx}
% %TODO: Probando imakeidx, que tiene que cargar antes que hyperref
% % \usepackage{imakeidx}
% \usepackage{esindex}
% \usepackage{romanidx} %Incompatible con imakeidx







%Bibliografía
%\usepackage[round, sort, numbers]{natbib}
%\usepackage[authoryear,sort&compress]{natbib}
% \usepackage[authoryear,round]{natbib}
% \usepackage[backend=bibtex]{biblatex}
%He usado la información de http://tex.stackexchange.com/questions/44040/biblatex-biber-texmaker-miktex para configurar TeXMaker correctamente, quitando .aux de la llamada a biber en el comando BibTeX
\usepackage[
   backend=biber,
   hyperref=true,
   natbib=true,
   url=false,
   doi=false,
   isbn=false,
   backref=true,
   style=authoryear,
   maxcitenames=3,
   maxbibnames=100,
   block=none]{biblatex}
% \bibliography{./bib/TesisEnriqueLazcorreta.bib,./bib/CitanApriori2}
% \addbibresource{bib/TesisEnriqueLazcorreta.bib,bib/CitanApriori2.bib}
% \bibliography{bib/TesisEnriqueLazcorreta,bib/CitanApriori2}
\bibliography{./bib/TesisEnriqueLazcorreta}
%\bibpunct{[}{]}{,}{n}{}{;}
\usepackage{csquotes}

%\usepackage{bibunits} %Para escribir bibliografías en capítulos...
%
%\let\stdthebibliography\thebibliography
%\renewcommand{\thebibliography}{%
%\let\chapter\section
%\stdthebibliography}




%TODO: Probando makeidx, que tiene que cargar después que hyperref
\usepackage{makeidx}
%TODO: Probando imakeidx, que tiene que cargar antes que hyperref
% \usepackage{imakeidx}
\usepackage{esindex}
%\usepackage{romanidx} %Incompatible con imakeidx





%Diseño de la página física

\usepackage[paperheight = 224mm,
            paperwidth = 174mm,
            inner = 2.2cm,%2
            outer = 2.8cm,%2.3
            bottom = 2cm,%2.5
            % margin=2cm,
            marginparwidth=1.7cm,
            marginparsep=0.3cm,
            columnsep=1cm
            ]
            {geometry}




% Hay que ponerlo después de geometry para que estén ya fijadas las dimensiones del papel...
\usepackage[Bjornstrup]{fncychap} %Cabeceras de capítulos







%\usepackage[spanish,activeacute]{babel}
%\usepackage[english,spanish,es-lcroman,es-noindentfirst]{babel}  %es-lcroman permite añadir entradas al índice en frontmatter
\usepackage[english,spanish]{babel}







%Hyperenlaces en .pdf

\usepackage[
            pdftex,
            plainpages = false, 
            pdfpagelabels, 
            %pdfpagelayout = useoutlines,
            %bookmarks,
            %bookmarksopen = true,
            %bookmarksnumbered = true,
            breaklinks = true,
            %linktocpage,
            % pagebackref, %LO HE COMENTADO PARA PROBAR biblatex
            colorlinks = false,  % was true
            linkcolor = blue,
            urlcolor  = blue,
            citecolor = red,
            anchorcolor = green,
            hyperindex = true,
            hyperfigures,
            pdfduplex=DuplexFlipLongEdge %Sugiere impresión a doble cara
            ]
            {hyperref} 

\hypersetup{
            % bookmarksopen=true,
            % bookmarksopenlevel=3    % Muestra X niveles abiertos al inicio
            % bookmarksdepth=3        % Crea X niveles en bookmarks del pdf
            % bookmarks=true,         % show bookmarks bar
            % unicode=false,          % non-Latin characters in Acrobat’s bookmarks
            pdftoolbar=true,        % show Acrobat’s toolbar?
            pdfmenubar=true,        % show Acrobat’s menu?
            pdffitwindow=false,     % window fit to page when opened
            pdfstartview={FitH},    % fits the width of the page to the window
            pdftitle = {Análisis eficiente de catálogos mediante reglas de asociación},
            pdfauthor={Enrique Lazcorreta Puigmartí},     % author
            pdfsubject={Tesis doctoral - Universidad Miguel Hernández de Elche},   % subject of the document
            % pdfcreator={Creator},   % creator of the document
            % pdfproducer={Producer}, % producer of the document
            pdfkeywords={Sistemas de Recomendación Web} {Personalización de la Web} {Reglas de Asociación}, % list of keywords
            pdfnewwindow=true,      % links in new PDF window
            % colorlinks=false,       % false: boxed links; true: colored links
            % linkcolor=red,          % color of internal links (change box color with linkbordercolor)
            % citecolor=green,        % color of links to bibliography
            % filecolor=magenta,      % color of file links
            % urlcolor=cyan           % color of external links
}

\usepackage[%open,%
            openlevel=1,%
            %atend%
           ]{bookmark}%[2011/12/02]












%Tabla de contenidos
\usepackage[nottoc]{tocbibind}
 





%Gráficos
%\usepackage{graphics} % for improved inclusion of graphics
\usepackage{wrapfig} % to include figure with text wrapping around it

\usepackage[pdftex]{graphicx}
\DeclareGraphicsExtensions{.png,.jpg,.jpeg,.pdf} %GIF doesn't work??
\pdfcompresslevel=9
%TODO: Ajustar estos valores
\graphicspath{{./imagen/}%
              {./contenido/srw/imagen/}%
              {./contenido/srw/imagen/PNG/}%
              {./contenido/2-ARM/imagen/}%
              {./contenido/3-ARMCatalogos/imagen/}%
              {./contenido/4-ConclusionesYTrabajoFuturo/imagen/}%
				 }

\usepackage{tikz}
%\usetikzlibrary{arrows}
\usetikzlibrary{calc,fadings,shapes.arrows,shadows,backgrounds,positioning}

% \usepackage{subfigure} % subfiguras
\usepackage{caption}
\usepackage{subcaption}







%Varios
\usepackage{xspace}

%\usepackage{synttree} %Árboles de decisión

%%\usepackage[usenames,svgnames]{color}
\usepackage{xcolor}

\usepackage{appendix}
\renewcommand{\appendixname}{Apéndices}
\renewcommand{\appendixtocname}{Apéndices}
\renewcommand{\appendixpagename}{Apéndices}




%\usepackage{algorithm}
%\usepackage{algorithmic}
%Obtenido de https://code.google.com/p/tpso1c2010/source/browse/trunk/Informe/src/spanishAlgorithmic.tex?spec=svn2&r=2 y de http://www.rosapolis.net/2008/04/21/escribir-algoritmos-en-latex/
%\input{./include/spanishAlgorithmic} % Archivo de traducción

\usepackage{listings}
\lstloadlanguages{[ANSI]C++}
\renewcommand*{\lstlistingname}{Listado}
\renewcommand*{\lstlistlistingname}{Índice de listados}

\lstdefinelanguage{pseudocodigo}
	{frame=trBL,
	frameround=fttt,
  % float=htbp,
	mathescape=true,
	%numbers=left,
	keywords={for,do,begin,end,forall,foreach,break,continue,while,
            if,then,else,and,or,
            Answer,let,
            insert,into,select,from,where,delete,add,to,sort,
            repeat,until,on,generate,
            procedure,call,output,with,add,clear},
	keywordstyle=\textbf,
	tabsize=2,
	basicstyle=\small,
	backgroundcolor=\color{green!2!white},
	sensitive=false,
	morecomment=[l]{//},
	morecomment=[s]{/*}{*/}%,
	%morestring=[b]"
	}
\lstdefinelanguage{listado}
	{frame=trBL,
	frameround=fttt,
  % float=htbp,
	mathescape=true,
	%numbers=left,
  keywords={1.,2.,3.,4.,5.,6.},
	keywordstyle=\textbf,
	tabsize=2,
	basicstyle=\small,
	backgroundcolor=\color{green!2!white},
	sensitive=false,
	morecomment=[l]{//},
	morecomment=[s]{/*}{*/}%,
	%morestring=[b]"
	}
\lstset{literate=   % Para usar UTF-8 en los ficheros
  {á}{{\'a}}1 {é}{{\'e}}1 {í}{{\'i}}1 {ó}{{\'o}}1 {ú}{{\'u}}1
  {Á}{{\'A}}1 {É}{{\'E}}1 {Í}{{\'I}}1 {Ó}{{\'O}}1 {Ú}{{\'U}}1
  {à}{{\`a}}1 {è}{{\`e}}1 {ì}{{\`i}}1 {ò}{{\`o}}1 {ù}{{\`u}}1
  {À}{{\`A}}1 {È}{{\'E}}1 {Ì}{{\`I}}1 {Ò}{{\`O}}1 {Ù}{{\`U}}1
  {ä}{{\"a}}1 {ë}{{\"e}}1 {ï}{{\"i}}1 {ö}{{\"o}}1 {ü}{{\"u}}1
  {Ä}{{\"A}}1 {Ë}{{\"E}}1 {Ï}{{\"I}}1 {Ö}{{\"O}}1 {Ü}{{\"U}}1
  {â}{{\^a}}1 {ê}{{\^e}}1 {î}{{\^i}}1 {ô}{{\^o}}1 {û}{{\^u}}1
  {Â}{{\^A}}1 {Ê}{{\^E}}1 {Î}{{\^I}}1 {Ô}{{\^O}}1 {Û}{{\^U}}1
  {œ}{{\oe}}1 {Œ}{{\OE}}1 {æ}{{\ae}}1 {Æ}{{\AE}}1 {ß}{{\ss}}1
  {ç}{{\c c}}1 {Ç}{{\c C}}1 {ø}{{\o}}1 {å}{{\r a}}1 {Å}{{\r A}}1
  {€}{{\EUR}}1 {£}{{\pounds}}1
}
\lstset{language=pseudocodigo}




%\def\bibTeX{{\rm B\kern-.05em{\sc i\kern-.025em b}\kern-.08em T\kern-.1667em\lower.7ex\hbox{E}\kern-.125emX}}
\usepackage{dtklogos} %\BibTeX



\usepackage{longtable}

%\usepackage[standard]{ntheorem}
\usepackage{amsthm}
\usepackage{thmtools}
%\theoremstyle{break}
%\theorembodyfont{\normalfont}

%\theoremstyle{break}
%\theoremheaderfont{\normalfont\bfseries}
%\theorembodyfont{\slshape}
%\theoremsymbol{\ensuremath{\heartsuit}}
%\theoremsymbol{\ensuremath{\diamondsuit}}
%\theoremseparator{:}
%\theoremindent0.5cm
%\theoremnumbering{arabic}
\newtheorem{Definition}{Definición}
\newtheorem{Lemma}{Lema}


\usepackage{afterpage}

\def\BibTeX{\textsc{Bib}\TeX}

