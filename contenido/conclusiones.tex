% !TEX root = ../Lazcorreta.Tesis.tex
\noindent Las mejores ideas expuestas en esta tesis son muy simples. Desde el primer algoritmo que entra en juego, Apriori, hasta la elaboración de catálogos completos ínfimos son ideas muy simples que implementadas de forma eficiente pueden hacer lo que se le pide a la Minería de Datos: buscar una aguja en un pajar.

Los catálogos completos tienen un potencial fácil de descubrir mediante sencillas técnicas informáticas de Minería de Datos. Este trabajo presenta una teoría en torno a un tipo de datos muy utilizado que posibilita la obtención extrema de la información que contienen grandes colecciones de datos utilizando la tecnología actual en tiempo real. 

Los datos bien recogidos reflejan el estado actual del mundo que nos rodea, por eso es importante poder analizarlos rápidamente utilizando en algunos casos información histórica sobre el mismo problema o bien partiendo de un nuevo problema y analizando rápidamente las características de los datos que proporciona su estudio. Si sabemos qué puede descubrir la Minería de Datos a partir de la observación de los datos que hemos recogido podremos crear algoritmos que descubran lo que estamos buscando en tiempo real y con un uso aceptable de recursos de un servidor dedicado. Pero si no entendemos bien el problema que queremos resolver con los datos que tenemos es difícil que lo podamos explicar de una forma genérica a una máquina por muchos recursos y capacidad que tenga.

Este trabajo presenta unos antecedentes que encaminan al investigador a descubrir, quizá por casualidad, las características especiales de un modelo matemático de almacenamiento de información y el uso que se está dando a estas colecciones de datos por parte de especialistas en el problema de \clasificacion. La aparición del problema del \carm en [\ldots] era previsible, todas las reglas de asociación tienen un aspecto muy simple que sugiere a cualquier investigador que puede ser utilizado en el problema de clasificación. El hecho de que yo, especializado en el problema de asociación, observara los mismos datos que los especialistas en clasificación tendría que llevarnos al mismo resultado si ellos habían alcanzado el óptimo o a un mejor resultado si yo era capaz de aportar ideas sobre cómo utilizar los elementos de ARM.

El primer descubrimiento simple y útil de esta tesis son las \emph{Transacciones Estructuradas} expuestas en la sección~\ref{sec:clasificacion:transacciones-tipo-ii}. Con ellos descubrí que el modo de aplicar técnicas de ARM en los artículos que consultaba no era del todo correcto  [\ldots]. No soy especialista todavía en el Problema de Clasificación por lo que algunas conclusiones de esos artículos y, sobre todo, las pruebas de eficiencia de los algoritmos que proponían, estaban fuera de mi alcance. Se me ocurrió incorporar las restricciones iniciales del problema de clasificación a un problema general de asociación. Los problemas de asociación se resuelven mediante la fuerza bruta leyendo todos los datos que tenemos y mirándolos desde distintas perspectivas, si quiero resolver un problema distinto, un problema de clasificación, usando técnicas de minería de reglas de asociación debería aprovechar, al menos, la rígida estructura de los datasets usados para clasificación (en asociación sólo hay una norma: en un registro no se cuentan los datos repetidos, lo que hace que el número de reglas de asociación que se puede buscar sea tan grande que provoque desbordamiento de memoria en los programas que intentan analizar grandes colecciones de datos). Se me ocurrió que si todos los registros han de tener un valor para cada uno de los atributos en estudio podía reducir el número de datos a procesar y las dimensiones del dataset eliminando únicamente un valor de cada atributo en todo el dataset. Al hacerlo y comprobar que la nueva colección de datos, compresión sin pérdidas de la colección original, sí se podía analizar utilizando el clásico Apriori y obtener todas las reglas de asociación que contenía empecé a asimilar mejor las características de un catálogo.

Primero descubrí características matemáticas, restricciones teóricas que me permitían reducir las dimensiones del problema original y, usando muchos recursos, obtener toda la información que contienen esas pequeñas colecciones de datos en cuanto a reglas de asociación se refiere. Pero tenía que haber algo más, las características matemáticas que utilicé en~\ref{sec:clasificacion:publicaciones:interaccion-2012} me exigían usar muchos recursos y no me ofrecían información demasiado relevante, además seguía necesitando mucha RAM para trabajar con colecciones pequeñas de datos, a pesar de que ya sabía que contenían muchísima información. Quería encontrar mejor información en menos tiempo y usando menos RAM por lo que introduje la STL a mi desarrollo y comprobé en la primera aplicación que la teoría de conjuntos tenía mucho que aportar al análisis de catálogos.
