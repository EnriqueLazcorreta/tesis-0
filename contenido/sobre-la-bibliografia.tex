% !TEX root = ../Lazcorreta.Tesis.tex
\ABIERTO

%La bibliografía de este informe es una selección de todos los artículos que he consultado durante mi investigación. Es muy probable que haya olvidado citar algún artículo que aporte valor a la presente investigación por lo que he añadido en el DVD complementario a este informe una copia de la \db bibliográfica que he gestionado durante estos años. Se trata del fichero \texttt{bib/tesis.bib}.

Todas las citas bibliográficas que figuran en este informe, tanto de trabajos propios como de otros investigadores, contienen enlaces externos que han sido revisados entre noviembre y diciembre de 2015. Para no perder las direcciones utilizadas en la versión impresa se han añadido en forma de notas a pie de página, aunque algunas URLs sean difíciles de reproducir a mano.

Se han utilizado diferente imágenes\footnote{Obtenidas de \href{http://sourceforge.net/projects/openiconlibrary/}{http://sourceforge.net/projects/openiconlibrary/}, una completa librería de iconos e imágenes de código abierto.} para indicar los recursos adicionales sobre la bibliografía utilizada que he ido recopilando a través del proceso de investigación. En la versión impresa aparecen a modo informativo, describiendo visualmente la información adicional que contiene cada elemento bibliográfico.
 
\begin{itemize}
	\item[\iconoPDF] indica que hay una copia del artículo, resumen o presentación en el \dvdAdjunto que complementa esta tesis, en formato \texttt{.pdf}, en alguna de las carpetas de \texttt{bib}. Sólo es útil si se está revisando este documento desde el \dvdAdjunto adjunto o desde una copia de este documento colocada junto a una copia de la carpeta \texttt{bib} del \dvdAdjunto.
  %\item[\iconoDescargaPDF] enlaza a la URL en que se encuentra publicada la referencia, en formato \texttt{.pdf}.
  \item[\iconoDescarga] enlaza a la URL desde la que se puede descargar el artículo.
  \item[\iconoWWW] enlaza a la URL con información sobre la referencia.
  \item[\iconoSCOPUS] enlaza a la URL de \url{http://www.scopus.com} con información sobre la referencia. Requiere acceso identificado, en mi caso proporcionado por mi universidad.
  \item[\logoCIO] enlaza a la URL con el documento publicado como \textsl{Working Paper} para el Instituto Universitario Centro de Investigación Operativa de la Universidad Miguel Hernández de Elche.
  \item[\logoGitHub] enlaza a la URL en que he publicado este informe o la parte de él que se cita.
\end{itemize}

También están en el \dvdAdjunto adjunto las tres bases bibliográficas que he utilizado en este informe, todas ellas en la carpeta \texttt{bib} y con formato \BibTeX. En ellas encontrará el lector más información que la expuesta en este informe. He utilizado los gestores bibliográficos \texttt{JabRef}  \footnote{\includegraphics[height=10pt]{./imagen/JabRef} \url{http://jabref.sourceforge.net/}} y \texttt{BibDesk} \footnote{\includegraphics[height=10pt]{./imagen/bibDeskIcon} \url{http://bibdesk.sourceforge.net/}} pero por tratarse de ficheros de texto plano pueden abrirse con múltiples aplicaciones. Se incluye también la base bibliográfica \texttt{tesis.bib}, en la que se encuentran la mayoría de artículos que he utilizado a lo largo de estos años para documentarme sobre el estado del arte de la materia tratada. No se han incluido en este informe todas estas referencias pero no quería que todos estos autores y artículos se olvidaran pues de ellos extraje mucho conocimiento en su día.

Todos los artículos incluidos en formato \texttt{.pdf} han sido descargados sin usar servicios de pago por lo que asumo que su utilización a través de la edición digital de este informe no infringe ninguna licencia de uso. Personalmente me ha resultado muy productivo tener preparados estos enlaces para elaborar un informe de mayor calidad y espero que al lector le sea útil para tener a mano mucha más información sobre los temas tratados en esta tesis.
