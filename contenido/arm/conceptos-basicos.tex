% !TEX root = ../../Lazcorreta.Tesis.tex
% \ABIERTO%
\citet{AgrawalImielinskiSwami-MiningAssociationRulesBetweenSetsOfItemsInLargeDB-1993} son pioneros en la \arm. La definición formal que dan a a esta disciplina es muy elemental\footnote{He adaptado la nomenclatura a la utilizada en este informe.}:

\selectlanguage{english}
\begin{quote}
  Let $\I = \{I_1, I_2, \ldots I_N\}$ be a set of binary attributes, called items. Let \D be a database of transactions. Each transaction $t$ is represented as a binary vector, with $t[k] = 1$ if $t$ bought the item $I_k$, and $t[k] = 0$ otherwise. There is one tuple in the database for each transaction. Let $X$ be a set of some items in \I. We say that a transaction $t$ satisfies $X$ if for all items $I_k$ in $X$, $t[k] = 1$.
  
  By an association rule, we mean an implication of the form 
  $$X \rightarrow I_j$$
  where $X$ is a set of some items in \I, and $I_j$ is a single item in \I that is not present in $X$. The rule $X \rightarrow I_j$ is satisfied in the set of transactions \D with the confidence factor $0 \leq c \leq 1$ iff at least $c\%$ of transactions in \D that satisfy $X$ also satisfy $I_j$.
  
  Given the set of transactions \D, we are interested in generating all rules that satisfy certain additional constraints of two different forms: Syntantic Constrains and Support Constrains.
\end{quote}
\selectlanguage{spanish}

Definen \I como un conjunto de $N$ atributos binarios y \D como una \db de \transacciones, vectores binarios de $N$ elementos cuyo valor $k$-ésimo indica si el ítem $I_k$ está o no presente en la \transaccion. El \consecuente de una \AR es un ítem de \I no presente en el \antecedente de la regla. Las investigaciones posteriores sobre este tema amplían esta definición para estudiar reglas en que el \consecuente sea un \kitemset, como se indica en la definición~\ref{def:1-3-2-AR}. Por último definen la \arm como la disciplina que busca en un gran almacén \D las \ARs que cumplen ciertas restricciones, bien de tipo sintáctico (como la presencia de cierto ítem en el antecedente o el consecuente) o de \soporte (obteniendo únicamente las reglas más frecuentes).

Las restricciones no son imposiciones si no necesidades. Si \I está formado por $N$ ítems, para tratar una transacción tendremos que utilizar $N$ valores (ceros o unos), y se podrán obtener hasta $2^N$ \itemsets distintos\footnote{Si $N=100$ tendremos $2^{100}\approx1.3\cdot10^{30}$ posibles \itemsets en las \transacciones si no contamos repeticiones, lo que da una idea del número de relaciones que se pueden encontrar entre los \itemsets de \D cuando éste es muy grande y el número de ítems distintos también lo es.}, pudiendo actuar como \antecedente de una \ar $2^N-2$ al descontar $\emptyset$ e \I. Considerando que cada \kitemset[1] puede ser el \antecedente de hasta $2^{N-1}-1$ \ars distintas, cada \kitemset[2] puede ser el \antecedente de $2^{N-2}-1$ reglas distintas\ldots es fácil descubrir que el número de \ARs que se podría obtener es enorme: $\sum_{i=1}^{N-1}{{N \choose i}(2^{N-i}-1)}$. Como la mayoría de datos se ha de manipular en memoria RAM para poder acceder a ellos rápidamente, sólo si trabajamos con pocos ítems y pocas \transacciones podremos evitar las restricciones.



\subsection{Tipo de Datos}
\label{sec:arm:tipo-de-datos}
% !TEX root = ../../../Lazcorreta.Tesis.tex
% \ABIERTO%
Al definir las transacciones como vectores binarios se puede interpretar que es mejor guardar los datos a nivel de bit ya que sólo son necesarios dos valores para representar cada dato, lo que convertiría a \D en una matriz de ceros y unos cuyas $|\D|$ filas representan cada una de las \transacciones y sus $N$ columnas cada uno de los ítems de \I. Es fácil pensar en trabajar con matrices de estas características para aprovechar la potencia de los ordenadores al trabajar a nivel de bit, sin embargo no se plantea nadie este formato hasta el trabajo de~\citet{DongHan-BitTableFIEfficientFIMalg-2007}, que proponen comprimir \D en una {BitTable} y usar el algoritmo~\algoritmo{BitTableFI}. Indican que esta estructura posibilita una rápida generación de candidatos y de su recuento por utilizar funciones de unión e intersección de bits, obteniendo mejor rendimiento que los algoritmos con que se compara. En la sección~\ref{sec:4-bit} volveremos a este planteamiento como propuesta de trabajo futuro.

Trabajar con uniones e intersecciones de bits puede ser muy eficiente para un procesador si se consigue programar de forma eficiente, sin embargo al desarrollar un programa con un lenguaje de alto nivel no es fácil trabajar a nivel de bit (en \langCpp se guarda un valor \texttt{bool} usando un Byte, ocho bits) por lo que la mayoría de implementaciones hechas sobre algoritmos de \arm no utilizarán esta definición de \I. Generalmente se representa el ítem $I_k$ utilizando el número $k$ con lo que las \transacciones no se guardan como $N$-tuplas de ceros y unos si no como conjuntos de números enteros. El formato más utilizado para guardar \D de este modo es escribiendo en un fichero de texto plano cada \transaccion en una línea del fichero, lo que se conoce como \emph{representación horizontal de \D}. Algunos algoritmos están especialmente diseñados para aprovechar la \emph{representación vertical de \D}, en que en cada línea se escribe el par $\TID\ item$ de modo que cada \transaccion ocupe tantas líneas consecutivas como ítems contenga. A lo largo de este informe se tratará \D como un conjunto de números enteros y no una matriz de bits.




\subsection{Primeros Algoritmos}
\label{sec:arm:primeros-algoritmos}
% !TEX root = ../../../Lazcorreta.Tesis.tex
% \ABIERTO%
En este primer artículo proponen el algoritmo \algoritmo{AIS} (ver algoritmo~\ref{alg:AIS}) que se basa en múltiples lecturas de la \db de transacciones \D. Comienzan con un conjunto vacío de \itemsets frecuentes, \aprioriL, y un conjunto de \itemsets al que llaman frontera, $\mathcal{F}$, y que contiene inicialmente el conjunto vacío. En cada lectura de \D, $k=1\ldots$, se realizan los siguientes pasos:

\begin{enumerate}
  \item Se vacía el conjunto de candidatos, \aprioriC.
  \item Se busca cada \kitemset[k-1] de la frontera en cada \transaccion.
  \item Si se encuentra se extiende con todos los ítems de la \transaccion y se añade la extensión al conjunto \aprioriC[k+1], creándolo con \soporte 1 si no existe o incrementando su \soporte si ya existe.
  \item Al terminar la $k$-ésima lectura de \D se vacía la frontera, se guardan en \aprioriL[k] los candidatos que tienen \soporte mínimo y se añaden a la frontera los que se estima que se podrán extender en la siguiente iteración.
  \item Si la frontera no está vacía se vuelve al paso 1.
\end{enumerate}

Empiezan a encontrarse los problemas de recursos intrínsecos a esta disciplina, llegando a proponer un algoritmo que guarda en disco la información que está en memoria y no es necesaria. Este algoritmo se llamará cada vez que se necesite memoria para almacenar nuevos datos, lo que restará algo de eficiencia al proceso completo.

Proponen también una serie de heurísticas para estimar el \soporte de un \kitemset antes de generar su candidato, de modo que si se estima que tendrá un \soporte bajo no se genere el candidato y el sistema no caiga por falta de memoria para almacenar los candidatos. La generación de candidatos se hace bajo la suposición de independencia de los ítems presentes en cada transacción. Si los \itemsets $X$ e $Y$ son disjuntos y tienen \soporte $x$ e $y$ respectivamente, estiman que el \itemset $X \cup Y$ tendrá \soporte $z = x\cdot y$, por lo que esperan que será frecuente en \D si $z \geq minSup$.

% \afterpage{\clearpage}
\lstinputlisting[label=alg:AIS,
                 caption={Algoritmo {AIS}, 1993},
                 float=htbp,
                 basicstyle=\scriptsize]
                {./contenido/arm/codigo/alg-AIS}%

\citet{HoutsmaSwami-SETMofAR-1993} sugieren trabajar directamente sobre las \dbs de \transacciones aprovechando las funciones propias de los \dbms en lugar de desarrollar aplicaciones específicas para trabajar con almacenes \D guardados como texto plano. Proponen el algoritmo \algoritmo{SETM} (ver algoritmo~\ref{alg:SETM}), que genera los candidatos al leerlos en las transacciones, igual que \algoritmo{AIS}, y guarda una copia de cada \itemset junto al \TID de la transacción que lo contiene. Al trabajar con \DB en disco no necesita muchos recursos de memoria para guardar tanta información pero no puede competir en eficiencia con programas preparados específicamente para extraer la misma información de un fichero de texto plano. Sin embargo la asociación de cada \itemset con el \TID de la transacción en que está contenido puede agilizar un estudio más profundo de ciertos \itemsets por lo que no descartamos el uso de esta idea cuando podamos realizar en paralelo estas sentencias que no benefician la eficiencia del propósito actual, encontrar los \itemsets frecuentes presentes en \D, pero cuyos resultados pueden ser guardados en ficheros cuyo análisis posterior puede proporcionar gran cantidad de conocimiento sobre la población en estudio. 

% \afterpage{\clearpage}
\lstinputlisting[label=alg:SETM,
                 caption={Algoritmo {SETM}, 1993},
                 float=htbp,
                 basicstyle=\scriptsize]
                {./contenido/arm/codigo/alg-SETM}

Como se está buscando la co-existencia de ítems en \transacciones se puede aprovechar el orden lexicográfico del código utilizado para representar a cada ítem. El \soporte del \itemset $XY$ coincide con el de $YX$ por lo que si al guardar los \itemsets lo hacemos con sus ítems ordenados lexicográficamente sólo guardaremos el \itemset $XY$. Este orden se va a aprovechar para hacer más eficientes las operaciones a realizar por el algoritmo.

\lstinputlisting[label=alg:SETM-mergeScan,
                 caption={Algoritmo {SETM}, función merge-scan},
                 float=htbp,
                 basicstyle=\scriptsize]
                {./contenido/arm/codigo/alg-SETM-mergeScan}
                
Las funciones del algoritmo son consultas a la \db. Para obtener $R'_k$ se ejecuta la consulta del listado~\ref{alg:SETM-mergeScan}, que extiende cada \kitemset frecuente encontrado en el paso anterior con los ítems frecuentes de \D que sean lexicográficamente mayores que el mayor de los ítems del \kitemset. Concretamente en la extensión de un \kitemset basta con añadir uno a uno los ítems de la \transaccion que contienen al \kitemset y son lexicográficamente mayores al mayor de los ítems del \kitemset.


\aprioriC se obtiene contando los \itemsets del conjunto $R'_k$ y guardando sólo los que tienen \soporte mínimo, como muestra el listado~\ref{alg:SETM-Ck}.

\lstinputlisting[label=alg:SETM-Ck,
                 caption={Algoritmo {SETM}, selección de candidatos},
                 float=htbp,
                 basicstyle=\scriptsize]
                {./contenido/arm/codigo/alg-SETM-Ck}

El conjunto auxiliar $R_k$ lo obtienen seleccionando las tuplas de $R'_k$ que pueden ser extendidas, como muestra el listado~\ref{alg:SETM-Rk}.
\lstinputlisting[label=alg:SETM-Rk,
                 caption={Algoritmo {SETM}, \itemsets extendibles},
                 float=htbp,
                 basicstyle=\scriptsize]
                {./contenido/arm/codigo/alg-SETM-Rk}

Estos primeros algoritmos de \arm utilizan la misma estrategia para evitar desbordamiento de memoria al generar los candidatos a \itemset frecuente. Como el número de candidatos es teóricamente muy grande cuando trabajamos con grandes almacenes \D y muchos ítems en \I, y como es de esperar que el número de \itemsets en \D sea sensiblemente menor que el teórico, en estos algoritmos se propone reservar memoria únicamente para contar los \itemsets que realmente están en \D, es decir, se reserva memoria cada vez que se encuentra un \itemset en \D para el que aún no hemos hecho dicha reserva. Este planteamiento consigue su objetivo a costa de eficiencia, ya cada una de las millones de operaciones de reserva de memoria que se llevan a cabo consume un tiempo y su acumulación resulta en un tiempo de ejecución muy superior al que conseguiríamos si hiciéramos una buena estimación de los \itemsets contenidos en \D y realizáramos la reserva de memoria en una única instrucción. En los algoritmos que se muestran en la sección~\ref{sec:arm:fim} se adoptará esta última estrategia.

En ambos artículos se trata por encima la generación de \ars. En su definición se especifica que el consecuente es un único ítem por lo que una vez conocemos todos los \kitemsets frecuentes basta con recorrerlos uno a uno y separarlos en dos \itemsets, el antecedente con $k-1$ ítems y el consecuente con el ítem restante, con lo que se obtendrán todas las \ARs de \D que tienen \soporte mínimo.



\subsection{Formato de \D}
\label{sec:arm:formato-de-D}
% !TEX root = ../../../Lazcorreta.Tesis.tex
% \ABIERTO%
En la sección~\ref{sec:arm:tipo-de-datos} hemos cuestionado el modo de guardar los datos de \D para el análisis, adelantando que la codificación de los ítems de \I mediante números enteros es el formato más utilizado por los desarrolladores de algoritmos de \ARM. \citet{AgrawalImielinskiSwami-MiningAssociationRulesBetweenSetsOfItemsInLargeDB-1993} no especifican nada sobre este asunto en su artículo, mientras que~\citet{HoutsmaSwami-SETMofAR-1993} proponen el uso de funciones ya implementadas en los \dbmss (\DBMS). Como la \arm es una fase del proceso de \KDD lo más inmediato es usar \dbs para guardar \D, sin embargo la eficiencia de las funciones de un \DBMS es notablemente inferior a la obtenida usando lenguajes de programación compilados si son muchos los datos a procesar.

\citet{AgrawalImielinskiSwami-DatabaseMiningAPerformancePerspective-1993} proponen dotar a los \DBMS de nuevas operaciones bien implementadas para poder hacer \dm sobre \clasificacion, {asociación} y \secuencias directamente sobre la \DB utilizando combinaciones de dichas operaciones para obtener \patrones válidos para los tres modelos abordados.

Hay más investigadores que buscan el uso de \DBMS para generar \ARs, como~\citet{HoutsmaSwami-SetOrientedMiningForAR-1995} que proponen el estudio de \ARM mediante lenguajes nativos de \dbs, SQL concretamente.

Otro trabajo que busca el uso de \DBMS para extraer reglas de asociación es el de \citet{HolsheimerKerstenMannilaToivonen-APerspectiveOnDatabasesAndDataMining-1995}. En su favor está la flexibilidad que permiten los \DBMS para tratar los datos desde múltiples perspectivas, en su contra el tiempo necesario para obtener resultados frente a los algoritmos específicos de \ARM desarrollados mediante lenguajes de programación compilado. Una de las propuestas de este artículo es el uso de la jerarquía de los ítems para reducir el tamaño del repositorio a analizar, concretamente hablan de la cesta de la compra y de descubrir reglas que involucren genéricamente al ítem "`cerveza"' en lugar de usar como ítem cada una de las marcas de cerveza que están en \I, idea que será tomada por otros investigadores para reducir las dimensiones del problema y poder llevar a cabo estudios más amplios.

Aunque los algoritmos presentados de forma teórica a lo largo de esta investigación no especifican el formato físico de los datos a procesar, las comparaciones realizadas sobre los distintos algoritmos no sería correcta si no se experimenta en condiciones similares por lo que en la mayoría de los casos encontraremos implementaciones realizadas con lenguajes de programación compilados. De hecho, los almacenes \D disponibles en \url{http://fimi.ua.ac.be/data/} y \url{https://archive.ics.uci.edu/ml/datasets.html} tienen formato de texto plano, la mayoría de ellos en formato horizontal.

% En estos primeros trabajos se generan los candidatos conforme se va leyendo las \transacciones de \D, lo que supone una gestión de memoria muy dinámica. Conforme se van incrementando las


\subsection{Fases de \ARM}
\label{sec:arm:fases-de-ARM}
% !TEX root = ../../../Lazcorreta.Tesis.tex
% \ABIERTO%

Como se ha visto en la sección anterior y exponen \citeauthor{AgrawalImielinskiSwami-MiningAssociationRulesBetweenSetsOfItemsInLargeDB-1993} en su primer artículo, la \ARM se compone de dos fases bien diferenciadas:
\begin{enumerate}
  \item En primer lugar hay que encontrar los conjuntos de ítems que están presentes en un porcentaje mínimo de \transacciones de \D.
  \item A continuación se descubren las \ARs que se deducen de esos conjuntos.
\end{enumerate}

La primera fase es la que mayores aportaciones científicas ha recibido en la \ARM pues al trabajar con grandes colecciones de \transacciones puede consumir más recursos de los disponibles o tardar más tiempo del adecuado para dar por finalizado el análisis. Cabe destacar que la tecnología utilizada para el desarrollo de los algoritmos expuestos en esta sección ha evolucionado notablemente desde los inicios de esta metodología, las comparaciones realizadas a través de los artículos presentados pueden haber variado debido a que actualmente tenemos a nuestra disposición recursos de memoria y de cómputo muy superiores a los utilizados en las dos décadas precedentes. En la sección~\ref{sec:arm:fim} se estudian las propuestas más interesantes sobre esta fase, \fim, y se muestra una comparativa sobre la implementación del algoritmo más utilizado en la \arm.

La segunda fase es elemental y no consume tiempo ni recursos excesivos en comparación con la primera por lo que no ha habido grandes avances desde sus inicios excepto en el estudio de las métricas que mejores reglas de asociación proporcionan, dado que el número de reglas generadas puede ser demasiado grande para ser analizado correctamente. En la sección~\ref{sec:arm:generacion-ar} se aborda esta fase, la generación de \ars a partir del conjunto de \itemsets frecuentes, y se presenta una propuesta para ejecutarla de un modo más eficiente, propuesta que forma parte de nuestro artículo más citado.

