% !TEX root = ../../../Lazcorreta.Tesis.tex
\ABIERTO

Las \ROPs son \ARs interpretadas en sentido inverso. Pueden ser útiles cuando aparece el \dilemaIR en nuestra colección de datos. El principal motivo de la aparición de este problema son los recursos de memoria utilizados, si tenemos que guardar mucha información sobre muchos datos podemos llegar a desbordar la memoria del sistema y perder todo el trabajo hecho hasta el momento. Hay muchas propuestas para aliviar este problema, todas ellas buscando perder cierta información que por algún motivo se va a considerar irrelevante a cambio de ganar espacio en memoria para poder analizar otros datos de la colección con información más relevante.
























En un \srw se utiliza información sobre taxonomías, contenido semántico y uso del portal para hacer sugerencias a sus usuarios. Sin embargo cuando un ítem es nuevo no puede obtenerse información de su uso hasta que no logra un cierto \soporte. Las \ROPs permiten incorporar en el sistema la información de uso de los ítems nuevos de modo automático.

Las \rops contienen los ítems frecuentes que son visitados conjuntamente con los ítems que no alcanzan el \soporte mínimo. Esto nos permite sugerir un \ir a los usuarios que visitan alguna de las páginas que más favorecen la presencia de dicho ítem, p.ej. la página $A$.

\begin{itemize}
  \item Si la sugerencia es acertada es de esperar que el \ir vaya aumentando su \soporte y se incorpore de modo natural al proceso de estudio de \ars, mostrando una asociación vez más fuerte a la página $A$.
  \item Si la sugerencia no es acertada el ítem seguirá siendo raro y su asociación con la página $A$ se irá diluyendo a favor de una asociación con otras páginas con las que realmente sea más afín. Esto se debe a que el número de veces que aparece el \ir en \D es pequeño y cualquier aparición nueva del ítem modificará sustancialmente los porcentajes en que se basan los algoritmos de búsqueda de \ARs y \ROPs.
\end{itemize}

