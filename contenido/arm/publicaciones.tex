% !TEX root = ../../Lazcorreta.Tesis.tex
El trabajo expuesto en la sección~\ref{sec:arm:fim:evaluacion-de-diferentes-implementaciones} se presentó en Beijing (\urlConNotaAlPie{http://www.hci.international/index.php?module=conference&CF_op=view&CF_id=5}{HCII'07}). La investigación mostrada en la sección~\ref{sec:arm:ar:apriori2} dio lugar a la publicación de un artículo en la revista \urlConNotaAlPie{http://www.journals.elsevier.com/expert-systems-with-applications/}{ESWA}, lo que posibilitó mucho más la difusión de nuestro trabajo. Por último presentamos en Barcelona (\urlConNotaAlPie{http://interaccion2009.aipo.es}{Interacción'09}) el trabajo expuesto en la sección~\ref{sec:arm:el-item-raro}.





\phantomsection
\subsection*{Selecting the Best Tailored Algorithm for Personalizing a Web Site, 2007}
\addcontentsline{toc}{subsection}{Actas de {HCII}'07}
\label{sec:nuestro-Selecting-2007}
En este artículo presentamos los resultados obtenidos al comparar diferentes implementaciones del algoritmo \apriori. Para seguir en la línea que estamos investigando hemos de crear nuestra propia implementación si queremos introducir los resultados obtenidos hasta la fecha y los que puedan ir surgiendo en el curso de la investigación. El código de las implementaciones puestas a prueba nos resultará útil para trabajar en nuestra propia implementación.

\begin{quote}
  Botella Beviá, F., Lazcorreta Puigmartí, E., Fernández-Caballero, A., González López, P., Gallud,
J.A. y Bia Platas, A. Selecting the Best Tailored Algorithm for Personalizing a Web Site. \emph{Proceedings of the 12th International Conference on Human-Computer Interaction ({HCII})}, Beijing (China), 2007. \leePDF{bib/nuestros/BotellaLazcorretaFCaballeroGonzalezGalludBia-SelectingTheBestTailoredAlgForPersonalAWebSite-2007.pdf}
\end{quote}

\begin{quotation}
	\noindent\textbf{Resumen}

\selectlanguage{english}
	\nopagebreak Automatic personalization for dynamic web systems has been carried out by several methods and techniques, like data mining or Web usage mining. The \apriori algorithm is one of the most used for web personalization. We have designed a web recommender system based on this algorithm to select the best user-tailored links. We needed an efficient algorithm to preserve updated our system in real time. We tested several implementations of \apriori algorithm but none of them looked to run properly with our data. We tested these implementations with different access log files of public domain (BMSWebView.dat, BMS-POS) and with our log data. We decided to develop a new implementation that better adapts to our data.
\selectlanguage{spanish}
\end{quotation}








\phantomsection
\subsection*{Towards personalized recommendation by two-step modified \apriori data mining algorithm, 2008}
\addcontentsline{toc}{subsection}{{ESWA}, vol. 35(3) 2008}
\label{sec:nuestro-Torwards-2008}

El artículo
\begin{quote}%\citet{Lazcorreta20081422}
  Lazcorreta Puigmartí, E., Botella Beviá, F. y Fernández-Caballero, A. Towards personalized recommendation by two-step modified \apriori data mining algorithm. {\em Expert Systems with Applications}, 35(3):1422--1429, 2008. \leePDF{bib/nuestros/LazcorretaBotellaFCaballero-TowardsPR-CIO-2008.pdf} \descargaCIO{http://cio.umh.es/files/2011/12/CIO_2007_26.pdf} \descarga{https://www.researchgate.net/publication/222690209_Towards_personalized_recommendation_by_two-step_modified_Apriori_data_mining_algorithm?ev=pub_cit}
\end{quote}

	\begin{quotation}
	\noindent\textbf{Resumen}

\selectlanguage{english}
	\nopagebreak In this paper a new method towards automatic personalized recommendation based on the behavior of a single user in accordance with all other users in web-based information systems is introduced. The proposal applies a modified version of the well-known \apriori data mining algorithm to the log files of a web site (primarily, an e-commerce or an e-learning site) to help the users to the selection of the best user-tailored links. The paper mainly analyzes the process of discovering association rules in this kind of big repositories and of transforming them into user-adapted recommendations by the two-step modified \apriori technique, which may be described as follows. A first pass of the modified \apriori algorithm verifies the existence of association rules in order to obtain a new repository of transactions that reflect the observed rules. A second pass of the proposed \apriori mechanism aims in discovering the rules that are really inter-associated. This way the behavior of a user is not determined by "what he does" but by "how he does". Furthermore, an efficient implementation has been performed to obtain results in real-time. As soon as a user closes his session in the web system, all data are recalculated to take the recent interaction into account for the next recommendations. Early results have shown that it is possible to run this model in web sites of medium size.
\selectlanguage{spanish}
	\end{quotation}

%ha sido citado en las siguientes publicaciones\footnote{Todos los enlaces han sido revisados en noviembre de 2014. Los que enlazan a \url{http://www.scopus.com} requieren identificación, puede verse un resumen de estas publicaciones en \scopus[7pt]{http://www.scopus.com/results/citedbyresults.url?sort=plf-f&cite=2-s2.0-44949263923&src=s&imp=t&sid=D17660FCC894527FCCD93BAF74F677F3.ZmAySxCHIBxxTXbnsoe5w\%3a1280&sot=cite&sdt=a&sl=0&origin=recordpage&txGid=D17660FCC894527FCCD93BAF74F677F3.ZmAySxCHIBxxTXbnsoe5w\%3a128} y en el fichero \texttt{bib/NosCitan/scopus.bib}, en formato \BibTeX, del \dvdAdjunto que complementa este informe. En \url{http://scholar.google.es/scholar?oi=bibs&hl=es&cites=15432134191557693855} nos atribuyen 45 citas.}.
%
%
%
%%\subsection*{Artículos que nos citan}
%
%\selectlanguage{english}
%\begin{enumerate}
%
%
%
%  %2008
%  
%  %~\nocite{RasteragiMdSap_DataMiningAndECommerce_2008}
%	\item Hamid Rastegari y Mohd.~Noor Md.~Sap. Data mining and e-commerce : methods, applications, and challenges. \emph{Jurnal Teknologi Maklumat}, 20:116--128, 2008. \leePDF{bib/NosCitan/RasteragiMdSap_DataMiningAndECommerce_2008.pdf} \descarga{http://eprints.utm.my/9427/1/MohdNoorMd2008_DataMiningAndE-Commerce.pdf}
%
%	\item Kittisak Kerdprasop y Nittaya Kerdprasop. A Rough Set Approach to Personalization in 
%Web-Based Learning Systems. In \emph{Proceedings of {IADIS} International Conference e-Learning}, pages 22--25, july 2008. \leePDF{bib/NosCitan/KerdprasopKerdprasop_ARoughSetApproachToPersonalizationInWebBasedSystems_2008.pdf} \descarga{https://www.researchgate.net/publication/220970028_A_Rough_Set_Approach_To_Personalization_In_Web-Based_Learning_Systems}
%
%
%
%
%  %2009
%
%  %~\nocite{LiXuWangChu_StrongestAssociationRulesMiningForPersonalizedRecommendation_2009}
%	\item J.~Li, Y.~Xu, Y.~Wang y C.~Chu. Strongest association rules mining for personalized recommendation. \emph{Xitong Gongcheng Lilun yu Shijian/System Engineering Theory and Practice}, 29(8):144--152, 2009.
%    %\leePDF{bib/NosCitan/LiXuWangChu_StrongestAssociationRulesMiningForPersonalizedRecommendation_2009.pdf}
%    \descarga{https://www.researchgate.net/publication/251709116_Strongest_Association_Rules_Mining_for_Personalized_Recommendation} \scopus{http://www.scopus.com/inward/record.url?eid=2-s2.0-70249125844&partnerID=40&md5=9b944e88ca9e796c4b539bda50d0f24e}
%
%  %~\nocite{KahramanliAllahverdi_NewMethodForComposingClassificationRulesARplusOPTBP_2009}
%	\item Humar Kahramanli y Novruz Allahverdi. A new method for composing classification rules: {AR+OPTBP}. In \emph{Proceedings of the 5th International Advanced Technologies Symposium (IATS'09)}, pages 1--6, may 2009. \leePDF{bib/NosCitan/KahramanliAllahverdi_NewMethodForComposingClassificationRulesARplusOPTBP_2009.pdf} \descarga{http://iats09.karabuk.edu.tr/press/bildiriler_pdf/IATS09_01-01_105.pdf}
%
%  %\nocite{ClewleyChenLiu_ApplicationsForDMTechniquesInCRM_2009}
%  \item N.~Clewley, S.Y.~Chen y X.~Liu \emph{Encyclopedia of Information Science and Technology 2nd edition}, 188--192, Idea Group Publishing, 2009. \leePDF{bib/NosCitan/ClewleyChenLiu_ApplicationsForDMTechniquesInCRM_2009.pdf} \descarga{http://www.irma-international.org/viewtitle/13571/} \visita{http://what-when-how.com/information-science-and-technology/applications-for-data-mining-techniques-in-customer-relationship-management-information-science/}
%  
%  \item L.~Jie, X.~Yong, Y.~Wang y C.~Chu \emph{Strongest association rules mining for personalized recommendation}. In \emph{Systems Engineering-Theory \& Practice}, vol. 29 issue 8, pgs. 144--152, aug 2009. \leePDF{bib/NosCitan/JieYongWangChu-StrongestARMforPersonalizedRecommendation-2009.pdf}
%  
%  
%  
%  
%  %2010
%
%  %~\nocite{Kirkos2010193}
%	\item E.~Kirkos. \emph{Modeling the auditors'{} opinions by using association rules}. Nova Science Publishers, Inc., 2010. \scopus{http://www.scopus.com/inward/record.url?eid=2-s2.0-84892052266&partnerID=40&md5=1f19c4a0e9edf4ffe238cedbaee5c05e}
%
%  %~\nocite{MarkowskaKwasnickaParadowski_IntelligentTechniquesInPersonalizationOfLearningInELearningSystems_2010}
%	\item U.~Markowska-Kaczmar, H.~Kwasnicka y M.~Paradowski. Intelligent techniques in personalization of learning in e-learning systems. \emph{Studies in Computational Intelligence}, 273:1--23, 2010. \leePDF{bib/NosCitan/MarkowskaKwasnickaParadowski_IntelligentTechniquesInPersonalizationOfLearningInELearningSystems_2010.pdf} \descarga{http://www.researchgate.net/publication/202144697_Intelligent_Techniques_in_Personalization_of_Learning_in_e-Learning_Systems/links/02bfe50e46c68dbb7f000000} \scopus{http://www.scopus.com/inward/record.url?eid=2-s2.0-77950283870&partnerID=40&md5=52470b95e2e46e28f0f33ccf2da79d1b}
%
%  %~\nocite{YuZhou_ParallelTIDBasedFrequentPatternMiningAlgorithmOnAPCClusterAndGridComputingSystem_2010}
%  \item K.-M.~Yu y J.~Zhou. Parallel tid-based frequent pattern mining algorithm on a pc cluster and grid computing system. \emph{Expert Systems with Applications}, 37(3):2486--2494, 2010.
%   % \leePDF{bib/NosCitan/YuZhou_ParallelTIDBasedFrequentPatternMiningAlgorithmOnAPCClusterAndGridComputingSystem_2010.pdf}
%   \scopus{http://www.scopus.com/inward/record.url?eid=2-s2.0-70449535621&partnerID=40&md5=271da43f8d3775ab73eab6e088791e11}
%
%  %~\nocite{ZafraVentura_WebUsageMiningForImprovingStudentsPerformanceInLearningManagementSystems_2010}
%  \item A.~Zafra y S.~Ventura. Web usage mining for improving students performance in learning
%  management systems. \emph{Lecture Notes in Computer Science (including subseries Lecture Notes in Artificial Intelligence and Lecture Notes in Bioinformatics)}, 6098(PART 3):439--449, 2010. \leePDF{bib/NosCitan/ZafraVentura_WebUsageMiningForImprovingStudentsPerformanceInLearningManagementSystems_2010.pdf} \descarga{http://sci2s.ugr.es/keel/pdf/keel/congreso/60980439.pdf} \scopus{http://www.scopus.com/inward/record.url?eid=2-s2.0-79551507673&partnerID=40&md5=dd539d3a93187ea49a7b9ff60f5257c1}
%
%  \item Reza~Samizadeh y Babak~Ghelichkhani. Use of Semantic Similarity and Web Usage Mining to Alleviate the Drawbacks of User-Based Collaborative Filtering Recommender Systems. \emph{International Journal of Industrial Engineering \& Production Research}, 23(3):137--146, sept 2010.
%\leePDF{bib/NosCitan/SamizadehGhelichkhani_UseOfSemanticSimilarityAndWUM_2010.pdf} \descarga{https://www.researchgate.net/publication/49606961_Use_of_Semantic_Similarity_and_Web_Usage_Mining_to_Alleviate_the_Drawbacks_of_User-Based_Collaborative_Filtering_Recommender_Systems}
%
%  %~\nocite{Zhou2010435}
%  \item J.~Zhou, K.-M.~Yu y B.-C.~Wu. Parallel frequent patterns mining algorithm on {GPU}. \emph{Conference Proceedings - IEEE International Conference on Systems, Man and Cybernetics}, 435--440, Istanbul, 2010. \scopus{http://www.scopus.com/inward/record.url?eid=2-s2.0-78751550624&partnerID=40&md5=bc8834c1550b6485fc0cae7a2bf31643}
%  
%  \item J.~Pinho Lucas \emph{Métodos de clasificación basados en asociación aplicados a sistemas de recomendación}, Tesis doctoral de la Universidad de Salamanca, octubre de 2010. \leePDF{bib/NosCitan/Lucas-MetodosDeClasificacion-2010-Tesis.pdf}
%  
%  
%  
%  
%  %2011
%
%  %~\nocite{Acharya:Modi:AlgForFindingFIBasedOnLatticeApproach:11}
%  \item Ajay Acharya y Shweta Modi. An algorithm for finding frequent \itemset based on lattice approach for lower cardinality dense and sparse dataset. \emph{International Journal on Computer Science and Engineering}, 3(1):371--378, jan 2011. \leePDF{bib/NosCitan/AcharyaModi_AlgForFindingFIBasedOnLatticeApproach_2011.pdf} \descarga{http://www.enggjournals.com/ijcse/doc/IJCSE11-03-01-121.pdf}
%
%  \item N.~Kerdprasop \emph{The Handbook of Emergent Technologies in Social Research}, chap. 18:412--434, Oxford University Press, 2011. \visita{https://books.google.es/books?id=Q9HlpMF7GgkC}
%
%  %~\nocite{Afify2011360}
%  \item A.A.~Afify. Discovery of association rules from manufacturing data. \emph{International Journal of Computer Aided Engineering and Technology}, 3(3-4):360--371, 2011. \visita{https://www.deepdyve.com/lp/inderscience-publishers/discovery-of-association-rules-from-manufacturing-data-C0PG4I07B4}  \scopus{http://www.scopus.com/inward/record.url?eid=2-s2.0-84869832476&partnerID=40&md5=e6dfc12e7feba592ac51d80cba44e46a}
%
%  %~\nocite{Cai2011117}
%  \item W.~Cai, D.~Vassalos, D.~Konovessis y G.~Mermiris. Safety (total risk) management for passenger ships - learning from the past, managing the future risk. \emph{Proceedings of the International Conference on Design and Operation of Passenger Ships}, 117--126, London, 2011. \scopus{http://www.scopus.com/inward/record.url?eid=2-s2.0-80052598465&partnerID=40&md5=e08c2909c56375417f7ee64c25b092cb}
%
%  %~\nocite{HuangLuDuan_MiningAssociationRulesToSupportResourceAllocationInBusinessProcessManagement_2011}
%  \item Z.~Huang, X.~Lu y H.~Duan. Mining association rules to support resource allocation in business process management. \emph{Expert Systems with Applications}, 38(8):9483--9490, 2011.
%   % \leePDF{bib/NosCitan/HuangLuDuan_MiningAssociationRulesToSupportResourceAllocationInBusinessProcessManagement_2011.pdf}
%   \scopus{http://www.scopus.com/inward/record.url?eid=2-s2.0-79953688532&partnerID=40&md5=d618a2f9133695ad3e419add6f4f5c0a}
%
%  %~\nocite{Li20111067}
%  \item M.~Li, Z.~Ge, Z.~Yu, Y.~Qi, K.~Liu y H.~Guo. Root-soil system domain knowledge constructing method based on relational analysis of single impact factor. \emph{Proceedings of the 2nd International Conference on Digital Manufacturing and Automation (ICDMA)}, pages 1067--1070, Zhangjiajie, Hunan, 2011. \scopus{http://www.scopus.com/inward/record.url?eid=2-s2.0-80855140462&partnerID=40&md5=94f65d480b6da285c1fa07f0b90f4b1a}
%
%  %~\nocite{LiLiuLi_AnApproachToExpertRecommendationBasedOnFuzzyLinguisticMethodAndFuzzyTextClassificationInKnowledgeManagementSystems_2011}
%  \item M.~Li, L.~Liu y C.-B. Li. An approach to expert recommendation based on fuzzy linguistic method and fuzzy text classification in knowledge management systems. \emph{Expert Systems with Applications}, 38(7):8586--8596, 2011.
%   % \leePDF{bib/NosCitan/LiLiuLi_AnApproachToExpertRecommendationBasedOnFuzzyLinguisticMethodAndFuzzyTextClassificationInKnowledgeManagementSystems_2011.pdf}
%   \scopus{http://www.scopus.com/inward/record.url?eid=2-s2.0-79952438361&partnerID=40&md5=06a98c8a9ad710ee80c45cc5248e655c}
%
%  %~\nocite{ZafraRomeroVentura_MultipleInstanceLearningForClassifyingStudentsInLearningManagementSystems_2011}
%  \item A.~Zafra, C.~Romero y S.~Ventura. Multiple instance learning for classifying students in learning management systems. \emph{Expert Systems with Applications}, 38(12):15020--15031, 2011. \leePDF{bib/NosCitan/ZafraRomeroVentura_MultipleInstanceLearningForClassifyingStudentsInLearningManagementSystems_2011.pdf} \descarga{http://sci2s.ugr.es/keel/pdf/keel/articulo/FinalVersionPublished.pdf} \scopus{http://www.scopus.com/inward/record.url?eid=2-s2.0-80052021793&partnerID=40&md5=2a24a577493e65ba172586aa59587354}
%
%  %~\nocite{Zhong_TheResearchAndApplicationOfWebLogMiningBasedOnThePlatformWeka_2011}
%  \item X.-Y. Zhong. The research and application of web log mining based on the platform weka. \emph{Proceedings of the International Conference on Advanced in Control Engineering and Information Science (CEIS), Procedia Engineering} vol.~15:4073--4078, Dali, Yunnam, 2011.
%   % \leePDF{bib/NosCitan/Zhong_TheResearchAndApplicationOfWebLogMiningBasedOnThePlatformWeka_2011.pdf} \descarga{http://ac.els-cdn.com/S187770581102265X/1-s2.0-S187770581102265X-main.pdf?_tid=d2a3dc4a-69cc-11e4-8509-00000aab0f6c&acdnat=1415729095_1d0d2963e8c7b9c11ae94f1a9a2ed3f3}
%   \scopus{http://www.scopus.com/inward/record.url?eid=2-s2.0-84055223551&partnerID=40&md5=a1ccccbe07b3047a9568d49a911807e1}
%
%  %~\nocite{Zuhtuogullari201139}
%  \item K.~Zuhtuogullari y N.~Allahverdi. An improved \itemset generation approach for mining medical databases. \emph{International Symposium on INnovations in Intelligent SysTems and Applications (INISTA)}, 39--43, Istanbul-Kadikoy, 2011. \scopus{http://www.scopus.com/inward/record.url?eid=2-s2.0-79961201396&partnerID=40&md5=3fc459a344a8da1d485871a753af351f}
%  
%  
%  
%  
%  %2012
%
%  %~\nocite{Carmona201211243}
%  \item C.J. Carmona, S.~Ram\'irez-Gallego, F.~Torres, E.~Bernal, M.J. Del~Jesus y
%  S.~García. Web usage mining to improve the design of an e-commerce website: Orolivesur.com. \emph{Expert Systems with Applications}, 39(12):11243--11249, 2012. \scopus{http://www.scopus.com/inward/record.url?eid=2-s2.0-84861193731&partnerID=40&md5=5ab85a7f28c8008bd2af1bf062088ab6}
%
%  %~\nocite{Carmona2012239}
%  \item C.J.~Carmona, S.~Ramírez Gallego, F.~Torres, E.~Bernal, M.J. Del~Jesús y S.~García. Subgroup discovery applied to the e-commerce website orolivesur.com. \emph{Proceedings of the 14th International Conference on Enterprise Information Systems (ICEIS)}, 2:239--244, Wroclaw, 2012. \scopus{http://www.scopus.com/inward/record.url?eid=2-s2.0-84865755007&partnerID=40&md5=d6827429fc96e281d5c1a465f9f11e59}
%
%  %~\nocite{Hirose2012246}
%  \item F.~Hirose, K.~Uenosono y S.~Komiya. A {CAI} system to identify weak parts of a learner: On the numbers of category layers and questions in a set of questions. \emph{Proceedings of the IASTED International Conference on Engineering and Applied Science (EAS)}, 246--253, Colombo, 2012. \scopus{http://www.scopus.com/inward/record.url?eid=2-s2.0-84883532949&partnerID=40&md5=1e5609f509211842894c897ed81fc856}
%
%  %~\nocite{PinhoLucas20121273}
%  \item J.~Pinho~Lucas, S.~Segrera y M.N.~Moreno. Making use of associative classifiers in order to alleviate typical drawbacks in recommender systems.\emph{Expert Systems with Applications}, 39(1):1273--1283, 2012. \scopus{http://www.scopus.com/inward/record.url?eid=2-s2.0-81855166958&partnerID=40&md5=a8c1bd24d8e1077f5638a860ac565b3c}
%
%  %~\nocite{Rollande2012108}
%  \item R.~Rollande y J.~Grundspenkis. Representation of study program as a part of graph based framework for tutoring module of intelligent tutoring system.\emph{Proceedings of the 2nd International Conference on Digital Information Processing and Communications (ICDIPC)}, 108--113, Klaipeda City, 2012. \scopus{http://www.scopus.com/inward/record.url?eid=2-s2.0-84866696670&partnerID=40&md5=ccaa94aa86887c2593b2c4926c3fe3e6}
%
%  %~\nocite{Zafra20122693}
%  \item A.~Zafra y S.~Ventura. Multi-instance genetic programming for predicting student performance in web based educational environments. \emph{Applied Soft Computing Journal}, 12(8):
%  2693--2706, 2012. \scopus{http://www.scopus.com/inward/record.url?eid=2-s2.0-84861901483&partnerID=40&md5=83ee9766d53a62bbdce1bd94bdb0819a}
%  
%  \item Rafi Ahmad Khan. \emph{KDD for Business Intelligence}. In \emph{Journal of Knowledge Management Practice}, Vol. 13, No. 2, June 2012. \visita{http://www.tlainc.com/articl304.htm}
%  
%  
%  
%  %2013
%
%  %~\nocite{Kim20132281}
%  \item S.~Kim y D.A.~Nembhard. Rule mining for scheduling cross training with a heterogeneous
%  workforce. \emph{International Journal of Production Research}, 51(8):2281--2300, 2013. \scopus{http://www.scopus.com/inward/record.url?eid=2-s2.0-84874621570&partnerID=40&md5=552edc1b424f7bc2c2b3e6c544f206ba}
%
%  %~\nocite{Rollande2013137}
%  \item R.~Rollande y J.~Grundspenkis. Graph based framework and its implemented prototype for personalized study planning. \emph{Proceedings of the 2nd International Conference on E-Learning and E-Technologies in Education (ICEEE)}, 137--142, Lodz, 2013. IEEE Computer Society. \scopus{http://www.scopus.com/inward/record.url?eid=2-s2.0-84898655380&partnerID=40&md5=b3e578daec28f927f82ec488671add0c}
%
%  %~\nocite{Wei2013302}
%  \item C.~Wei. Concept association mining based on \clustering and association rules. \emph{International Journal of Applied Mathematics and Statistics}, 47(17):302--310, 2013. \scopus{http://www.scopus.com/inward/record.url?eid=2-s2.0-84887507914&partnerID=40&md5=c9390f88ea34670cc9fe6ea0a06533fa}
%  
%  
%  
%  %2014
%
%  %~\nocite{Azyurt2014145}
%  \item O.~Ozyurt. The classification of the probability unit ability levels of the eleventh grade turkish students by cluster analysis. \emph{Turkish Online Journal of Distance Education}, 15(2):145--160, 2014. \leePDF{bib/NosCitan/Ozyurt_TheClassificationOfTheProbabilityUnitAbilityLevels_2014.pdf} \descarga{https://tojde.anadolu.edu.tr/tojde56/pdf/article_11.pdf} \scopus{http://www.scopus.com/inward/record.url?eid=2-s2.0-84898749605&partnerID=40&md5=dc839972abdc75761b5b358fbeb2a7b6}
%
%  %~\nocite{PenaAyala201465}
%  \item A.~Peña-Ayala y L.~Cárdenas. How educational data mining empowers state policies to reform education: the mexican case study. \emph{Studies in Computational Intelligence}, 524:65--101, 2014. \scopus{http://www.scopus.com/inward/record.url?eid=2-s2.0-84891845403&partnerID=40&md5=481db82aad7f8fd5b71e26b54614ef19}
%
%  %~\nocite{Ruiz20144}
%  \item P.A.~Potes Ruiz, B.~Kamsu-Foguem y B.~Grabot. Generating knowledge in maintenance from experience feedback. \emph{Knowledge-Based Systems}, 68:4--20, 2014. \leePDF{bib/NosCitan/PotesKamsuGrabot_GeneratingKnowledgeInMaintenanceFromExperienceFeedback_2014.pdf} \descarga{https://www.researchgate.net/profile/Bernard_Kamsu-Foguem/publication/257480419_Improving_Maintenance_Strategies_from_Experience_Feedback/links/542be4dd0cf27e39fa91ba82.pdf?origin=publication_detail}
%\scopus{http://www.scopus.com/inward/record.url?eid=2-s2.0-84906951482&partnerID=40&md5=53d86da9ece5b7afb59096c82d611678}
%
%  %~\nocite{Zhu20142137}
%  \item H.~Zhu y Y.~He. Analysis on tourist characteristics based on rough sets theory and \apriori algorithm. \emph{FALTA REVISTA}, volume 46 VOLUME 3:2137--2143, Wuhan, 2014. \scopus{http://www.scopus.com/inward/record.url?eid=2-s2.0-84887945257&partnerID=40&md5=fe619df980d5476702f7b66c38a27021}
%  
%  \item Shiva Asadianfam y Masoud Mohammadi. \emph{Identify Navigational Patterns of Web Users}. In \emph{International Journal of Computer-Aided Technologies (IJCAx)} Vol.1,No.1,April 2014. \leePDF{bib/NosCitan/AsadianfamMohammadi-IdentifyNavigationalPatternsOfWebUsers-2014.pdf}
%  
%  \item H.~Hashemi, M.~Bayanati, K.~Eisapour, M.A.~Alavi y A.~Alavi. Provide a Model for Web Content Strategy, Using Data Mining Techniques. \emph{Advances in Environmental Biology}, 8(7):3299--3304, may 2014. \leePDF{bib/NosCitan/HashemiBayanatiEisapourAlaviAlavi_ProvideAModelForWebContentStrategyUsingDataMiningTechniques_2014.pdf}
%  
%  \item Samuel  Nowakowski,  Ivana  Ognjanovic,  Monique  Grandbastien,  Jelena  Jovanovic,  Ramo Sendelj. \emph{Two  Recommending  Strategies  to  Enhance  Online  Presence  in  Personal  Learning Environments}. In \emph{Springer  Science+Business  Media}  New  York  2014. \leePDF{bib/NosCitan/NowakowskiOgnjanovicGrandbastienJovanovicSendelj-TwoRecommendingStrategies-2014.pdf}
%  
%  %TODO: Tengo dos más de sep 2014 y enero 2015
%
%\end{enumerate}
\selectlanguage{spanish}






\phantomsection
\subsection*{Recomendaciones en sistemas web mediante el estudio de \irs en \transacciones, 2010}
\addcontentsline{toc}{subsection}{Actas de Interacción'10}
\label{sec:nuestro-Recomendaciones-2010}

En Interacción'10 presentamos las \ROPs a través del siguiente artículo:
\begin{quote}
  Lazcorreta Puigmartí, E., Botella Beviá, F. y Fernández-Caballero, A. Recomendaciones en sistemas web mediante el estudio de ítems raros en transacciones. {\em Actas del XI Congreso Internacional de Interacción Persona-Ordenador}, 385--388, 2010. \leePDF{bib/nuestros/LazcorretaBotellaFCaballero-RecomendacionesEnSistemasWeb-2010.pdf}
\end{quote}

	\begin{quotation}
	\noindent\textbf{Resumen}

	\nopagebreak Las recomendaciones en \portalesWeb se enriquecen cuando incorporan información sobre el uso real del portal. Una de las herramientas que proporcionan dicha información es el \ARM (ARM) en las sesiones de navegación de los usuarios a través del portal. Si el número de páginas del portal es muy grande la \ARM se tropieza con el \dilemaIR, que postula que no se puede encontrar información sobre las páginas poco visitadas sin obtener una explosión de \ars. Este dilema ha sido estudiado desde una perspectiva que sobrecarga las tareas del algoritmo usado y de sus analistas y no aporta reglas de uso frecuente. En este artículo se introduce un nuevo enfoque al dilema que permite a los algoritmos de \ARM encontrar simultáneamente información de interés sobre los páginas del portal poco visitadas, con un consumo mínimo de recursos, sin intervención de los analistas y aportando conocimiento útil.
	\end{quotation}


%\begin{quote} %\citet{Lazcorreta:Botella:FCaballero:ReglasDeOportunidad:09}
  %Lazcorreta Puigmartí, E., Botella Beviá, F. y Fernández-Caballero, A. Reglas de Oportunidad: mejorando las recomendaciones web. {\em Actas del X Congreso Internacional de Interacción Persona-Ordenador}, 2009
%\end{quote}
