% !TEX root = ../../Lazcorreta.Tesis.tex
La línea de investigación se va definiendo poco a poco. En sus inicios teníamos la intención de crear un \srw (\SRW) capaz de adaptarse al usuario en sitios web con gran cantidad de páginas y de usuarios generando gran cantidad de datos sobre su uso. Lo planteamos como un proceso de \wum (\WUM), un proceso específico de \kdd (\KDD) en el que los datos y el conocimiento final giran en torno al uso de la \www (\WWW). Como en todo proceso de \KDD, la \WUM se divide en cinco tareas secuenciales y recurrentes: \emph{Selección}, \emph{Preproceso}, \emph{Transformación}, \dm e "`\emph{Integración y Evaluación}"'. Las tres primeras tareas se han expuesto a fondo en la sección~\ref{sec:srw:muw}, la \dm ha ocupado un papel especial en la sección~\ref{sec:srw:md} pero no se ha dejado resuelta y no ha sido posible llevar a cabo la parte de integración en un sitio web real que genere gran cantidad de datos de uso, por lo que no ha sido posible evaluar el conocimiento adquirido.

La \dm comienza a tomar protagonismo en nuestra investigación. Las \ARs son relaciones muy interesantes y los algoritmos que las obtienen están recibiendo cada vez más atención de la comunidad científica, como muestran las revisiones y comparaciones elaboradas por~\citet{HippGuntzerNakhaeizadeh-AlgorithmsForAssociationRuleMining-2000}, ~\citet{ZhaoBhowmick-ARMSurvey-2003} y~\citet{Goethals-SurveyOnFPM-2003}. Aparecen decenas de algoritmos y estructuras de datos para gestionar los problemas que van surgiendo conforme se va avanzando en esta nueva disciplina donde la mayor dificultad está en administrar correctamente los recursos disponibles para estudiar las cada vez más grandes cantidades de datos a analizar. Nosotros mismos proponemos en la sección~\ref{sec:srw:md:ra} algunos cambios sobre uno de los algoritmos más destacados en la \arm, \apriori. Decidimos profundizar en este elegante algoritmo, complicarlo lo justo para que se pueda adaptar a nuestras necesidades y que pueda incorporar nuestras aportaciones. Decidimos poner a prueba nuestras propuestas para lo que había que desarrollar el código y comenzamos a observar más a fondo el código que otros investigadores habían puesto a nuestra disposición a través de su publicación en las actas de FIMI'03~\citep{ZakiGoethals-ProceedingsFIMI-2003}.

Al analizar a fondo el algoritmo \apriori descubrimos que la generación de \ars se plantea de un modo en que se repiten gran cantidad de cálculos. Propusimos un algoritmo alternativo para el sugerido originalmente por~\citet{AgrawalSrikant-FastAlgorithmsForMiningAssociationRules-LARGO-1994} en el que se ahorran muchos costosos procesos de búsqueda lo que redunda en ganancia de tiempo sin pérdida de ningún tipo de información.

Conforme más profundizamos en el estudio de \ARs más vemos la necesidad de reducir las dimensiones de los datos a tratar. Podemos abordar el estudio completo de "`pequeñas colecciones"' de datos utilizando \apriori para obtener resultados en tiempo real. Pero cuando nos enfrentamos a grandes colecciones de datos siempre hemos de renunciar al análisis de muchos de ellos. En este capítulo intentaremos utilizar \apriori para dividir los datos iniciales en grupos más homogéneos en función de las reglas que cumpla cada registro. Si el \clustering obtenido es correcto se reducirán enormemente las necesidades de recursos pues en cada grupo habrá un número menor de ítems a procesar, podríamos obtener más y mejor información de cada uno de los grupos a partir de los mismos datos.

Otro de los puntos de interés de esta investigación está motivado por la existencia del \dilemaIR. En el capítulo anterior lo resolvimos parcialmente estudiando únicamente las transacciones (sesiones) que procedían de un mismo usuario con lo que se reducía notablemente el número de datos a procesar y se podía llevar a cabo un análisis individual en tiempo real para alimentar un \SRW. Sin embargo este planteamiento no puede aplicarse a grandes colecciones de datos por los problemas de desbordamiento de memoria ya citados. Este capítulo finaliza con una aportación con la que pretendemos aliviar este problema utilizando toda la información disponible en \D, posibilitando el análisis de los \irs mejor relacionados con los ítems frecuentes y su uso en un \SR mediante la definición de \ROPs.
