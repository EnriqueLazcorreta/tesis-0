% !TEX root = ../../Lazcorreta.Tesis.tex
% \ABIERTO%
Una vez resuelta la fase de \fim se utiliza el conjunto \aprioriL de \itemsets frecuentes para obtener las \ARs que contiene el almacén de transacciones \D. Ya tenemos todos los \itemsets cuyo \soporte sea superior al mínimo prefijado, $minSup$, pero aún no tenemos las \ARs, para ello hemos de recorrer todo \aprioriL, de cada uno de los \itemsets frecuentes que contiene obtener todas sus particiones en dos subconjuntos no vacíos y comparar la \confianza de las dos reglas que genera con el umbral mínimo establecido para el estudio, $minConf$.

\apriori es un algoritmo hermoso, sencillo y moldeable, de ahí la atención que ha recibido por parte de la comunidad científica. Está descrito de un modo muy general por lo que se puede modificar con facilidad cualquiera de sus partes sin perder su esencia. La mayoría de los trabajos expuestos en la sección~\ref{sec:2-2-FIM} y de los que mostraremos en la sección~\ref{sec:2-4-ElItemRaro} se centran en mejorar la primera fase de este algoritmo, la \fim, debido a su gran necesidad de recursos y de procesos de lectura/escritura en disco. Si leemos con atención el párrafo anterior podremos ver que la fase de generación de \ars también tiene mucha carga de procesos:

\begin{enumerate}
  \item Hemos de leer cada \itemset frecuente de \aprioriL, lo que se conseguirá mediante un simple bucle que recorra todos sus nodos.
  \item Cada \itemset se divide en un par de subconjuntos no vacíos, $X_1$ y $X_2$, que serán el antecedente y consecuente de dos \ARs diferentes, $X_1 \rightarrow X_2$ y $X_2 \rightarrow X_1$.
  \item Para obtener la \confianza de cada regla hemos de buscar en \aprioriL el \soporte de $X_1$ y de $X_2$, calcular su cociente y determinar si tienen o no confianza mínima.
\end{enumerate}

Los dos primeros pasos son elementales, sin embargo el paso 3 es más complejo de lo que parece a simple vista. Para encontrar el \soporte de cada \kitemset hemos de hacer $k$ búsquedas en \aprioriL. Primero hemos de localizar en \aprioriL[1] el nodo que representa a su primer ítem, una vez encontrado entramos en la rama que se deriva de él y localizar en \aprioriL[2] el nodo que representa a su segundo ítem, siguiendo el proceso hasta localizar su $k$-ésimo ítem en \aprioriL[k]. Aunque usemos algoritmos de búsqueda eficiente en cada rama tenemos que hacer cientos de comparaciones para encontrar en \aprioriL cada uno de los \itemsets obtenidos en el paso 2, lo que nos hizo pensar que es una fase en la que podría aplicar alguna optimización.









%TODO: Esto debería estar en la siguiente sección. Hay muchas propuestas para reducir el número de reglas obtenidas pero ninguna se centra en la propia obtención de las reglas...
% \citet{BrinMotwaniSilverstein_BeyondMarketBaskets_1997} proponen el filtrado de reglas de asociación generalizadas mediante el análisis estadístico de correlación. Su idea es apoyar el estudio en el test de correlación $\chi^2$, lo que puede reducir el número de relaciones encontradas y agilizar por tanto el análisis. Definen las \emph{reglas de correlación} y proponen un algoritmo para extraer las reglas de correlación presentes en grandes \dbs.% (ver algoritmo \ref{alg:chiDosSupport}).
%
% \begin{Definition}[\emph{Reglas de Correlación}]\label{def:2-1-ReglasDeCorrelacion}
% Sea \I un conjunto de ítems y $B$ un conjunto de subconjuntos de \I. Decimos que $\left\{i_{a_1},i_{a_2}\ldots i_{a_m}\right\}$ es una \emph{Regla de correlación} si las ocurrencias de los ítems $i_{a_1},i_{a_2}\ldots i_{a_m}$ están correladas.
% \end{Definition}

% \afterpage{\clearpage
           % \lstinputlisting[label=alg:chi2support,
           %                  caption={Algoritmo $\chi^2$-support, 1997},
           %                  float=htb,
           %                  basicstyle=\scriptsize]
           %                  {./2-ARM/codigo/alg-chi2support}
% }

% \input{algoritmo/alg_chi2support}
%
% \begin{ejem}
% \label{ejem:Brin}
% Los datos mostrados en la tabla \ref{tab:ejem-Brin} reflejan las compras de café ($c$) y té ($t$) realizadas en un comercio. $\bar{c}$ indica las transacciones que no contenían café y $\bar{t}$ las que no contenían té.
% \begin{table}[ht]
%   \centering
%     \begin{tabular}{|c|cc|c|}\hline
%        & $c$ & $\bar{c}$ & $\sum_{\text{filas}}$ \\\hline
%       $t$ & 20 & 5 & 25 \\
%       $\bar{t}$ & 70 & 5 & 75 \\\hline
%       $\sum_{\text{columnas}}$ & 90 & 10 & 100 \\\hline
%     \end{tabular}
%   \caption{Tabla de contingencia entre $c$ y $t$.}
%   \label{tab:ejem-Brin}
% \end{table}
% Con estos datos la regla $t\rightarrow c$ tiene un \soporte del 20\% y una confianza del 80\%. Si consideramos que $c$ tiene un \soporte del 90\% la confianza de la regla muestra una correlación negativa entre ambos ítems ya que si sabemos que el usuario ha adquirido té se reduce la probabilidad de que adquiera café en un 10\%. Esto puede cuantificarse a partir del cálculo $P(c\cap t) / \left(P(c)\cdot P(t)\right)$, que valdría 1 si los hechos de comprar café o té fueran independientes. En este ejemplo su valor es de 0.89, inferior a 1 e indicador de la existencia de una correlación negativa entre ambos eventos.

% También mencionan que la correlación $\ro$ entre comprar te y no comprar café es de 2, alta correlación positiva... Y terminan con una sugerencia al vendedor:  As a result, the store manager may decide to target non-coffee drinkers in his tea displays.
% \end{ejem}
%
% En el mismo artículo postula que la correlación negativa no es necesariamente indeseable. Por ejemplo, en la prevención de incendios sería más interesante encontrar correlaciones negativas entre materiales usados en una construcción y la ocurrencia de incendios. En casos como este se comprende la importancia que tiene la contextualización del estudio en curso.



\subsection{\texttt{genrules()}}
\label{sec:2-2-1-genrules}
%\input{./2-ARM/2-3-GeneracionDeAR/2-3-1-genrules}



\subsection{Apriori2}
\label{sec:2-2-2-Apriori2}
%\input{./2-ARM/2-3-GeneracionDeAR/2-3-2-Apriori2}




% \subsection{Experimentación}
% \label{sec:2-2-X-Experimentacion}
% \input{./2-ARM/2-3-GeneracionDeAR/2-3-X-Experimentacion}
