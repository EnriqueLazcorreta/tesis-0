% !TEX root = ../../../Lazcorreta.Tesis.tex
% \ABIERTO%
%TODO: Faltan los algoritmos que tengo explicados en "Google Drive/enriqueUMH/Dropbox/TesisEnrique/texto/FIM.tex"
Uno de los cuellos de botella de la \arm es la búsqueda de los \itemsets frecuentes presentes en \D. Es una fase en que los consumos de recursos se han de optimizar, sobre todo las necesidades de memoria RAM y el uso de procedimientos con alta carga de proceso. Es tal su importancia que ha creado su propia disciplina, la \fim (\FIM). 

En las siguientes secciones expondremos una selección de los muchos trabajos planteados sobre esta disciplina, organizados cronológicamente y destacando las principales aportaciones de cada uno de ellos. En la sección~\ref{sec:arm:fim:algoritmos-y-estructuras} se mostrarán algunos algoritmos y estructuras de datos especializadas en esta tarea, su lectura puede aportar muchas ideas para seguir investigando. En la sección~\ref{sec:arm:fim:evaluacion-de-diferentes-implementaciones} nos centraremos en la implementación real de los trabajos propuestos y mostramos los resultados que obtuvimos al evaluar diferentes implementaciones del algoritmo \apriori.
