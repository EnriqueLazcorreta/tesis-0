% !TEX root = ../../../Lazcorreta.Tesis.tex
% \ABIERTO%
Al definir las transacciones como vectores binarios se puede interpretar que es mejor guardar los datos a nivel de bit ya que sólo son necesarios dos valores para representar cada dato, lo que convertiría a \D en una matriz de ceros y unos cuyas $|\D|$ filas representan cada una de las \transacciones y sus $N$ columnas cada uno de los ítems de \I. Es fácil pensar en trabajar con matrices de estas características para aprovechar la potencia de los ordenadores al trabajar a nivel de bit, sin embargo no se plantea nadie este formato hasta el trabajo de~\citet{DongHan-BitTableFIEfficientFIMalg-2007}, que proponen comprimir \D en una {BitTable} y usar el algoritmo~\algoritmo{BitTableFI}. Indican que esta estructura posibilita una rápida generación de candidatos y de su recuento por utilizar funciones de unión e intersección de bits, obteniendo mejor rendimiento que los algoritmos con que se compara. En la sección~\ref{sec:4-bit} volveremos a este planteamiento como propuesta de trabajo futuro.

Trabajar con uniones e intersecciones de bits puede ser muy eficiente para un procesador si se consigue programar de forma eficiente, sin embargo al desarrollar un programa con un lenguaje de alto nivel no es fácil trabajar a nivel de bit (en \langCpp se guarda un valor \texttt{bool} usando un Byte, ocho bits) por lo que la mayoría de implementaciones hechas sobre algoritmos de \arm no utilizarán esta definición de \I. Generalmente se representa el ítem $I_k$ utilizando el número $k$ con lo que las \transacciones no se guardan como $N$-tuplas de ceros y unos si no como conjuntos de números enteros. El formato más utilizado para guardar \D de este modo es escribiendo en un fichero de texto plano cada \transaccion en una línea del fichero, lo que se conoce como \emph{representación horizontal de \D}. Algunos algoritmos están especialmente diseñados para aprovechar la \emph{representación vertical de \D}, en que en cada línea se escribe el par $\TID\ item$ de modo que cada \transaccion ocupe tantas líneas consecutivas como ítems contenga. A lo largo de este informe se tratará \D como un conjunto de números enteros y no una matriz de bits.
