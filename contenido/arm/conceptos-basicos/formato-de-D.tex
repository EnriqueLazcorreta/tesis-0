% !TEX root = ../../../Lazcorreta.Tesis.tex
% \ABIERTO%
En la sección~\ref{sec:arm:tipo-de-datos} hemos cuestionado el modo de guardar los datos de \D para el análisis, adelantando que la codificación de los ítems de \I mediante números enteros es el formato más utilizado por los desarrolladores de algoritmos de \ARM. \citet{AgrawalImielinskiSwami-MiningAssociationRulesBetweenSetsOfItemsInLargeDB-1993} no especifican nada sobre este asunto en su artículo, mientras que~\citet{HoutsmaSwami-SETMofAR-1993} proponen el uso de funciones ya implementadas en los \dbmss (\DBMS). Como la \arm es una fase del proceso de \KDD lo más inmediato es usar \dbs para guardar \D, sin embargo la eficiencia de las funciones de un \DBMS es notablemente inferior a la obtenida usando lenguajes de programación compilados si son muchos los datos a procesar.

\citet{AgrawalImielinskiSwami-DatabaseMiningAPerformancePerspective-1993} proponen dotar a los \DBMS de nuevas operaciones bien implementadas para poder hacer \dm sobre \clasificacion, {asociación} y \secuencias directamente sobre la \DB utilizando combinaciones de dichas operaciones para obtener \patrones válidos para los tres modelos abordados.

Hay más investigadores que buscan el uso de \DBMS para generar \ARs, como~\citet{HoutsmaSwami-SetOrientedMiningForAR-1995} que proponen el estudio de \ARM mediante lenguajes nativos de \dbs, SQL concretamente.

Otro trabajo que busca el uso de \DBMS para extraer reglas de asociación es el de \citet{HolsheimerKerstenMannilaToivonen-APerspectiveOnDatabasesAndDataMining-1995}. En su favor está la flexibilidad que permiten los \DBMS para tratar los datos desde múltiples perspectivas, en su contra el tiempo necesario para obtener resultados frente a los algoritmos específicos de \ARM desarrollados mediante lenguajes de programación compilado. Una de las propuestas de este artículo es el uso de la jerarquía de los ítems para reducir el tamaño del repositorio a analizar, concretamente hablan de la cesta de la compra y de descubrir reglas que involucren genéricamente al ítem "`cerveza"' en lugar de usar como ítem cada una de las marcas de cerveza que están en \I, idea que será tomada por otros investigadores para reducir las dimensiones del problema y poder llevar a cabo estudios más amplios.

Aunque los algoritmos presentados de forma teórica a lo largo de esta investigación no especifican el formato físico de los datos a procesar, las comparaciones realizadas sobre los distintos algoritmos no sería correcta si no se experimenta en condiciones similares por lo que en la mayoría de los casos encontraremos implementaciones realizadas con lenguajes de programación compilados. De hecho, los almacenes \D disponibles en \url{http://fimi.ua.ac.be/data/} y \url{https://archive.ics.uci.edu/ml/datasets.html} tienen formato de texto plano, la mayoría de ellos en formato horizontal.