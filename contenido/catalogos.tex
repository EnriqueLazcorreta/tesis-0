% !TEX root = ../Lazcorreta.Tesis.tex
Los trabajos expuestos en los capítulos anteriores muestran una dificultad presente en muchas investigaciones en el ámbito de la informática, la imposibilidad de comprobar si los resultados obtenidos son correctos y aplicables con la tecnología actual. Todas las Ciencias recogen una Teoría que la sustenta, la Informática también, pero las demostraciones teóricas basadas en otras Ciencias no siempre son válidas para la Informática. Hay muchos artículos teóricos sobre Minería de Datos, pero algunos de ellos no evolucionan en un artículo posterior que muestre cómo se ha realizado el experimento con la tecnología actual usando datos reales. Es fácil calcular teóricamente el número de reglas de asociación presente en una colección concreta de datos, pero si el algoritmo propuesto no es capaz de almacenar en RAM todas las reglas de asociación del problema no servirá de nada ese algoritmo en esa situación. \texttt{mushroom} fue el primer caso que encontré curioso en el ambicioso campo de la Minería de Datos, una colección de tan solo 5\,644 registros de 23 valores que sólo contenía 100 valores diferentes no podía ser analizada a fondo por el mejor algoritmo de ARM que conozco, Apriori, cuando el número de transacciones distintas que se puede obtener bajo estas circunstancias es X\,XXX\,XXX y con ellas tenemos un máximo de X\,XXX\,XXX\,XXX\,XXX reglas de asociación. La tecnología actual de un equipo de escritorio no puede gestionar en RAM tanta información, pero me negaba a creer que no se pudiera extraer \emph{toda la información} que contuviera un problema tan pequeño. Al profundizar en \texttt{mushroom} y los artículos que lo mencionaban y otras colecciones de datos que encontré publicadas en los mismos portales encontré un modo de indicar al algoritmo de ARM que estoy tratando con un tipo de colecciones de datos especial. No es simplemente una colección de transacciones como las analizadas en~\ref{sec:arm:conceptos-basicos} si no que éstas tienen unas restricciones muy fuertes en su definición.

Este informe refleja el trabajo que he realizado para descubrir lo suficiente como para ser merecedor del título de doctor en Informática. Los anteriores capítulos reflejan una gran labor de documentación y exposición científica de los resultados que podía obtener con colecciones de datos fijas, no disponía de un servidor capaz de poner a prueba nuestras aportaciones y realizar sugerencias en tiempo real a un gran número de usuarios. Podía comprobar que mis cálculos se podían realizar con la tecnología actual y las colecciones de datos que yo manejaba pero no sabía qué ocurriría si aumentaran mis colecciones de datos o si pudiera realizar sugerencias en tiempo real. Es un buen trabajo teórico apoyado en algunas realizaciones prácticas pero del que no puedo deducir la conclusión que es un buen trabajo de investigación en Informática.






\section{Catálogo}
\label{sec:catalogos:catalogo}
Los catálogos son colecciones de registros preparadas para resolver informáticamente un problema de clasificación. Y muchos investigadores de esta especialidad publican sus datos para que otros investigadores puedan hacer pruebas con las mismas condiciones de partida: una colección de datos con ciertas características. En UCI, KEEL, LUCS\ldots encontraremos muchos catálogos entre los datasets que publican para resolver problemas de clasificación.\marginpar{\footnotesize Acabo de descubrir LUCS, que discretiza las colecciones de UCI y me ofrece 97 valores distintos en adult, frente a los 27\,245 que tiene el de UCI, he de analizarlo con mi código y EXPLICAR MEJOR LAS CONSECUENCIAS DE APLICAR ANTES O DESPUÉS MI MÉTODO O LA AGRUPACIÓN DE VALORES EN ATRIBUTOS NUMÉRICOS ya que se obtendrán reglas y catálogos completos bastante diferentes, esto da para otro artículo y más si tengo en cuenta que tiene datos missing por lo que puedo obtener catálogos completos usando menos atributos con más registros o catálogos completos usando sólo los atributos registrados en cada registro (a no ser que el análisis nos diga que cierto atributo no aporta información\ldots.}

Cuando no sabíamos que esos ficheros contenían catálogos intentábamos aplicar bien conocidos algoritmos de ARM pero no podíamos extraer información que contienen los datos porque se desbordaba la RAM del equipo en que se está aplicando el algoritmo y se abortaba el proceso tras horas de cálculos que finalmente no obteníamos. Esto nos sorprendía porque el primer catálogo que intentamos analizar con Apriori sólo tiene 5\,644 registros de 23 datos, no son números excesivos para un problema de Minería de Datos analizado con un ordenador de escritorio con cierta potencia y capacidad de RAM. Eso nos llevó a descubrir cómo se creó el catálogo a través de \url{UCI/mushroom}\ldots









\section{Catálogo comprimido}
\label{sec:catalogos:catalogo-comprimido}
% !TEX root = ../../Lazcorreta.Tesis.tex
Aprovechando las restricciones implícitas de los catálogos como \texttt{mushroom}\ldots







\subsection{Lectura de catálogos comprimidos}
\label{sec:catalogos:catalogo-comprimido:lectura}









\section{Catálogo completo}
\label{sec:catalogos:catalogo-completo}
% !TEX root = ../../Lazcorreta.Tesis.tex
\ABIERTO
Al seguir trabajando con \mushroom encontramos otra versión del mismo dataset en KEEL

http://sci2s.ugr.es/keel/dataset.php?cod=178


\borrar{Necesito definir antes conjuntoDeValoresDeAtributos, quizá en otra sección.}
\begin{Definition}[\CC] Un \CC es un \catalogo sin incertidumbre.
   $$\CC = \left\{registro_i, i = 1\ldots N\ | \ registro_i \neq registro_j \forall i \neq j\right\},  conjuntoValoresAtributos son todos diferentes$$
\label{def:catalogo}
\end{Definition}






\subsection{Colecciones de \CCs}
\label{sec:clasificacion:catalogo-comprimido:colecciones}
%\input{contenido/clasificacion/catalogo-comprimido/colecciones}

Todos los \CCs pueden tener un \CC maximal y diversos \CCs de menores dimensiones. Volviendo al origen de esta investigación, disponemos de un dataset\ldots







\section{Publicaciones}
\label{sec:catalogos:publicaciones}
% !TEX root = ../../Lazcorreta.Tesis.tex
\subsection{Interacción'12}
En Interacción'12 expusimos\ldots



\subsection{[...]}
En [...]




\subsection{[...]}
Estamos terminando [...]






