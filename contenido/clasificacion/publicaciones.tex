% !TEX root = ../../Lazcorreta.Tesis.tex
En Interacción'12 comenzamos a publicar nuestros resultados sobre las colecciones de \transacciones estructuradas mediante \catalogos. Los mayores avances los hemos obtenido este último año y estamos en un proceso actual de publicación de resultados en diversas revistas y un nuevo congreso internacional.





%\cite{Lazcorreta:Botella:FCaballero:EfficientAnalysisOfTransactionsWebRecommendations:2012}
\subsection*{Efficient analysis of transactions to improve web recommendations, 2012}

\begin{quote}
  Lazcorreta Puigmartí, E., Botella Beviá, F. y Fernández-Caballero, A. Efficient analysis of transactions to improve web recommendations. {\em Actas del XIII Congreso Internacional de Interacción Persona-Ordenador}, ACM International Conference Proceeding Series, 2012. \leePDF{bib/nuestros/LazcorretaBotellaFCaballero_EfficientAnalysisOfTransactionsWebRecommendations_2012.pdf}
\leePDF{bib/nuestros/LazcorretaBotellaFCaballero_AnalisisEficienteDeTransacciones_12_Presentacion.pdf}
\scopus{http://www.scopus.com/inward/record.url?eid=2-s2.0-84869053054&partnerID=40&md5=ded243c196907c1e6eec90cd9624ab3d}
\descargaRevisado{https://www.researchgate.net/publication/266652926_Efficient_analysis_of_transactions_to_improve_Web_Recommendations}
\descargaRevisado{https://www.deepdyve.com/lp/association-for-computing-machinery/efficient-analysis-of-transactions-to-improve-web-recommendations-mBuBBNIv7c}
\github{https://github.com/EnriqueLazcorreta/tesis-0/blob/master/bib/nuestros/LazcorretaBotellaFCaballero_EfficientAnalysisOfTransactionsWebRecommendations_2012.pdf}
\github{https://github.com/EnriqueLazcorreta/tesis-0/blob/master/bib/nuestros/LazcorretaBotellaFCaballero_AnalisisEficienteDeTransacciones_12_Presentacion.pdf}
\end{quote}

\begin{quotation}
	\noindent\textbf{Resumen}

	\nopagebreak When we deal with big repositories to extract relevant information in a short period of time, pattern extraction using data mining can be employed. One of the most used patterns employed are Association Rules, which can measure item co-occurrence inside large set of transactions. We have discovered a certain type of transactions that can be employed more efficiently that have been used until today. In this work we have applied a new methodology to this type of transactions, and thus we have obtained execution times much faster and more information than that obtained with classical algorithms of Association Rule Mining. In this way we are trying to improve the response time of a recommendation web system in order to offer better responses to our users in less time. Copyright 2012 ACM.
\end{quotation}








\subsection{[...]}
En la actualidad estamos completando la redacción de un artículo sobre esta fase de nuestra investigación, que será enviada en breve para su revisión y posible publicación en revista. En él se exponen la teoría y experimentos que forman la sección~\ref{sec:clasificacion:catalogo-completo}.




\subsection{[...]}
La Interacción Persona-Ordenador es una disciplina en la que se analizan muchos catálogos por lo que presentaremos nuestra aportación a esta rama de la ciencia en el Congreso Internacional Interacción'16, que se celebrará en septiembre de 2016 en Salamanca.