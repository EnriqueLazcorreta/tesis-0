% !TEX root = ../../../Lazcorreta.Tesis.tex
%\ABIERTO
La definición de \transaccion (véase pg.~\pageref{def:1:3:2:transaccion}) es muy elemental: "`conjunto de ítems observados en ciertas circunstancias"'. Esta simple definición permite modelar muchas colecciones de datos mediante \transacciones, desde la clásica "`cesta de la compra"' hasta el estudio de una \sn web -- como si se tratara de un conjunto de páginas web visitadas conjuntamente -- pasando por la observación de las características de individuos que tienen enfermedades similares o la detección de fraude en el uso de tarjetas de crédito. En este capítulo estamos trabajando con la observación de ciertas características de los individuos de una población, ejemplarizando en el fichero \mushroom.

Si suponemos que no es casual el hecho de formar una \transaccion agrupando unos ítems y no otros, y observamos un conjunto grande de \transacciones, no es difícil pensar que encontraremos en ellas conjuntos de ítems que se repiten, a los que llamamos \itemsets frecuentes o patrones. Por ejemplo, en la cesta de la compra estos patrones nos pueden sugerir que "`el cliente que compra el producto $A$ suele comprar conjuntamente el producto $B$"'. En el análisis de navegación en un portal web podemos encontrar la regla que sugiere que "`el usuario que visita la página $A$ suele acudir en la misma visita a la página $B$"' o incluso que "`la página $A$ debe tratar sobre el mismo tema que la página $B$"', pues observamos que los usuarios las visitan conjuntamente con mucha frecuencia. En el problema de \clasificacion podríamos descubrir que "`la mayoría de setas que tienen \emph{estas características} son venenosas"'.

Es importante destacar que \ARM suele trabajar sin conocimiento previo de la población en estudio, dejando que sean los datos quienes muestren esa información. Todos los algoritmos de \ARM buscan extraer el máximo conocimiento de una gran cantidad de \transacciones en el menor tiempo posible. Al aplicarlos se pueden hacer ajustes para cada estudio concreto, generalmente en base a estudios previos hechos sobre la misma poblacón. Pero al plantearlos se debe procurar que sean aplicables a todas las poblaciones en estudio.

A lo largo de su evolución, \ARM se ha ido desarrollando en base a conceptos teóricos y a su aplicación en ciertos \datasets, Muchos desarrollos han resultado poco competitivos al ser aplicados en repositorios concretos, lo que genera una constante duda sobre si existe un algoritmo mejor que otro o si todo depende de los datos sobre los que son aplicados. Es el caso de \texttt{chess.dat} y \mushroom: pese a tener dimensiones reducidas no se suelen utilizar con \soporte mínimo bajo, pues su análisis mediante \ARM produce una explosión de reglas que o bien no pueden ser guardadas en memoria o bien tardan tanto tiempo en obtenerse que se pierde su utilidad en servicios que deben ser ágiles, como los \sr.

En ninguno de estos trabajos se ha tenido en cuenta que realmente estamos analizando dos tipos bien diferenciados de \transacciones:
\begin{itemize}
  \item la clásica "`cesta de la compra"' de la que surge el \ARM, compuesta por un número indeterminado de ítems de \I y sin ninguna restricción, y
  \item las \transacciones derivadas de la transformación de observaciones en el problema de \Clasificacion, compuestas por un número concreto de ítems de \I y con una estructura fija.
\end{itemize}

Para entenderlo mejor nos referiremos a la colección de datos \mushroom. En su diseño se consideraron 22 atributos diferentes (color, textura\ldots) de un total de 8\,124 setas diferentes, clasificadas cada una de ellas como "`venenosa"' o "`comestible"'. Para dar a estos datos forma de \transaccion y poder estudiarlos mediante \ARM se asigna un código único a cada uno de los pares \atributo-valor. Así, los códigos 1 y 2 corresponden a la \clasificacion realizada, los códigos 3, 4, 5, 6, 7 y 8 indican el color. Cada seta es observada y se anota mediante una \transaccion su color, etc., de modo que todas las \transacciones están compuestas por 23 ítems, su clase más el valor de cada uno de los 22 atributos medidos.

Una característica de este tipo de \transaccion que la diferencia de la cesta de la compra es la dependencia estructural entre los ítems de la población en estudio. Si una \transaccion tradicional contiene un ítem determinado no podemos asegurar que no contenga cualquier otro ítem de \I. Sin embargo si una \transaccion formada por códigos \atributo-valor contiene el código de un \atributo-valor concreto sabemos con seguridad que no contendrá otro código perteneciente al mismo atributo. En el caso de \mushroom, si una \transaccion contiene el código 1 (\emph{venenosa}) no contendrá el código 2 (\emph{comestible}).

Otra característica de este tipo de \transacciones es la gran densidad de su matriz de correlación. En la práctica nos encontramos con que la dificultad de trabajar con \transacciones de este tipo se debe a la gran cantidad de relaciones existentes en muchos de los \datasets utilizados. Cada par \atributo-valor está relacionado con casi todos los valores del resto de \atributos, hecho que no ocurre en la clásica cesta de la compra. Es esto lo que provoca la falta de recursos y el exceso de tiempo empleado cuando intentamos analizar a fondo estas colecciones de datos con los algoritmos clásicos de \ARM, si una transacción de este tipo no tuviera una matriz de correlación densa y tuviera un número reducido de ítems su análisis no presentaría ningún problema.

Este tipo de \transacciones sirve para modelizar muchos experimentos reales, como hacen \cite{CarmonaRGallegoTorresBernalDelJesusGarcia-WUMtoImprovePortalDesign-2012}, con las \sns de los usuarios de un portal de comercio electrónico. Los datos que recogen están previamente diseñados en forma de \atributos de modo que se observa el navegador usado por el usuario, el tipo de visitante, las palabras clave usadas\ldots Todos los \atributos se han dividido en un pequeño conjunto de valores representativos y para formar las \transacciones han anotado el valor concreto de cada atributo en una visita al portal.

A partir de ahora tilizaremos el término \registro para referirnos a las \transacciones \emph{estructuradas}.

Los \registros pueden presentar problemas en su análisis cuando existe relación entre la mayoría de sus ítems. Sin embargo, antes de proceder a su análisis podemos asegurar que los distintos ítems que forman un un \atributo son, por definición, excluyentes. Esta característica no ha sido utilizada en ninguno de los estudios previos sobre \ARM que hemos revisado, por lo que proponemos su uso para que no se pierda información contenida en los datos, información que no sabemos aún si es de calidad. Esto nos permitirá abordar problemas que con las propuestas existentes no se pueden resolver por falta de recursos de memoria o bien por tardar mucho más tiempo del aceptado por servicios que han de ser ágiles como los \srs.

%TODO: ¿Añadir transacción estructurada a indice-alfabetico?
La primera gran diferencia entre una \transaccion sin restricciones y un \registro aparece en el conjunto \I, que ha sido diseñado para un experimento concreto. El \dataset en que se recojan los datos obtenidos sobre los individuos de la población en estudio presentará características que no sólo se deben a los individuos estudiados, la elección de los \atributos a medir y de los valores que puede tomar cada \atributo son fundamentales y cualquier cambio en ellos producirá un \dataset diferente en el que podrán existir más o menos relaciones de co-ocurrencia entre los ítems que contiene. Otra diferencia entre los \datasets creados con \transacciones y los creados con \registros se debe a que en los primeros un individuo puede generar más de una \transaccion mientras que en los segundos cada individuo tiene asociada un único \registro, lo que produce colecciones de datos menores de las que deberíamos poder extraer más información. Por último los \registros tienen todos el mismo número de ítems y una restricción fundamental: en cada una de ellos aparece uno y sólo uno de los posibles valores de cada uno de los \atributos en estudio. Utilizaremos esta restricción para reducir las dimensiones del problema a resolver y poder utilizar los algoritmos existentes sobre \ARM.

