% !TEX root = ../../../Lazcorreta.Tesis.tex
%\ABIERTO
Desde sus inicios la \ARM se ha protegido del \dilemaIR mediante la definición de un \soporte mínimo que impide estudiar los ítems de menor soporte (\irs) para evitar problemas de desbordamiento de memoria y poder ofrecer resultados en poco tiempo. En la evolución del \ARM hay múltiples aportaciones que alivian este problema pero siempre a costa de renunciar al estudio de algunas relaciones existentes entre los datos en estudio.

En los primeros trabajos sobre \ARM se trataba de evitar la búsqueda de relaciones entre los \irs \citep{AgrawalImielinskiSwami-MiningAssociationRulesBetweenSetsOfItemsInLargeDB-1993,AgrawalSrikant-FastAlgorithmsForMiningAssociationRules-1994,ParkChenYu-UsingAHashBasedMethod-1997}. En \cite{LiuHsuMa-ARMWithMultipleMS-1999}, proponen el uso de múltiples soportes mínimos para que un \itemset sea considerado frecuente sólo si su \soporte es grande en relación al \soporte de los ítems que lo forman. De este modo se ahorra espacio en memoria que puede ser utilizado para analizar ítems con menor \soporte. El uso de un \soporte relativo para cada \itemset basado en la \confianza de las \ars que generan sus ítems, lo que permite guardar información de los \itemsets con menor \soporte que el \soporte mínimo que superen su \soporte relativo \citep{YunHaHwangRyu-MiningAROnSignificantRareDataUsingRelativeSupport-2003}, garantizando que las reglas que generan sean de calidad. En \cite{HuChen-MiningARwithMMS-2006}, se propone una mejora del algoritmo propuesto en \cite{LiuHsuMa-ARMWithMultipleMS-1999}, para dotarlo de escalabilidad. En \cite{TsengLin-EfficientMiningOfAR-2007}, trabajan con las mismas ideas de \soporte múltiple pero reducen el número de \itemsets a guardar en función de su \lift, que mide la mejora que produce la presencia de un ítem en el soporte del resto de ítems que contiene. En \cite{KiranReddy-ImprovedMultipleMSBasedAppMineRareAR-2009}, utilizan múltiples \soportes basados en la distribución conjunta de los ítems y no sólo en su distribución individual como proponían en \cite{LiuHsuMa-ARMWithMultipleMS-1999}. Todas estas aportaciones reducen el número de itemsets almacenados en memoria, con lo que pueden almacenarse otros \itemsets menos frecuentes pero con mejor información sobre la población en estudio.

Cuando trabajamos con un número grande de ítems diferentes y una gran cantidad de \transacciones el \dilemaIR sólo es resoluble utilizando mucho tiempo y/o potentes computadores. Lo que sorprende es que este dilema aparezca en colecciones de datos con dimensiones reducidas, como ocurre con muchos repositorios utilizados en artículos sobre el problema de \Clasificacion que incorporan técnicas de \arm.

Es el caso de \texttt{chess.dat} y \mushroom.  \texttt{chess} recoge información sobre la posición final de una serie de piezas en el juego del ajedrez, contiene tan solo 75 ítems distintos en un total de 3\,196 transacciones. \texttt{mushroom} recoge el valor de una serie de atributos medidos en ciertas setas, utilizando 119 ítems distintos en 8\,124 transacciones. Todos los investigadores que han trabajado con estas colecciones han tratado sus transacciones del mismo modo que la clásica \transaccion formada por la ``cesta de la compra''. En esta sección mostramos que se puede dar un tratamiento diferente sin perder las ventajas del \ARM ni aplicar nuevos algoritmos.
