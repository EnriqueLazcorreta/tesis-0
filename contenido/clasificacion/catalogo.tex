% !TEX root = ../../Lazcorreta.Tesis.tex
Los catálogos son colecciones de registros preparadas para resolver informáticamente un problema de clasificación. Y muchos investigadores de esta especialidad publican sus datos para que otros investigadores puedan hacer pruebas con las mismas condiciones de partida: una colección de datos con ciertas características. En UCI, KEEL, LUCS\ldots encontraremos muchos catálogos entre los datasets que publican para resolver problemas de clasificación.\marginpar{\footnotesize Acabo de descubrir LUCS, que discretiza las colecciones de UCI y me ofrece 97 valores distintos en adult, frente a los 27\,245 que tiene el de UCI, he de analizarlo con mi código y EXPLICAR MEJOR LAS CONSECUENCIAS DE APLICAR ANTES O DESPUÉS MI MÉTODO O LA AGRUPACIÓN DE VALORES EN ATRIBUTOS NUMÉRICOS ya que se obtendrán reglas y catálogos completos bastante diferentes, esto da para otro artículo y más si tengo en cuenta que tiene datos missing por lo que puedo obtener catálogos completos usando menos atributos con más registros o catálogos completos usando sólo los atributos registrados en cada registro (a no ser que el análisis nos diga que cierto atributo no aporta información\ldots.}

Cuando no sabíamos que esos ficheros contenían catálogos intentábamos aplicar bien conocidos algoritmos de ARM pero no podíamos extraer información que contienen los datos porque se desbordaba la RAM del equipo en que se está aplicando el algoritmo y se abortaba el proceso tras horas de cálculos que finalmente no obteníamos. Esto nos sorprendía porque el primer catálogo que intentamos analizar con Apriori sólo tiene 5\,644 registros de 23 datos, no son números excesivos para un problema de Minería de Datos analizado con un ordenador de escritorio con cierta potencia y capacidad de RAM. Eso nos llevó a descubrir cómo se creó el catálogo a través de \url{UCI/mushroom}\ldots

Los catálogos caracterizan un problema de clasificación concreto. Si queremos plantear otro problema de clasificación, bien etiquetando a los mismos individuos en otras clases o bien utilizando atributos diferentes no podemos utilizar directamente cualquier catálogo que tengamos sobre la misma población. Si los dos problemas usaran los mismos atributos pero diferentes clases y las clases en estudio son independientes no servirá de nada la información que tengamos sobre los catálogos completos del primer problema de clasificación si no sabemos analizar qué información puede ser relevante y cuál no, de hecho la información menos relevante en esta situación es la distribución de las clases en cada uno de los problemas de clasificación por lo que debemos huir de interpretaciones erróneas utilizando estos datos para estimar soportes o confianzas poblacionales.

De un catálogo se puede extraer información válida para otro problema de clasificación que utilice los mismos atributos ya que si en la muestra en que se basa el catálogo no presenta cierta relación entre los valores de los atributos YA SABEMOS QUE NO APARECERÁ ESA RELACIÓN AUNQUE CAMBIEMOS DE CLASES (siempre que el catálogo sea válido, aún tengo que hacer muchas definiciones sobre muestra, población, distribución de clases, problema de clasificación, atributos, clases, catálogos, catálogos completos, validez de un catálogo\ldots).

Aunque la ARM busca cualquier relación entre cualquier par (o $k$-itemset) de valores de D, el objetivo del problema de clasificación es siempre el mismo, etiquetar cada registro con una clase basándose en la información disponible sobre otros registros con valores idénticos en sus atributos.