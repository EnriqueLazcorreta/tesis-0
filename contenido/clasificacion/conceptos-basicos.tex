% !TEX root = ../../Lazcorreta.Tesis.tex
%TODO: Incluir \CAR en el índice como Regla de Clasificación Asociativa o Regla de Asociación de Clases (estudiarlo mejor en la literatura, esta última sólo me suena del proyecto de tesis de H.León).
El problema de \Clasificacion tiene ya su propia notación, igual que el de \arm (definida en el capítulo~\ref{chap:arm}), en esta sección intentaremos unificarla y discutiremos algunos aspectos de ambas disciplinas que no han sido considerados en la bibliografía usada para este trabajo, lo que nos llevará también a definir nuevos conceptos e intentar caracterizarlos.


\begin{Definition}[Individuo]
   Sean $A_i, i = 1 \ldots n$ un conjunto de atributos, siendo $n_i$ el número de valores distintos que puede tomar el atributo $A_i$. Definimos un \emph{individuo} como una $n$-túpla, un conjunto de $n$ valores obtenidos ordenadamente de los atributos $A_i$.
   $$individuo = \left(a_1, a_2\ldots a_n\right), a_i \in A_i$$
\label{def:individuo}
\end{Definition}


\cite{HLeonCarrascoHPalancarMTrinidad-DesarrolloDeClasificadoresBasadosEnRA-2010}, utilizan las siguientes definiciones para modelizar las Reglas de Asociación de Clase (que otros autores denominan Reglas de Clasificación Asociativa), las \ars adaptadas al problema de \Clasificacion.

\borrar{[HLeon-2010\ldots]}
Sean $\I = \{i_1,i_2,\ldots,l_n\}$ un conjunto de $n$ ítems y \T un conjunto de transacciones. Cada \transaccion en \T está formada por un conjunto de ítems $X$ tal que $X\in\I$.
\begin{Definition}[\Itemset]
  El tamaño de un conjunto de ítems está dado por su cardinalidad; un conjunto de ítems de cardinalidad $k$ se denomina $k$-itemset.
\label{def:itemset}
\end{Definition}

\begin{Definition}[Soporte]
  El \soporte de un conjunto de ítems $X$, en adelante $Sop(X)$, se define como la fracción de \transacciones en \T que contienen a $X$. El \soporte toma valores en el intervalo $[0,1]$.
\label{def:soporte}
\end{Definition}

\begin{Definition}[$minSup$]
  Sea $minSup$ un umbral previamente establecido, un conjunto de ítems $X$ se denomina \emph{frecuente} (FI por sus siglas en inglés) si $Sop(X) \geq minSup$.
\label{def:flujo-de-datos}
\end{Definition}

\begin{Definition}[\AR]
  Una \AR (AR por sus siglas en inglés) sobre el conjunto de transacciones \T es una implicación $X\Rightarrow Y$ tal que $X\subset\I, Y\subset\I$ y $X\cap Y=\emptyset$.
\label{def:reglas-de-asociacion}
\end{Definition}

\begin{Definition}[Especificidad]
  Dadas dos reglas $R_1: X_1\Rightarrow Y$ y $R_2: X_2\Rightarrow Y$, se dice que $R_1$ es más específica que $R_2$ si $X_2\subset X_1$.
\label{def:reglas-de-asociacion}
\end{Definition}

Las medidas más usadas en la literatura para evaluar la calidad de una \AR son el \soporte y la \confianza.

\begin{Definition}[Soporte de una regla]
  El \soporte de una \ar $X\Rightarrow Y$ es igual a $Sop(X\cup Y)$.
\label{def:soporte-de-una-AR}
\end{Definition}
\borrar{[\ldots HLeon-2010]}




\subsection{Tal}
\label{sec:clasificacion:conceptos-basicos:tal}
%\input{contenido/clasificacion/conceptos-basicos/tal}

\begin{Definition}[Individuo]
   Sean $A_i, i = 1 \ldots n$ un conjunto de atributos, siendo $n_i$ el número de valores distintos que puede tomar el atributo $A_i$. Definimos un \emph{individuo} como una $n$-túpla, un conjunto de $n$ valores obtenidos ordenadamente de los atributos $A_i$.
   $$individuo = \left(a_1, a_2\ldots a_n\right), a_i \in A_i$$
\label{def:individuo}
\end{Definition}


\begin{Definition}[Registro]
   Sean $A_i, i = 1 \ldots n$ un conjunto de atributos, siendo $n_i$ el número de valores distintos que puede tomar el atributo $A_i$. Definimos un registro como una $n$-túpla, un conjunto de $n$ valores obtenidos ordenadamente de los atributos $A_i$.
   $$registro = \left(item_1, item_2\ldots item_n\right), item_i \in A_i$$
\label{def:registro}
\end{Definition}

\begin{Definition}[\Catalogo] Un \catalogo es un conjunto de $N$ registros diferentes.
   $$\catalogo = \left\{registro_i, i = 1\ldots N\ | \ registro_i \neq registro_j \forall i \neq j\right\}$$
\label{def:catalogo}
\end{Definition}
