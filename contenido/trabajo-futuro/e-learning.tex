% !TEX root = ../../Lazcorreta.Tesis.tex
\ABIERTO
El e-learning es una de mis pasiones, funde la docencia con la tecnología y permite recabar muchos datos sobre la interacción de alumnos y profesores con el proceso de enseñanza/aprendizaje. En 2006 me introduje en esta disciplina publicando "`Auto-Adaptive Questions in E-Learning System"'\footnote{\scriptsize\url{http://aipo.es/articulos/4/35.pdf}} en las actas del \emph{Sixth IEEE International Conference on Advanced Learning Technologies} (ICALT'06). Sin embargo tuve que abandonar esta disciplina para centrarme en la investigación reflejada en esa tesis.

Existen muchas aportaciones que aún tengo que revisar a fondo, como la de \citet{GSalcinesRomeroVenturaDeCastro-CollaborativeRSUsingDistributedRM-2008} en la que podrían encajar mis aportaciones al uso de \catalogos con el uso de \ars en e-learning. Tomando estrategias como la planteada por \citet{CarmonaRGallegoTorresBernalDelJesusGarcia-WUMtoImprovePortalDesign-2012} en la que se seleccionan una serie de atributos y se modelizan los valores que puede tomar cada atributo construyendo un \SRW que se base en \datasets que puedan ser tratados como \catalogos si conseguimos seleccionar los atributos y valores necesarios para obtener información fidedigna sin un número excesivo de dimensiones.