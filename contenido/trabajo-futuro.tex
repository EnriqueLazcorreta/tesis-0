% !TEX root = ../Lazcorreta.Tesis.tex

Tantos años de trabajo han dado lugar a muchas ideas teóricas sobre la aplicación de técnicas de DM por lo que quedan abiertas muchas líneas de investigación que podrían ser continuación de este trabajo. Como \emph{trabajar a nivel de \texttt{bits}} buscando la máxima eficiencia en el uso informático de grandes colecciones de datos, o \emph{profundizar en el desarrollo de Clasificadores}, de \emph{lógica difusa para agrupación de valores en atributos numéricos o de amplios rangos} y de tantas otras cosas que han ido apareciendo en el estado del arte de esta tesis y que no he podido abarcar para centrarme en obtener algo tangible mediante el método científico.

La investigación mostrada en el último capítulo de esta tesis está avalada por su implementación en el campo de la Minería de Datos utilizando la tecnología actual.
\marginpar{\footnotesize Es evidente que tengo que reescribir este párrafo. La idea es interesante pero\ldots}
El preproceso de cualquier catálogo permite crear colecciones de catálogos que pueden ser utilizadas en tiempo real en grandes problemas de clasificación que pueden ser escalados sin tener que renunciar en cada nuevo estudio a todo el conocimiento adquirido en estudios sobre las mismas clases. En este trabajo se ha demostrado que cualquier subconjunto de un catálogo completo puede ser tratado como catálogo completo considerando siempre la incertidumbre que puede contener, si quisiéramos utilizar los datos de un problema de clasificación en otro problema de clasificación con otras clases podríamos comenzar con los registros-tipo del primer catálogo, todos los que puedan ser clasificados en el segundo problema se incorporan al catálogo del segundo problema pero así no  voy bien, lo que quería decir es que si empezamos con el menor de todos los catálogos ínfimos y vamos catalogando en la segunda clase todos sus registros podemos llegar a no tener incertidumbre (caso ideal y poco probable si la segunda clase es independiente de la primera, dato interesante) pero al menos si tenemos incertidumbre es posible que sea poca, si hacemos lo mismo con otros catálogos ínfimos podríamos descubrir qué atributos aportan más determinación al segundo problema y plantear un catálogo inicial para el segundo problema.
