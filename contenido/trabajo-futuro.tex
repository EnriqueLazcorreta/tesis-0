% !TEX root = ../Lazcorreta.Tesis.tex
\section{Trabajo Futuro}
\label{sec:clasificacion:trabajo-futuro}
Tantos años de trabajo han dado lugar a muchas ideas teóricas sobre la aplicación de técnicas de \DM por lo que quedan abiertas muchas líneas de investigación que podrían ser continuación de este trabajo. Como \emph{trabajar a nivel de \texttt{bits}} buscando la máxima eficiencia en el uso informático de grandes colecciones de datos, o \emph{profundizar en el desarrollo de Clasificadores}, de \emph{lógica difusa para agrupación de valores en atributos numéricos o de amplios rangos} y de tantas otras cosas que han ido apareciendo en el estado del arte de esta tesis y que no he podido abarcar para centrarme en obtener algo tangible mediante el método científico.

La investigación mostrada en el último capítulo de esta tesis está avalada por su implementación en el campo de la Minería de Datos utilizando la tecnología actual.
\marginpar{\footnotesize Es evidente que tengo que reescribir este párrafo. La idea es interesante pero\ldots}
El preproceso de cualquier catálogo permite crear colecciones de catálogos que pueden ser utilizadas en tiempo real en grandes problemas de clasificación que pueden ser escalados sin tener que renunciar en cada nuevo estudio a todo el conocimiento adquirido en estudios sobre las mismas \clases. En este trabajo se ha demostrado que cualquier subconjunto de un catálogo completo puede ser tratado como catálogo completo considerando siempre la incertidumbre que puede contener, si quisiéramos utilizar los datos de un problema de clasificación en otro problema de clasificación con otras \clases podríamos comenzar con los registros-tipo del primer catálogo, todos los que puedan ser clasificados en el segundo problema se incorporan al catálogo del segundo problema pero así no  voy bien, lo que quería decir es que si empezamos con el menor de todos los catálogos ínfimos y vamos catalogando en la segunda \clase todos sus registros podemos llegar a no tener incertidumbre (caso ideal y poco probable si la segunda \clase es independiente de la primera, dato interesante) pero al menos si tenemos incertidumbre es posible que sea poca, si hacemos lo mismo con otros catálogos ínfimos podríamos descubrir qué atributos aportan más determinación al segundo problema y plantear un catálogo inicial para el segundo problema.

También queda para el futuro la agrupación de valores en los atributos numéricos. Hay ya muchas investigaciones en torno a este campo y creo que con los primeros análisis hechos a un dataset se puede obtener información que pueda ayudar al investigador a hacer las agrupaciones de modo que se pueda seguir trabajando con catálogos completos ya que el agrupamiento puede generar incertidumbre. Este aspecto es muy importante pero es mucho lo que hay que investigar para llegar a conclusiones y resultados útiles, como los propuestos por \citet{DeshpandeKarypis-UsingConjunctionofAttributeValuesforClassification-2002}.




\subsection{e-learning}
\label{sec:clasificacion:e-learning}
% !TEX root = ../../Lazcorreta.Tesis.tex
\ABIERTO
El e-learning es una de mis pasiones, funde la docencia con la tecnología y permite recabar muchos datos sobre la interacción de alumnos y profesores con el proceso de enseñanza/aprendizaje. En 2006 me introduje en esta disciplina publicando "`Auto-Adaptive Questions in E-Learning System"'\footnote{\scriptsize\url{http://aipo.es/articulos/4/35.pdf}} en las actas del \emph{Sixth IEEE International Conference on Advanced Learning Technologies} (ICALT'06). Sin embargo tuve que abandonar esta disciplina para centrarme en la investigación reflejada en esa tesis.

Existen muchas aportaciones que aún tengo que revisar a fondo, como la de \citet{GSalcinesRomeroVenturaDeCastro-CollaborativeRSUsingDistributedRM-2008} en la que podrían encajar mis aportaciones al uso de \catalogos con el uso de \ars en e-learning. Tomando estrategias como la planteada por \citet{CarmonaRGallegoTorresBernalDelJesusGarcia-WUMtoImprovePortalDesign-2012} en la que se seleccionan una serie de atributos y se modelizan los valores que puede tomar cada atributo construyendo un \SRW que se base en \datasets que puedan ser tratados como \catalogos si conseguimos seleccionar los atributos y valores necesarios para obtener información fidedigna sin un número excesivo de dimensiones.




\subsection{Flujos de Datos}
\label{sec:clasificacion:flujos-de-datos}
% !TEX root = ../../Lazcorreta.Tesis.tex
%\ABIERTO
%TODO: Con author={Yaniela {Fernández Mena} and Raudel {Hdez. León} and José {Hdez. Palancar}} tengo varios errores de \endcsname que no controlo, con author={Yaniela {Fernández Mena} and Raudel {Hernández León} and José Hernández Palancar} no aparecen esos errores pero en la cita aparece Fernández Mena, Hernández León y Palancar, 2013.
\cite{FMenaHLeonHPalancar-ClasificacionBasadaEnCARFlujosDeDatos-2013}, presentan un estado del arte actualizado sobre el uso de \CARs sobre Flujos de Datos, que está estrechamente relacionado con las propuestas hechas en esta sección ya que centran el foco en el \clasificador obtenido tras las primeras lecturas de los datasets disponibles para el estudio de \clasificacion en curso.



\begin{quote}
Los \clasificadores para flujos de datos, a diferencia de los \clasificadores para datos estáticos, no disponen de todo el conjunto de transacciones etiquetadas (conjunto de entrenamiento) a priori, sino que estas arriban de forma incremental a lo largo del tiempo. Por otro lado, dado que los flujos de datos son, en teoría, infinitos, resulta imposible cargar todas las transacciones en memoria y se requiere de un algoritmo incremental que actualice el clasificador con la información de las nuevas transacciones, reutilizando la información extraída de las transacciones previas. En general, los clasificadores para flujos de datos [1,8,23] deben garantizar que:

\begin{enumerate}
  \item Cada \transaccion del flujo de datos sea procesada a lo sumo una vez.
  \item Los resultados de \clasificacion estén disponibles en todo momento.
  \item El modelo resultante sea consistente con las nuevas \transacciones que llegan, pues los datos pueden variar a lo largo del tiempo debido a cambios en su distribución de probabilidad.
\end{enumerate}

La \clasificacion en flujos de datos tiene diversas aplicaciones como: detección de software malicioso [46], \clasificacion de paquetes en el área de las telecomunicaciones [53], monitoreo de procesos industriales [4,37], monitoreo de datos de navegación en vehículos automotrices [38], detección de fraudes en transacciones bancarias [65], entre otros.

Varios han sido los \clasificadores adaptados al entorno de flujos de datos. Entre los más reportados se encuentran los árboles de decisión [19,25,26,34,36,59,61] y los métodos de ensamble de \clasificadores [9,39,47,51,57]. Otros \clasificadores que han recibido menor atención son los basados en el vecino más cercano [5,54], las máquinas de vectores de soporte [43,66] y los basados en reglas. Particularmente, los \clasificadores basados en \CARs, son preferidos por muchos especialistas debido a su interpretabilidad, aspecto que los hace más expresivos y fáciles de comprender. Su interpretabilidad permite a los especialistas modificar las reglas con base en su experiencia y así mejorar la eficacia del clasificador. Además de los clasificadores basados en CARs, los árboles de decisión también generan reglas comprensibles. Para construir un clasificador utilizando árboles de decisión se sigue una estrategia voraz seleccionando en cada momento la característica que mejor separa las \clases. Sin embargo, esta estrategia voraz puede podar reglas interesantes. En [55], los autores probaron que las reglas obtenidas de los árboles de decisión son un subconjunto de las reglas generadas por los clasificadores basados en CARs, asumiendo un umbral relativamente bajo de concurrencia (Soporte) de los elementos que componen la regla.
\end{quote}

Los \catalogos consiguen reducir el número de objetos a utilizar en memoria RAM, de modo que el \clasificador puede encontrar una \emph{regla determinista basada en los datos conocidos}.\borrar{Añadir \emph{regla determinista basada en los datos conocidos} a definiciones e índice.}

