% !TEX root = ../../../Lazcorreta.Tesis.tex
La abundancia de grandes colecciones de datos y la necesidad de extraer de ellos conocimiento desconocido ha lanzado a muchos investigadores de diversas disciplinas al desarrollo de algoritmos y metodologías con el objetivo común de descubrir \patrones ocultos entre los datos~\citep{HanKamberPei-DMConceptsAdnTechniques-2011}. Una vez vistas las fases del proceso de \wum nos centraremos en las posibilidades que nos aporta la \dm en función de los datos obtenidos tras la fase de transformación, en nuestro caso las \sns. Aún deberemos transformar estas sesiones en otro tipo de datos que sean realmente manejables por los algoritmos de \DM que decidamos aplicar. De momento tenemos un conjunto de $N$ sesiones, cada una de ellas con la siguiente forma:
$$ s_i = \left({userID}, p_1, u_1, p_2, u_2\ldots p_{n_i-1}, u_{n_i-1}, p_{n_i}\right), i = 1\ldots N $$

Estas \sns pueden ser representadas de diferentes modos, destacando dos modelos matemáticos sobre el resto: las \secuencias y las \transacciones. En las \secuencias es importante el orden en que se visitaron las páginas en una misma \sn, en las \transacciones sólo nos interesamos por el hecho de que dos o más páginas han sido visitadas en la misma sesión. En una \secuencia se puede averiguar cuántas veces aparece una página en cada sesión, incluso si el usuario simplemente ha refrescado su navegador (\texttt{F5}) y disponemos de mecanismos para detectarlo, en una \transaccion no se suele guardar esta información por lo que sólo sabremos si una página ha sido visitada o no durante una sesión.

% Tanto la carga de trabajo como los resultados obtenidos varía notablemente en ambos casos. Aunque el primer planteamiento es más completo (de hecho incluye la información necesaria para obtener todos los resultados aportados por el segundo planteamiento) al trabajar con recursos limitados es posible que obtengamos menos conocimiento pues no disponemos de recursos ilimitados y el análisis de \secuencias es computacionalmente más complejo, a la par que más completo.

% Aunque ya hemos superado la fase de \emph{transformación} aún no tenemos los datos idóneos para ser procesados por una máquina que sólo es capaz de analizar simultáneamente mucha información y operar con ella.

Decidimos comenzar por el estudio de \secuencias para poder analizar qué importancia tiene el orden de navegación en el análisis de las \sns de nuestro sitio web~\citep{AgrawalSrikant-MiningSequentialPatterns-1995,SrikantAgrawal-MiningSequentialPatternsGeneralizationsAndPerformandeImprovements-1996,BorgesLevene-DataMiningOfUserNavigationPatterns-1999}. Es de esperar que cuanta más información analicemos más conocimiento adquiriremos por lo que sugerimos el uso de \mnw para representar las \sns y analizarlas mediante técnicas de Teoría de Grafos.
