% !TEX root = ../../Lazcorreta.Tesis.tex

Un \portalWeb que sea capaz de adaptar su contenido a las necesidades de sus usuarios es un reto para los investigadores desde la popularización de la web hasta la actualidad. Para personalizar un sitio web, el desarrollador del sitio debe conocer las necesidades de sus usuarios. \citeauthor{PerkowitzEtzioni-AdaptiveSites-1997} (\cite*{PerkowitzEtzioni-AdaptiveSites-1997}, \cite*{PerkowitzEtzioni-AWSAnAIChallenge-1997}, \cite*{PerkowitzEtzioni-AWSAutomaticallySynthesizingWebPages-1998}, \cite*{PerkowitzEtzioni-AdaptiveWebSites-2000}) proponen el término "`sitios adaptativos"' para designar los sitios web que adaptan semi-automáticamente sus páginas aprendiendo de los \patrones de acceso de los propios usuarios.

Según~\citet{Nielsen-PersonalizationIsOverRated-1998}:
\selectlanguage{english}
\begin{quotation}
%\em
  Web personalization is much over-rated and mainly used as a poor excuse for not designing a navigable website . The real way to get individualized interaction between a user and a website is to present the user with a variety of options and let the user choose what is of interest to that individual at that specific time. If the information space is designed well, then this choice is easy, and the user achieves optimal information through the use of natural intelligence rather than artificial intelligence. In other words, I am the one entity on the world to know exactly what I need right now. Thus, I can tailor the information I see and the information I skip so that it suits my needs perfectly.
\end{quotation}
\selectlanguage{spanish}

Nielsen pone en evidencia que la correcta personalización de una herramienta, un \portalWeb en nuestro caso, está estrechamente relacionada con las expectativas del usuario. Algo difícil de lograr por el hecho de que los usuarios no tienen los mismos objetivos, deseos o necesidades al visitar el mismo \portalWeb. El diseño de la herramienta no está en sus manos pero las herramientas que estamos analizando en esta investigación son fácilmente moldeables en su diseño y contenido por los administradores y desarrolladores del \portalWeb.

\citeauthor{PerkowitzEtzioni-TowardsAdaptiveWebSites-1999} (\cite*{PerkowitzEtzioni-AWSAutomaticallySynthesizingWebPages-1998}, \cite*{PerkowitzEtzioni-TowardsAdaptiveWebSites-1999}, \cite*{PerkowitzEtzioni-AWSConceptualClusterMining-1999}, \cite*{PerkowitzEtzioni-TowardsAdaptiveWebSites-2000}) proponen personalizar un sitio web mediante su algoritmo \texttt{PageGather}, que crea dinámicamente índices con enlaces a otras páginas por las que podría estar interesado el usuario. Se trata de un caso de estudio de sus "`sitios adaptativos"', enfocado a la recomendación en tiempo real de las páginas que otros usuarios del sitio web visitan junto a la que está visitando nuestro usuario.

La \WWW se ha convertido en un lugar en que cualquiera puede realizar gran variedad de tareas o utilizar multitud de servicios. Cuando se diseña un \portalWeb se ha de tener en mente que las personas usan herramientas para conseguir sus objetivos con la máxima eficacia y eficiencia~\citep{ConstantineLockwood-SoftwareForUse-1999}, y un \portalWeb puede ser considerado como una herramienta con la que los usuarios realizan sus tareas. Para personalizar un \portalWeb necesitamos recoger información sobre los usuarios que lo visitan e intentar ofrecer una página web ajustada a las necesidades de nuestros visitantes~\citep{Nielsen-PersonalizationIsOverRated-1998}. El principal objetivo de la personalización ha de ser la satisfacción del usuario por lo que el proceso de personalización no debe olvidar el fin de la usabilidad, la personalización de la web no es una excusa para diseños pobres. Esto es posible si los usuarios encuentran un \portalWeb con un buen diseño de sus páginas, sus contenidos y, en general, con un buen diseño del sitio donde los principios de usabilidad y accesibilidad se hayan tenido en cuenta durante el proceso de diseño del \portalWeb~\citep{Nielsen-DesigningWebUsability-2000}.

Si fueramos capaces de ofrecer a nuestros usuarios información personalizada podríamos satisfacerles pues obtendrían lo que realmente buscan en el menor tiempo posible, o al menos en un tiempo asumible~\citep{DuyneLandayHong-TheDesignOfSites-2002}.

Si un usuario no tiene fácil acceso a sus objetivos en un sitio web es muy probable que busque satisfacer sus necesidades en otro sitio web. Pero satisfacer las necesidades de diferentes usuarios no es posible sin tener información exahustiva sobre sus objetivos y métodos de navegación. El desarrollador del sitio web puede esforzarse en conseguir un diseño que a él o a un grupo de usuarios de prueba les resulte cómodo y fácil de usar para la consecución de sus objetivos, pero la globalización de la web permite cada vez más el acceso al sitio web de usuarios de diferentes culturas, nivel de estudios o intereses y le permiten hacerlo desde diferentes dispositivos. La personalización del sitio web debería conseguir que un usuario particular sienta que el sitio se adapta a sus necesidades, con lo que es más fácil que cuando quiera utilizar un servicio de las características del sitio web en cuestión elija éste en lugar de otro que no se adapte a su modo de trabajar.

Todos los usuarios generan información al navegar a través de un sitio web. Cada uno de ellos tiene un objetivo al acceder al sitio web y lo intenta llevar a cabo, con mayor o menor eficiencia, a través del uso de los hiperenlaces facilitados por el desarrollador del sitio web. Es el usuario quien decide qué enlaces utilizar para lograr su objetivo, pero está sujeto al diseño del sitio web: aunque quisiera acceder desde la página $A$ directamente a la página $C$ no podrá hacerlo si el desarrollador no ha incluido un enlace a $C$ en la página $A$. Según \citet{KimCho-PersonalizedMining-2007} la personalización de los servicios de búsqueda de la \WWW es ya una realidad, presentan su propio sistema utilizando diferentes métodos de \DM en un proceso de \KDD.

Para personalizar un \portalWeb, pues, necesitamos recoger toda la información que seamos capaces de convertir en conocimiento sobre su uso. Sin embargo los administradores del portal deben preservar la privacidad y confidencialidad de los datos recogidos~\citep{W3CP3P,LamFrankowskiRiedl-DoYouTrustYourRecommendations-2006,RozenbergGudes-ARMInVerticallyPartitionedDBs-2006}. {P3P 1.0}, desarrollado por el \textsl{World Wide Web Consortium}, surgió como un estándar de la industria que proporciona una forma sencilla y automatizada para que los usuarios tengan más control sobre el uso de información personal en los sitios web que visitan. Cubren nueve aspectos de la privacidad en línea. Cinco de ellos giran en torno a los datos que pueden ser recogidos por el sitio web: \textsl{¿Quién está recogiendo estos datos?} \textsl{¿Exactamente qué información está siendo recopilada?} \textsl{¿Para qué fines?} \textsl{¿Qué información se comparte con los demás?} y \textsl{¿Quiénes son estos receptores de datos?}. Los cuatro aspectos restantes explican las políticas internas de privacidad del sitio: \textsl{¿Los usuarios pueden realizar cambios en cómo se utiliza su información?} \textsl{¿Cómo se resuelven los conflictos?} \textsl{¿Cuál es la política de retención de datos?} y, finalmente, \textsl{¿Dónde se puede encontrar las políticas detalladas en "`forma legible por humanos"'?} 

Para garantizar el cumplimiento de estas recomendaciones decidimos hacer uso de nuestra metodología sobre los datos registrados por los servidores web en un archivo especial llamado \flog~\citep{W3Clogfile}, donde se recoge sólo información anónima acerca de las acciones de los usuarios en un \portalWeb.

Hasta aquí hemos visto que la personalización de la web es posible con la confluencia de muchos actores:
\begin{itemize}
	\item Administradores que conozcan las condiciones de privacidad del sitio web y puedan controlar su aplicación.
  \item Desarrolladores y diseñadores que puedan presentar a cada usuario una página web personalizada, en función de la información proporcionada por el usuario y de los \patrones descubiertos por otros especialistas.
  \item Especialistas en \UX que analicen la usabilidad del la página web personalizada obtenida.
  \item Especialistas en las áreas cubiertas en el \portalWeb que evalúen la calidad de las recomendaciones.
  \item Especialistas en \dm capaces de analizar los datos de uso del sitio web y ofrecer \patrones de comportamiento que puedan ser utilizados por los desarrolladores del sistema.
\end{itemize}

La metodología expuesta permitirá al administrador del sitio web analizar el comportamiento de los usuarios del sitio de forma permanente, lo que le permitirá rediseñar en todo o en parte el sitio web basándose en el \emph{diseño centrado en el usuario}, "`\emph{El diseño de la funcionalidad de un sitio debe estar dirigido por y para los usuarios.}"'~\citep{BaezaYatesRivera-UbicuidadYUsabilidadEnLaWeb-2003}. En servicios con gran cantidad de usuarios se puede esperar que sean éstos quienes ayuden en su diseño, sin embargo la gran cantidad de sitios web dedicados a la misma temática hace que sea más fácil que el usuario cambie de servicio antes de emplear su tiempo en adaptar el que está utilizando, el administrador de un sitio web ha de ser capaz de adaptar los enlaces a las necesidades o preferencias de sus usuarios sin necesidad de su participación en el proceso~\citep{FinkKobsaNill-UserOrientedAdaptivity-1996,JoachimsFreitagMitchell-WebWatcherATourGuideForTheWorldWideWeb-1997,FerreJuristoWindlConstantine-UsabilityBasicsForSoftwareDevelopment-2001}.

Aunque todos los actores tienen gran importancia en la obtención de un \SRW de calidad, en este trabajo nos centraremos en el aporte de los especialistas en \dm. Analizaremos los tipos de datos con que hemos de trabajar y prepararemos estructuras que posibiliten su gestión de forma eficiente y realizable en un gran número de sistemas informáticos.% El resto de actores volverá a aparecer en algún párrafo del capítulo~\ref{chap:4-ConclusionesYTrabajoFuturo} si es necesaria su presencia en las ideas que se han quedado en el camino de esta investigación.