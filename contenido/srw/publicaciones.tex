% !TEX root = ../../Lazcorreta.Tesis.tex
La investigación científica ha de divulgarse, es el mejor modo que tenemos para recibir retroalimentación de la comunidad científica. En este primer periodo nuestra investigación se difundió a través de tres trabajos, aceptados para su exposición en dos Congresos Internacionales y un Simposio. Con el primero de ellos presentamos en Las Vegas (\urlConNotaAlPie{http://www.hci.international/index.php?module=conference&CF_op=view&CF_id=4}{HCII'05}) el trabajo expuesto en la sección~\ref{sec:srw:md:mnw}. La investigación mostrada en la sección~\ref{sec:srw:md:ar} dio lugar a dos ponencias en Granada (\urlConNotaAlPie{http://aipo.es/congresos/interaccion/interacción-2005}{Interacción'05} y \urlConNotaAlPie{http://cedi2005.ugr.es/2005/simposio_s16_sico.shtml}{SICO'05}).



\phantomsection
\subsection*{Personalization through Inferring User Navigation Maps from Web Log Files, 2005}
\addcontentsline{toc}{subsection}{Actas de {HCII}'05}
\label{sec:nuestro-Personalization-2005}

%~\citet{Botella:Lazcorreta:Fernandez:Gonzalez:Personalization:05}
\begin{quote}
  Botella Beviá, F., Lazcorreta Puigmartí, E., Fernández-Caballero, A. y González López, P. Personalization through Inferring User Navigation Maps from Web Log Files. \emph{Proceedings of the 11th International Conference on Human-Computer Interaction}, Las Vegas, 2005. \leePDF{bib/nuestros/BotellaLazcorretaFCaballeroGonzalez-Personalization-CIO-jul05.pdf} \descargaCIO{http://cio.umh.es/files/2011/12/CIO_2005_03.pdf}
\end{quote}

\begin{quotation}
	\noindent\textbf{Resumen}

\selectlanguage{english}
	\nopagebreak One of the most challenging areas in human computer interaction is personalization. Nonetheless, usable and well design sites can not be replaced by personalization, which must be used in just measure. In order to avoid dealing with personal data from users and thus not to worry about privacy, we decide to work with anonymous data. Moreover, these data can be found in all web server logs. In this paper, we propose a methodology for simple personalization. Amazon.com does with book recommendations based only on anonymous data about user's navigation in the website. Our target is to facilitate the navigation of users by means of suggested links towards the next page that users likely want to go. And this can be accomplished in real time using an intelligent agent system or at user requirements design by running a specific process in the system that offers instantaneously the suggested links.
\end{quotation}
\selectlanguage{spanish}





\phantomsection
\subsection*{Mejora de la usabilidad y la adaptabilidad mediante técnicas de Minería de Uso Web, 2005}
\addcontentsline{toc}{subsection}{Actas de Interacción'05}
\label{sec:nuestro-Mejora-2005}

%\citet{Botella:Lazcorreta:Fernandez:Gonzalez:MejoraDeLaUsabilidad:05}
\begin{quote}
  Botella Beviá, F., Lazcorreta Puigmartí, E., Fernández-Caballero, A. y González López, P. Mejora de la usabilidad y la adaptabilidad mediante técnicas de Minería de Uso Web. \emph{Actas del VI Congreso Interacción Persona-Ordenador}, 299--306, Granada, 2005. \leePDF{bib/nuestros/BotellaLazcorretaFernandezGonzalez-MejoraDeLaUsabilidad-sep05.pdf} \descarga{http://www.researchgate.net/profile/Antonio_Fernandez-Caballero/publication/228963180_Mejora_de_la_usabilidad_y_la_adaptabilidad_mediante_tcnicas_de_minera_de_uso_Web/links/0912f509c0f3f3f63d000000?origin=publication_detail}
\end{quote}

\begin{quote}
%	\noindent
	\textbf{Resumen}

	\nopagebreak Una de las áreas más activas en la Minería de Uso Web es la \prediccion. La personalización de un sitio web es uno de los objetivos fundamentales de la \prediccion. Si se estudia el comportamiento de los usuarios de un sitio web, es posible conocer qué páginas suelen visitar y cuáles son las siguientes páginas que visitan. Sería de gran ayuda para los usuarios si pudieran tener los dos o tres enlaces a las páginas que más suelen visitar en un sitio web. En este trabajo proponemos un método, basado en el algoritmo \apriori, que permite inferir las páginas web que un usuario visita a partir de los datos registrados en los \flogs del servidor web de visitas previas. Estos enlaces serán sugeridos al usuario en un área concreta de la página web lo que permitirá mejorar la usabilidad y adaptabilidad del sitio web.
\end{quote}





%\citet{Lazcorreta:Botella:Fernandez:Gonzalez:TecnicaDeMineria:05}
\phantomsection
\subsection*{Técnica de minería de datos para la adaptación de sitios Web, 2005}
\addcontentsline{toc}{subsection}{Actas de {SICO'05}}
\label{sec:nuestro-Tecnica-2005}

\begin{quote}
  Botella Beviá, F., Lazcorreta Puigmartí, E., Fernández-Caballero, A. y González López, P. Técnica de minería de datos para la adaptación de sitios Web. \emph{Actas del Simposio de Inteligencia Computacional ({SICO})}, 455--462, Granada, 2005. ISBN 84-9732-444-7. \leePDF{bib/nuestros/LazcorretaBotellaFCaballeroGonzalez-TecnicaDeMineria-sep05.pdf} \leePDF{bib/nuestros/LazcorretaBotellaFCaballeroGonzalez-TecnicaDeMineria-POSTER-2005.pdf} \descarga{http://www.dsi.uclm.es/personal/AntonioFdez/nais/nais/investigacion/publicaciones/congresos_2005/SICO2005.pdf}
\end{quote}

\begin{quote}
%	\noindent
	\textbf{Resumen}

	\nopagebreak En este trabajo presentamos el uso de técnicas de \dm y la búsqueda de técnicas de \prediccion sobre los \flogs de un servidor web para mejorar la facilidad de uso del sitio. Se intenta adaptar el sitio web a las necesidades reales de los usuarios, de modo que el usuario reciba información detallada que le permitirá "`recordar"' qué páginas visitó en anteriores ocasiones donde también solicitó la página actual, así como información sobre el uso general del sitio web.
\end{quote}
