% !TEX root = ../../../Lazcorreta.Tesis.tex

La informática ayuda a virtualizar la realidad, las distintas ciencias determinan qué características tienen los individuos de cualquier población pero no podemos obtener todas esas características de la realidad ya que muchas de ellas aún no han sido descubiertas o no tenemos procedimientos para medirlas correctamente. Con los medios y métodos disponibles se ha de decidir qué parte de la realidad se guardará en formato digital mediante \dbs (\DB), representada por \texttt{Data} en la figura~\ref{fig:fasesProcesoKDDFayyad}.

La mayoría de entidades (empresas públicas y privadas, servicios\ldots) disponen de \dbs con cada vez más información sobre la propia entidad, sus usuarios o productos. Esta información crece constantemente en diferentes dimensiones: nuevos usuarios, nuevas características mensurables sobre cada usuario, nuevas interacciones con cada usuario, distribución geográfica de nuevas sucursales de la entidad, desaparición de sucursales\ldots

Estas grandes \dbs no sirven de nada si no son explotadas convenientemente. Pero las magnitudes que van adquiriendo a lo largo del tiempo, su dinamismo y constante crecimiento en diferentes dimensiones hacen inviable un estudio completo y comprensible de toda la información que contienen. Es necesario convertir los datos disponibles (\texttt{Data}) en conocimiento (\texttt{Knowledge}) para lo que es necesario planificar un proceso de \KDD, o incluso varios procesos según el volumen de información del que se disponga y los objetivos perseguidos con su análisis.

En la \wum se busca conocer cómo se usa la web. Si nos centramos en un sitio web, los datos disponibles pueden ser tanto de la estructura como del contenido o del uso del sitio web o incluso de enlaces a otros sitios web. Los datos sobre estructura y contenido tienen mucha dependencia de los administradores, diseñadores y redactores de contenido del sitio web, sin embargo los datos de uso pueden guardar cierta independencia respecto a los actores mencionados. La información generada por el uso del sitio web puede ser almacenada por el servidor en que se aloja el sitio web mediante \flogs, de los que nosotros usamos los de formato \ELogFF cuyas especificaciones se encuentran en~\citet{W3Clogfile}. Se trata de ficheros de texto plano con una línea por cada elemento solicitado al servidor por sus usuarios, línea en que aparece información sobre la IP que ha solicitado el elemento, instante de la solicitud, URI del elemento solicitado, protocolo, estado y otros datos que no van a ser utilizados en este trabajo.
