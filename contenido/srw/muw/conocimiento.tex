% !TEX root = ../../../Lazcorreta.Tesis.tex
% \ABIERTO%\clearpage

El \textsl{conocimiento} es el resultado del proceso de \KDD. No es algo único que una vez adquirido pueda ser anotado en un libro para no necesitar ser "`calculado"' de nuevo. Los datos originales cambiarán, aumentarán o dejarán de tener relevancia por la desaparición de un artículo, su análisis puede dar como resultado nuevo \textsl{conocimiento}, complementario, independiente o contrario al adquirido en procesos previos. Su correcta gestión es la que logra una adaptación real de los servicios de una empresa a los usuarios del servicio.

El proceso de \KDD ha de realizarse continuamente, estudiar el acierto de los parámetros usados en sus distintas fases y hacer variaciones de los mismos para mejorar la calidad del conocimiento adquirido, comprobar el dinamismo de la población en estudio y el efecto que producen los cambios realizados al servicio. No siempre se puede utilizar todo el conocimiento adquirido, hay que decidir qué cambios son realmente aplicables en la empresa que ha realizado el proceso de descubrimiento de conocimiento.

En un \srw se busca conocer lo que quiere el usuario antes de que éste lo solicite para poder preparar enlaces a las páginas que se le van a sugerir en función de las que el usuario vaya visitando. Un sitio web con muchas páginas debería tener una estructura de navegación bien diseñada para que los usuarios pudieran obtener con facilidad las páginas de su interés. Este diseño suele ser realizado por expertos en contenidos o en diseño web, y a menudo no tienen en cuenta los aspectos de usabilidad y/o adaptabilidad de un sitio web~\citep{PenarrubiaFCaballeroGonzalez-PortalesWebAdaptativos-2004}. El usuario real del sitio nunca interviene en este diseño, si deseara las páginas $A$, $F$ y $H$, pero el experto decide que existe una relación más directa entre las páginas $A$, $B$ y $C$, éstas serán las páginas con enlaces de navegación y no aquellas que realmente el usuario quería visitar. Esto ocasionará que el usuario emplee mucho tiempo en encontrar lo que busca o en el peor de los casos que no lo encuentre, a no ser que se aplique un buen proceso de \wum que permita la adaptación progresiva de las recomendaciones del sitio web, lo que podría derivar en un pleno conocimiento de cómo usan el sitio web sus usuarios y una perfecta adaptación del sitio web a sus usuarios.

