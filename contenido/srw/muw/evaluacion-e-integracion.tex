% !TEX root = ../../../Lazcorreta.Tesis.tex
% \ABIERTO%

La evaluación e interpretación de los \patrones obtenidos, teniendo en cuenta los objetivos planteados en el proceso completo, ha de poner el foco en la comprensión y utilidad del modelo obtenido. Se debe documentar correctamente en esta fase el conocimiento adquirido para poder utilizarlo en posteriores procesos. El primer problema que se presenta en esta fase es el exceso de conocimiento adquirido. Hemos cambiado una gran cantidad de datos iniciales por una gran cantidad de conocimiento ¿podremos analizar todo este conocimiento en profundidad? Un sistema informático puede ayudar al analista del proceso en este punto, como apuntan~\citet{LiuHsu-PostAnalysisOfLearnedRules-1996}, utilizando diferentes criterios para decidir qué conocimiento es más \emph{útil} en el caso concreto que estamos evaluando.

La integración de este conocimiento en otro sistema que sea capaz de obtener provecho del mismo es el objetivo final de cualquier proceso de \KDD. El proceso se ha realizado a partir de una "`foto fija"' de la población en estudio, los datos obtenidos y seleccionados en las primeras fases. Ahora han de integrarse en un sistema que sea capaz de traducir el conocimiento adquirido en beneficios para los usuarios y los propietarios del sistema (mayor usabilidad, personalización, recomendaciones más acertadas, ventas\ldots). Esta transición no es necesariamente sencilla pero se escapa al proceso expuesto en este trabajo.

La última fase del proceso es la que da validez al mismo: integrar el conocimiento al sistema de modo que pueda ser usado y evaluar los resultados obtenidos. En este punto se ha de comprobar qué conocimientos se tenían al principio del proceso y qué conocimientos nuevos se han adquirido tras el análisis de los datos.

Al afrontar un proceso de \wum se hace bajo la creencia de que un usuario no puede decidir sobre el diseño del sitio, pero muchos usuarios sí podrían decidir. Si un gran número de usuarios visitaran conjuntamente las páginas $A$, $F$ y $H$, el webmaster podría definir los enlaces entre ellas, lo que mejoraría su rango de alcance. Una mejora en la facilidad de uso del sitio web que no requiere la intervención del webmaster es la \prediccion: presentar a cualquier usuario que visite la página $A$, enlaces a las páginas $F$ y $H$, aunque realmente no se encuentren en el diseño inicial de la página $A$. Para lograrlo se debe estudiar previamente el uso del sitio, detectando si un porcentaje considerable de usuarios que solicitan el recurso $A$ también solicitan los recursos $F$ y $H$ en la misma sesión~\citep{NgHan-EfficientAndEffectiveClusteringMethods-1994}. Existen varios trabajos planteando diversas alternativas a esta solución \citep{SuYangLuZhang-WhatNext-2000,SunChenWenyinMa-IntentionModelingForWebNavigation-2002,ChenZhang-APopularityBasedPredictionModelForWebPrefetching-2003}, todos ellos con un esquema común: considerar la información que aportan todos los usuarios, analizarla más o menos exhaustivamente para obtener \patrones de comportamiento de los usuarios en general y resumirla de modo que con pocos recursos podamos ayudar al máximo al usuario anónimo, mediante la \prediccion del próximo recurso que solicitará.

Esta fase no la pudimos llevar a cabo durante el transcurso de esta investigación por no tener acceso al código de un sitio web real en explotación. Si no podemos modificar dinámicamente las páginas sugeridas por el sitio web a sus usuarios en función de las páginas que está visitando no podremos contrastar los datos iniciales con los nuevos datos. No podemos observar si una página ha dejado de ser visitada por servir únicamente para transitar entre dos páginas entre las que no existía originalmente un enlace directo. No podemos observar si un enlace ha dejado de usarse por ser inicialmente un enlace que los usuarios esperaban les llevara a su objetivo debido al texto engañoso del enlace. No podemos observar si los usuarios dejan de abandonar nuestro sitio web por encontrarlo más adaptado a su modo de trabajar.

Esta fase es fundamental para un \srw por lo que este informe sería incompleto si no hubiéramos cambiado el rumbo de nuestra investigación, dedicando nuestros esfuerzos a la tarea de \dm del proceso inicial abordado.
