% !TEX root = ../../../Lazcorreta.Tesis.tex
%TODO: He cambiado el principio, sólo hablo de MUW y debería mencionar algo de MCW y MEW
La personalización de la web ha generado muchas investigaciones debido a la capacidad de las páginas web para contener datos (noticias, artículos, palabras, imágenes\ldots) y enlaces internos o a otras páginas web. Si se escriben en cada página los enlaces más adaptados al usuario que la está visitando es muy probable que su \ux mejore. Esto genera cambios continuos en la estructura del sitio web, alojando cada vez más páginas con contenido más variado. Al tratarse siempre de solicitudes de un cliente (usuario) a un servidor (sitio web) se guarda automáticamente mucha información sobre cada solicitud de página hecha por un usuario. Cuando un sitio web comienza a guardar estos datos es fácil observarlos mediante sencillos gráficos y tablas y mejorar la estructura y el contenido del sitio web. Sin embargo el número de datos recogido puede ir creciendo día a día, hora a hora, en función de la popularidad del sitio web. Cuando ya comienzan a ser muchos los datos guardados sobre el uso del sitio web comienzan a ser intratables mediante esos sencillos gráficos y tablas, perdemos gran parte de la información que contienen los datos por no estar usando las herramientas adecuadas para analizarlos. A grandes rasgos son las características que hicieron aparecer la \wm (\WM)~\citep{CooleyMobasherSrivastava-WebMining-1997,Pitkow-InSearchOfReliableUsageDataWWW-1997,KosalaBlockeel-WebMiningResearch-2000,Scime-WebMiningApplicationsAndTechniques-2004} y sus tres especialidades: \wsm (\WSM), \wcm (\WCM) y \wum (\WUM). Todas ellas enfocan el proceso de \KDD con el propósito de extraer el basto conocimiento que encierran los datos gestionados en la \WWW. Todas ellas son necesarias para lograr la personalización de la web, cada una aporta conocimiento útil para las restantes.

En nuestra investigación nos centraremos en la \wum, que como todo proceso de \KDD puede llevarse a cabo mediante la ejecución consecutiva y recursiva de diferentes tareas sobre los datos disponibles y los resultados intermedios obtenidos en el proceso (figura~\ref{fig:fasesProcesoKDDFayyad}). En cada una de las siguientes secciones especificaremos qué tipo de datos están disponibles y qué tareas se pueden aplicar en este proceso para convertirlos en conocimiento útil que facilite la personalización de la web\citep{SrivastavaCooleyDeshpandeTan-WUMDiscoveryAndApplicationsOfUsagePatterns-2000,MobasherCooleySrivastava-AutomaticPersonalizationBasedOnWUM-2000,DeBraAroyoChepegin-NextBigThingAdaptiveWebBasedSystems-2004}.
