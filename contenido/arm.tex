% !TEX root = ../Lazcorreta.Tesis.tex

% \ABIERTO%

La línea de investigación se va definiendo poco a poco. En sus inicios teníamos la intención de crear un \srw (\SRW) capaz de adaptarse al usuario en sitios web con gran cantidad de páginas y de usuarios generando gran cantidad de datos sobre su uso. Lo planteamos como un proceso de \wum (\WUM), un proceso específico de \kdd (\KDD) en el que los datos y el conocimiento final giran en torno al uso de la \www (\WWW). Como en todo proceso de \KDD, la \WUM se divide en cinco tareas secuenciales y recurrentes: \emph{Selección}, \emph{Preproceso}, \emph{Transformación}, \dm e "`\emph{Integración y Evaluación}"'. Las tres primeras tareas se han expuesto a fondo en la sección~\ref{sec:srw:muw}, la \dm ha ocupado un papel especial en la sección~\ref{sec:srw:md} pero no se ha dejado resuelta y no ha sido posible llevar a cabo la parte de integración en un sitio web real que genere gran cantidad de datos de uso, por lo que no ha sido posible evaluar el conocimiento adquirido.

La \dm comienza a tomar protagonismo en nuestra investigación. Las \ARs son relaciones muy interesantes y los algoritmos que las obtienen están recibiendo cada vez más atención de la comunidad científica, como muestran las revisiones y comparaciones elaboradas por~\citet{HippGuntzerNakhaeizadeh-AlgorithmsForAssociationRuleMining-2000}, ~\citet{ZhaoBhowmick-ARMSurvey-2003} y~\citet{Goethals-SurveyOnFPM-2003}. Aparecen decenas de algoritmos y estructuras de datos para gestionar los problemas que van surgiendo conforme se va avanzando en esta nueva disciplina donde la mayor dificultad está en administrar correctamente los recursos disponibles para estudiar las cada vez más grandes cantidades de datos a analizar. Nosotros mismos proponemos en la sección~\ref{sec:srw:md:ra} algunos cambios sobre uno de los algoritmos más destacados en la \arm, \apriori. Decidimos profundizar en este elegante algoritmo, complicarlo lo justo para que se pueda adaptar a nuestras necesidades y que pueda incorporar nuestras aportaciones. Decidimos poner a prueba nuestras propuestas para lo que había que desarrollar el código y comenzamos a observar más a fondo el código que otros investigadores habían puesto a nuestra disposición a través de su publicación en las actas de FIMI'03~\citep{ZakiGoethals-ProceedingsFIMI-2003}.

Al analizar a fondo el algoritmo \apriori descubrimos que la generación de \ars se plantea de un modo en que se repiten gran cantidad de cálculos. Propusimos un algoritmo alternativo para el sugerido originalmente por~\citet{AgrawalSrikant-FastAlgorithmsForMiningAssociationRules-LARGO-1994} en el que se ahorran muchos costosos procesos de búsqueda lo que redunda en ganancia de tiempo sin pérdida de ningún tipo de información.

Conforme más profundizamos en el estudio de \ARs más vemos la necesidad de reducir las dimensiones de los datos a tratar. Podemos abordar el estudio completo de "`pequeñas colecciones"' de datos utilizando \apriori para obtener resultados en tiempo real. Pero cuando nos enfrentamos a grandes colecciones de datos siempre hemos de renunciar al análisis de muchos de ellos. En este capítulo intentaremos utilizar \apriori para dividir los datos iniciales en grupos más homogéneos en función de las reglas que cumpla cada registro. Si el \clustering obtenido es correcto se reducirán enormemente las necesidades de recursos pues en cada grupo habrá un número menor de ítems a procesar, podríamos obtener más y mejor información de cada uno de los grupos a partir de los mismos datos.

Otro de los puntos de interés de esta investigación está motivado por la existencia del \dilemaIR. En el capítulo anterior lo resolvimos parcialmente estudiando únicamente las transacciones (sesiones) que procedían de un mismo usuario con lo que se reducía notablemente el número de datos a procesar y se podía llevar a cabo un análisis individual en tiempo real para alimentar un \SRW. Sin embargo este planteamiento no puede aplicarse a grandes colecciones de datos por los problemas de desbordamiento de memoria ya citados. Este capítulo finaliza con una aportación con la que pretendemos aliviar este problema utilizando toda la información disponible en \D, posibilitando el análisis de los \irs mejor relacionados con los ítems frecuentes y su uso en un \SR mediante la definición de \ROPs.




\section{Conceptos básicos}
\label{sec:arm:conceptos-basicos}
% !TEX root = ../../Lazcorreta.Tesis.tex
% \ABIERTO%
\label{sec:arm:introduccion}
% !TEX root = ../../../Lazcorreta.Tesis.tex
% \ABIERTO%
\citet{AgrawalImielinskiSwami-MiningAssociationRulesBetweenSetsOfItemsInLargeDB-1993} son pioneros en la \arm. La definición formal que dan a a esta disciplina es muy elemental\footnote{He adaptado la nomenclatura a la utilizada en este informe.}:

\selectlanguage{english}
\begin{quote}
  Let $\I = \{I_1, I_2, \ldots I_N\}$ be a set of binary attributes, called items. Let \D be a database of transactions. Each transaction $t$ is represented as a binary vector, with $t[k] = 1$ if $t$ bought the item $I_k$, and $t[k] = 0$ otherwise. There is one tuple in the database for each transaction. Let $X$ be a set of some items in \I. We say that a transaction $t$ satisfies $X$ if for all items $I_k$ in $X$, $t[k] = 1$.
  
  By an association rule, we mean an implication of the form 
  $$X \rightarrow I_j$$
  where $X$ is a set of some items in \I, and $I_j$ is a single item in \I that is not present in $X$. The rule $X \rightarrow I_j$ is satisfied in the set of transactions \D with the confidence factor $0 \leq c \leq 1$ iff at least $c\%$ of transactions in \D that satisfy $X$ also satisfy $I_j$.
  
  Given the set of transactions \D, we are interested in generating all rules that satisfy certain additional constraints of two different forms: Syntantic Constrains and Support Constrains.
\end{quote}
\selectlanguage{spanish}

Definen \I como un conjunto de $N$ atributos binarios y \D como una \db de \transacciones, vectores binarios de $N$ elementos cuyo valor $k$-ésimo indica si el ítem $I_k$ está o no presente en la \transaccion. El \consecuente de una \AR es un ítem de \I no presente en el \antecedente de la regla. Las investigaciones posteriores sobre este tema amplían esta definición para estudiar reglas en que el \consecuente sea un \kitemset, como se indica en la definición~\ref{def:1-3-2-AR}. Por último definen la \arm como la disciplina que busca en un gran almacén \D las \ARs que cumplen ciertas restricciones, bien de tipo sintáctico (como la presencia de cierto ítem en el antecedente o el consecuente) o de \soporte (obteniendo únicamente las reglas más frecuentes).

Las restricciones no son imposiciones si no necesidades. Si \I está formado por $N$ ítems, para tratar una transacción tendremos que utilizar $N$ valores (ceros o unos), y se podrán obtener hasta $2^N$ \itemsets distintos\footnote{Si $N=100$ tendremos $2^{100}\approx1.3\cdot10^{30}$ posibles \itemsets en las \transacciones si no contamos repeticiones, lo que da una idea del número de relaciones que se pueden encontrar entre los \itemsets de \D cuando éste es muy grande y el número de ítems distintos también lo es.}, pudiendo actuar como \antecedente de una \ar $2^N-2$ al descontar $\emptyset$ e \I. Considerando que cada \kitemset[1] puede ser el \antecedente de hasta $2^{N-1}-1$ \ars distintas, cada \kitemset[2] puede ser el \antecedente de $2^{N-2}-1$ reglas distintas\ldots es fácil descubrir que el número de \ARs que se podría obtener es enorme: $\sum_{i=1}^{N-1}{{N \choose i}(2^{N-i}-1)}$. Como la mayoría de datos se ha de manipular en memoria RAM para poder acceder a ellos rápidamente, sólo si trabajamos con pocos ítems y pocas \transacciones podremos evitar las restricciones.





\subsection{Tipo de Datos}
\label{sec:arm:tipo-de-datos}
% !TEX root = ../../../Lazcorreta.Tesis.tex
% \ABIERTO%
Al definir las transacciones como vectores binarios se puede interpretar que es mejor guardar los datos a nivel de bit ya que sólo son necesarios dos valores para representar cada dato, lo que convertiría a \D en una matriz de ceros y unos cuyas $|\D|$ filas representan cada una de las \transacciones y sus $N$ columnas cada uno de los ítems de \I. Es fácil pensar en trabajar con matrices de estas características para aprovechar la potencia de los ordenadores al trabajar a nivel de bit, sin embargo no se plantea nadie este formato hasta el trabajo de~\citet{DongHan-BitTableFIEfficientFIMalg-2007}, que proponen comprimir \D en una {BitTable} y usar el algoritmo~\algoritmo{BitTableFI}. Indican que esta estructura posibilita una rápida generación de candidatos y de su recuento por utilizar funciones de unión e intersección de bits, obteniendo mejor rendimiento que los algoritmos con que se compara. En la sección~\ref{sec:4-bit} volveremos a este planteamiento como propuesta de trabajo futuro.

Trabajar con uniones e intersecciones de bits puede ser muy eficiente para un procesador si se consigue programar de forma eficiente, sin embargo al desarrollar un programa con un lenguaje de alto nivel no es fácil trabajar a nivel de bit (en \langCpp se guarda un valor \texttt{bool} usando un Byte, ocho bits) por lo que la mayoría de implementaciones hechas sobre algoritmos de \arm no utilizarán esta definición de \I. Generalmente se representa el ítem $I_k$ utilizando el número $k$ con lo que las \transacciones no se guardan como $N$-tuplas de ceros y unos si no como conjuntos de números enteros. El formato más utilizado para guardar \D de este modo es escribiendo en un fichero de texto plano cada \transaccion en una línea del fichero, lo que se conoce como \emph{representación horizontal de \D}. Algunos algoritmos están especialmente diseñados para aprovechar la \emph{representación vertical de \D}, en que en cada línea se escribe el par $\TID\ item$ de modo que cada \transaccion ocupe tantas líneas consecutivas como ítems contenga. A lo largo de este informe se tratará \D como un conjunto de números enteros y no una matriz de bits.





\subsection{Primeros Algoritmos}
\label{sec:arm:primeros-algoritmos}
% !TEX root = ../../../Lazcorreta.Tesis.tex
% \ABIERTO%
En este primer artículo proponen el algoritmo \algoritmo{AIS} (ver algoritmo~\ref{alg:AIS}) que se basa en múltiples lecturas de la \db de transacciones \D. Comienzan con un conjunto vacío de \itemsets frecuentes, \aprioriL, y un conjunto de \itemsets al que llaman frontera, $\mathcal{F}$, y que contiene inicialmente el conjunto vacío. En cada lectura de \D, $k=1\ldots$, se realizan los siguientes pasos:

\begin{enumerate}
  \item Se vacía el conjunto de candidatos, \aprioriC.
  \item Se busca cada \kitemset[k-1] de la frontera en cada \transaccion.
  \item Si se encuentra se extiende con todos los ítems de la \transaccion y se añade la extensión al conjunto \aprioriC[k+1], creándolo con \soporte 1 si no existe o incrementando su \soporte si ya existe.
  \item Al terminar la $k$-ésima lectura de \D se vacía la frontera, se guardan en \aprioriL[k] los candidatos que tienen \soporte mínimo y se añaden a la frontera los que se estima que se podrán extender en la siguiente iteración.
  \item Si la frontera no está vacía se vuelve al paso 1.
\end{enumerate}

Empiezan a encontrarse los problemas de recursos intrínsecos a esta disciplina, llegando a proponer un algoritmo que guarda en disco la información que está en memoria y no es necesaria. Este algoritmo se llamará cada vez que se necesite memoria para almacenar nuevos datos, lo que restará algo de eficiencia al proceso completo.

Proponen también una serie de heurísticas para estimar el \soporte de un \kitemset antes de generar su candidato, de modo que si se estima que tendrá un \soporte bajo no se genere el candidato y el sistema no caiga por falta de memoria para almacenar los candidatos. La generación de candidatos se hace bajo la suposición de independencia de los ítems presentes en cada transacción. Si los \itemsets $X$ e $Y$ son disjuntos y tienen \soporte $x$ e $y$ respectivamente, estiman que el \itemset $X \cup Y$ tendrá \soporte $z = x\cdot y$, por lo que esperan que será frecuente en \D si $z \geq minSup$.

% \afterpage{\clearpage}
\lstinputlisting[label=alg:AIS,
                 caption={Algoritmo {AIS}, 1993},
                 float=htbp,
                 basicstyle=\scriptsize]
                {./contenido/arm/codigo/alg-AIS}%

\citet{HoutsmaSwami-SETMofAR-1993} sugieren trabajar directamente sobre las \dbs de \transacciones aprovechando las funciones propias de los \dbms en lugar de desarrollar aplicaciones específicas para trabajar con almacenes \D guardados como texto plano. Proponen el algoritmo \algoritmo{SETM} (ver algoritmo~\ref{alg:SETM}), que genera los candidatos al leerlos en las transacciones, igual que \algoritmo{AIS}, y guarda una copia de cada \itemset junto al \TID de la transacción que lo contiene. Al trabajar con \DB en disco no necesita muchos recursos de memoria para guardar tanta información pero no puede competir en eficiencia con programas preparados específicamente para extraer la misma información de un fichero de texto plano. Sin embargo la asociación de cada \itemset con el \TID de la transacción en que está contenido puede agilizar un estudio más profundo de ciertos \itemsets por lo que no descartamos el uso de esta idea cuando podamos realizar en paralelo estas sentencias que no benefician la eficiencia del propósito actual, encontrar los \itemsets frecuentes presentes en \D, pero cuyos resultados pueden ser guardados en ficheros cuyo análisis posterior puede proporcionar gran cantidad de conocimiento sobre la población en estudio. 

% \afterpage{\clearpage}
\lstinputlisting[label=alg:SETM,
                 caption={Algoritmo {SETM}, 1993},
                 float=htbp,
                 basicstyle=\scriptsize]
                {./contenido/arm/codigo/alg-SETM}

Como se está buscando la co-existencia de ítems en \transacciones se puede aprovechar el orden lexicográfico del código utilizado para representar a cada ítem. El \soporte del \itemset $XY$ coincide con el de $YX$ por lo que si al guardar los \itemsets lo hacemos con sus ítems ordenados lexicográficamente sólo guardaremos el \itemset $XY$. Este orden se va a aprovechar para hacer más eficientes las operaciones a realizar por el algoritmo.

\lstinputlisting[label=alg:SETM-mergeScan,
                 caption={Algoritmo {SETM}, función merge-scan},
                 float=htbp,
                 basicstyle=\scriptsize]
                {./contenido/arm/codigo/alg-SETM-mergeScan}
                
Las funciones del algoritmo son consultas a la \db. Para obtener $R'_k$ se ejecuta la consulta del listado~\ref{alg:SETM-mergeScan}, que extiende cada \kitemset frecuente encontrado en el paso anterior con los ítems frecuentes de \D que sean lexicográficamente mayores que el mayor de los ítems del \kitemset. Concretamente en la extensión de un \kitemset basta con añadir uno a uno los ítems de la \transaccion que contienen al \kitemset y son lexicográficamente mayores al mayor de los ítems del \kitemset.


\aprioriC se obtiene contando los \itemsets del conjunto $R'_k$ y guardando sólo los que tienen \soporte mínimo, como muestra el listado~\ref{alg:SETM-Ck}.

\lstinputlisting[label=alg:SETM-Ck,
                 caption={Algoritmo {SETM}, selección de candidatos},
                 float=htbp,
                 basicstyle=\scriptsize]
                {./contenido/arm/codigo/alg-SETM-Ck}

El conjunto auxiliar $R_k$ lo obtienen seleccionando las tuplas de $R'_k$ que pueden ser extendidas, como muestra el listado~\ref{alg:SETM-Rk}.
\lstinputlisting[label=alg:SETM-Rk,
                 caption={Algoritmo {SETM}, \itemsets extendibles},
                 float=htbp,
                 basicstyle=\scriptsize]
                {./contenido/arm/codigo/alg-SETM-Rk}

Estos primeros algoritmos de \arm utilizan la misma estrategia para evitar desbordamiento de memoria al generar los candidatos a \itemset frecuente. Como el número de candidatos es teóricamente muy grande cuando trabajamos con grandes almacenes \D y muchos ítems en \I, y como es de esperar que el número de \itemsets en \D sea sensiblemente menor que el teórico, en estos algoritmos se propone reservar memoria únicamente para contar los \itemsets que realmente están en \D, es decir, se reserva memoria cada vez que se encuentra un \itemset en \D para el que aún no hemos hecho dicha reserva. Este planteamiento consigue su objetivo a costa de eficiencia, ya cada una de las millones de operaciones de reserva de memoria que se llevan a cabo consume un tiempo y su acumulación resulta en un tiempo de ejecución muy superior al que conseguiríamos si hiciéramos una buena estimación de los \itemsets contenidos en \D y realizáramos la reserva de memoria en una única instrucción. En los algoritmos que se muestran en la sección~\ref{sec:arm:fim} se adoptará esta última estrategia.

En ambos artículos se trata por encima la generación de \ars. En su definición se especifica que el consecuente es un único ítem por lo que una vez conocemos todos los \kitemsets frecuentes basta con recorrerlos uno a uno y separarlos en dos \itemsets, el antecedente con $k-1$ ítems y el consecuente con el ítem restante, con lo que se obtendrán todas las \ARs de \D que tienen \soporte mínimo.




\subsection{Formato de \D}
\label{sec:arm:formato-de-D}
% !TEX root = ../../../Lazcorreta.Tesis.tex
% \ABIERTO%
En la sección~\ref{sec:arm:tipo-de-datos} hemos cuestionado el modo de guardar los datos de \D para el análisis, adelantando que la codificación de los ítems de \I mediante números enteros es el formato más utilizado por los desarrolladores de algoritmos de \ARM. \citet{AgrawalImielinskiSwami-MiningAssociationRulesBetweenSetsOfItemsInLargeDB-1993} no especifican nada sobre este asunto en su artículo, mientras que~\citet{HoutsmaSwami-SETMofAR-1993} proponen el uso de funciones ya implementadas en los \dbmss (\DBMS). Como la \arm es una fase del proceso de \KDD lo más inmediato es usar \dbs para guardar \D, sin embargo la eficiencia de las funciones de un \DBMS es notablemente inferior a la obtenida usando lenguajes de programación compilados si son muchos los datos a procesar.

\citet{AgrawalImielinskiSwami-DatabaseMiningAPerformancePerspective-1993} proponen dotar a los \DBMS de nuevas operaciones bien implementadas para poder hacer \dm sobre \clasificacion, {asociación} y \secuencias directamente sobre la \DB utilizando combinaciones de dichas operaciones para obtener \patrones válidos para los tres modelos abordados.

Hay más investigadores que buscan el uso de \DBMS para generar \ARs, como~\citet{HoutsmaSwami-SetOrientedMiningForAR-1995} que proponen el estudio de \ARM mediante lenguajes nativos de \dbs, SQL concretamente.

Otro trabajo que busca el uso de \DBMS para extraer reglas de asociación es el de \citet{HolsheimerKerstenMannilaToivonen-APerspectiveOnDatabasesAndDataMining-1995}. En su favor está la flexibilidad que permiten los \DBMS para tratar los datos desde múltiples perspectivas, en su contra el tiempo necesario para obtener resultados frente a los algoritmos específicos de \ARM desarrollados mediante lenguajes de programación compilado. Una de las propuestas de este artículo es el uso de la jerarquía de los ítems para reducir el tamaño del repositorio a analizar, concretamente hablan de la cesta de la compra y de descubrir reglas que involucren genéricamente al ítem "`cerveza"' en lugar de usar como ítem cada una de las marcas de cerveza que están en \I, idea que será tomada por otros investigadores para reducir las dimensiones del problema y poder llevar a cabo estudios más amplios.

Aunque los algoritmos presentados de forma teórica a lo largo de esta investigación no especifican el formato físico de los datos a procesar, las comparaciones realizadas sobre los distintos algoritmos no sería correcta si no se experimenta en condiciones similares por lo que en la mayoría de los casos encontraremos implementaciones realizadas con lenguajes de programación compilados. De hecho, los almacenes \D disponibles en \url{http://fimi.ua.ac.be/data/} y \url{https://archive.ics.uci.edu/ml/datasets.html} tienen formato de texto plano, la mayoría de ellos en formato horizontal.




% En estos primeros trabajos se generan los candidatos conforme se va leyendo las \transacciones de \D, lo que supone una gestión de memoria muy dinámica. Conforme se van incrementando las
\subsection{Fases de \ARM}
\label{sec:arm:fases-de-ARM}
% !TEX root = ../../../Lazcorreta.Tesis.tex
% \ABIERTO%

Como se ha visto en la sección anterior y exponen \citeauthor{AgrawalImielinskiSwami-MiningAssociationRulesBetweenSetsOfItemsInLargeDB-1993} en su primer artículo, la \ARM se compone de dos fases bien diferenciadas:
\begin{enumerate}
  \item En primer lugar hay que encontrar los conjuntos de ítems que están presentes en un porcentaje mínimo de \transacciones de \D.
  \item A continuación se descubren las \ARs que se deducen de esos conjuntos.
\end{enumerate}

La primera fase es la que mayores aportaciones científicas ha recibido en la \ARM pues al trabajar con grandes colecciones de \transacciones puede consumir más recursos de los disponibles o tardar más tiempo del adecuado para dar por finalizado el análisis. Cabe destacar que la tecnología utilizada para el desarrollo de los algoritmos expuestos en esta sección ha evolucionado notablemente desde los inicios de esta metodología, las comparaciones realizadas a través de los artículos presentados pueden haber variado debido a que actualmente tenemos a nuestra disposición recursos de memoria y de cómputo muy superiores a los utilizados en las dos décadas precedentes. En la sección~\ref{sec:arm:fim} se estudian las propuestas más interesantes sobre esta fase, \fim, y se muestra una comparativa sobre la implementación del algoritmo más utilizado en la \arm.

La segunda fase es elemental y no consume tiempo ni recursos excesivos en comparación con la primera por lo que no ha habido grandes avances desde sus inicios excepto en el estudio de las métricas que mejores reglas de asociación proporcionan, dado que el número de reglas generadas puede ser demasiado grande para ser analizado correctamente. En la sección~\ref{sec:arm:generacion-ar} se aborda esta fase, la generación de \ars a partir del conjunto de \itemsets frecuentes, y se presenta una propuesta para ejecutarla de un modo más eficiente, propuesta que forma parte de nuestro artículo más citado.






\section{\fim}
\label{sec:arm:fim}
% !TEX root = ../../Lazcorreta.Tesis.tex
Frequent Itemset Minning\ldots


\subsection{Algoritmos y estructuras}
\ldots





\subsection{Evaluación de diferentes implementaciones}
\ldots








\section{Generación de \ars}
\label{sec:arm:generacion-ar}
% !TEX root = ../../Lazcorreta.Tesis.tex
% \ABIERTO%
Una vez resuelta la fase de \fim se utiliza el conjunto \aprioriL de \itemsets frecuentes para obtener las \ARs que contiene el almacén de transacciones \D. Ya tenemos todos los \itemsets cuyo \soporte sea superior al mínimo prefijado, $minSup$, pero aún no tenemos las \ARs, para ello hemos de recorrer todo \aprioriL, de cada uno de los \itemsets frecuentes que contiene obtener todas sus particiones en dos subconjuntos no vacíos y comparar la \confianza de las dos reglas que genera con el umbral mínimo establecido para el estudio, $minConf$.

\apriori es un algoritmo hermoso, sencillo y moldeable, de ahí la atención que ha recibido por parte de la comunidad científica. Está descrito de un modo muy general por lo que se puede modificar con facilidad cualquiera de sus partes sin perder su esencia. La mayoría de los trabajos expuestos en la sección~\ref{sec:2-2-FIM} y de los que mostraremos en la sección~\ref{sec:2-4-ElItemRaro} se centran en mejorar la primera fase de este algoritmo, la \fim, debido a su gran necesidad de recursos y de procesos de lectura/escritura en disco. Si leemos con atención el párrafo anterior podremos ver que la fase de generación de \ars también tiene mucha carga de procesos:

\begin{enumerate}
  \item Hemos de leer cada \itemset frecuente de \aprioriL, lo que se conseguirá mediante un simple bucle que recorra todos sus nodos.
  \item Cada \itemset se divide en un par de subconjuntos no vacíos, $X_1$ y $X_2$, que serán el antecedente y consecuente de dos \ARs diferentes, $X_1 \rightarrow X_2$ y $X_2 \rightarrow X_1$.
  \item Para obtener la \confianza de cada regla hemos de buscar en \aprioriL el \soporte de $X_1$ y de $X_2$, calcular su cociente y determinar si tienen o no confianza mínima.
\end{enumerate}

Los dos primeros pasos son elementales, sin embargo el paso 3 es más complejo de lo que parece a simple vista. Para encontrar el \soporte de cada \kitemset hemos de hacer $k$ búsquedas en \aprioriL. Primero hemos de localizar en \aprioriL[1] el nodo que representa a su primer ítem, una vez encontrado entramos en la rama que se deriva de él y localizar en \aprioriL[2] el nodo que representa a su segundo ítem, siguiendo el proceso hasta localizar su $k$-ésimo ítem en \aprioriL[k]. Aunque usemos algoritmos de búsqueda eficiente en cada rama tenemos que hacer cientos de comparaciones para encontrar en \aprioriL cada uno de los \itemsets obtenidos en el paso 2, lo que nos hizo pensar que es una fase en la que podría aplicar alguna optimización.









%TODO: Esto debería estar en la siguiente sección. Hay muchas propuestas para reducir el número de reglas obtenidas pero ninguna se centra en la propia obtención de las reglas...
% \citet{BrinMotwaniSilverstein_BeyondMarketBaskets_1997} proponen el filtrado de reglas de asociación generalizadas mediante el análisis estadístico de correlación. Su idea es apoyar el estudio en el test de correlación $\chi^2$, lo que puede reducir el número de relaciones encontradas y agilizar por tanto el análisis. Definen las \emph{reglas de correlación} y proponen un algoritmo para extraer las reglas de correlación presentes en grandes \dbs.% (ver algoritmo \ref{alg:chiDosSupport}).
%
% \begin{Definition}[\emph{Reglas de Correlación}]\label{def:2-1-ReglasDeCorrelacion}
% Sea \I un conjunto de ítems y $B$ un conjunto de subconjuntos de \I. Decimos que $\left\{i_{a_1},i_{a_2}\ldots i_{a_m}\right\}$ es una \emph{Regla de correlación} si las ocurrencias de los ítems $i_{a_1},i_{a_2}\ldots i_{a_m}$ están correladas.
% \end{Definition}

% \afterpage{\clearpage
           % \lstinputlisting[label=alg:chi2support,
           %                  caption={Algoritmo $\chi^2$-support, 1997},
           %                  float=htb,
           %                  basicstyle=\scriptsize]
           %                  {./2-ARM/codigo/alg-chi2support}
% }

% \input{algoritmo/alg_chi2support}
%
% \begin{ejem}
% \label{ejem:Brin}
% Los datos mostrados en la tabla \ref{tab:ejem-Brin} reflejan las compras de café ($c$) y té ($t$) realizadas en un comercio. $\bar{c}$ indica las transacciones que no contenían café y $\bar{t}$ las que no contenían té.
% \begin{table}[ht]
%   \centering
%     \begin{tabular}{|c|cc|c|}\hline
%        & $c$ & $\bar{c}$ & $\sum_{\text{filas}}$ \\\hline
%       $t$ & 20 & 5 & 25 \\
%       $\bar{t}$ & 70 & 5 & 75 \\\hline
%       $\sum_{\text{columnas}}$ & 90 & 10 & 100 \\\hline
%     \end{tabular}
%   \caption{Tabla de contingencia entre $c$ y $t$.}
%   \label{tab:ejem-Brin}
% \end{table}
% Con estos datos la regla $t\rightarrow c$ tiene un \soporte del 20\% y una confianza del 80\%. Si consideramos que $c$ tiene un \soporte del 90\% la confianza de la regla muestra una correlación negativa entre ambos ítems ya que si sabemos que el usuario ha adquirido té se reduce la probabilidad de que adquiera café en un 10\%. Esto puede cuantificarse a partir del cálculo $P(c\cap t) / \left(P(c)\cdot P(t)\right)$, que valdría 1 si los hechos de comprar café o té fueran independientes. En este ejemplo su valor es de 0.89, inferior a 1 e indicador de la existencia de una correlación negativa entre ambos eventos.

% También mencionan que la correlación $\ro$ entre comprar te y no comprar café es de 2, alta correlación positiva... Y terminan con una sugerencia al vendedor:  As a result, the store manager may decide to target non-coffee drinkers in his tea displays.
% \end{ejem}
%
% En el mismo artículo postula que la correlación negativa no es necesariamente indeseable. Por ejemplo, en la prevención de incendios sería más interesante encontrar correlaciones negativas entre materiales usados en una construcción y la ocurrencia de incendios. En casos como este se comprende la importancia que tiene la contextualización del estudio en curso.



\subsection{\texttt{genrules()}}
\label{sec:2-2-1-genrules}
%\input{./2-ARM/2-3-GeneracionDeAR/2-3-1-genrules}



\subsection{Apriori2}
\label{sec:2-2-2-Apriori2}
%\input{./2-ARM/2-3-GeneracionDeAR/2-3-2-Apriori2}




% \subsection{Experimentación}
% \label{sec:2-2-X-Experimentacion}
% \input{./2-ARM/2-3-GeneracionDeAR/2-3-X-Experimentacion}





\section{El \IR}
\label{sec:arm:el-item-raro}
% !TEX root = ../../Lazcorreta.Tesis.tex
\ABIERTO
% !TEX root = ../../../Lazcorreta.Tesis.tex
% \ABIERTO%
Desde la irrupción de la informática en pequeñas y grandes empresas se guarda en \dbs todo tipo de información, entre otra las \transacciones que modelan la cesta de la compra o las \sns de un usuario a través de un \portalWeb. El volumen de datos almacenados crece a diario por lo que su análisis inicial se realiza con técnicas como la \arm, capaces de extraer información sobre la coexistencia de ítems en grandes colecciones de datos.

El uso de \soporte mínimo mediante \arm permite abordar el problema sobre grandes \dbs con la tecnología actual pero impide obtener información sobre una gran cantidad de los ítems en estudio, ítems que tiene menor \soporte del fijado como mínimo para ser estudiados por lo que se denominan \irs.

El \dilemaIR ha sido estudiado por muchos investigadores para obtener reglas de asociación que involucren a ítems poco frecuentes, sin embargo todas las propuestas realizadas sobrecargan las tareas del algoritmo usado y de sus analistas y no aportan reglas de uso frecuente. En la sección~\ref{sec:arm:items-raros} expondremos estas propuestas.

En la sección~\ref{sec:arm:ROP} se introducen las \ROPs, que permiten a los algoritmos de búsqueda de reglas de asociación encontrar simultáneamente información de interés sobre los ítems que no superen el \soporte mínimo con un consumo mínimo de recursos, sin intervención de los analistas y aportando conocimiento útil. Con esta propuesta conseguimos aliviar parte del problema generado por el \dilemaIR.




\subsection{Estudio de ítems raros}
\label{sec:arm:items-raros}
% !TEX root = ../../../Lazcorreta.Tesis.tex
\ABIERTO
%El \dilemaIR ha provocado muchas investigaciones~\citep{LiuHsuMa_ARMWithMultipleMS_1999,PalshikarKaleApte_ARMUsingHeavyItemsets_2007}

Los \irs son ítems que por alguna razón no aparecen con frecuencia en las \transacciones. En una cesta de la compra los \irs suelen ser muy caros o duraderos, y también pueden ser novedades que tardarán en ser comprados de forma frecuente. En un \portalWeb los \irs son páginas muy especializadas, de baja calidad o páginas de novedades.

Los ítems exclusivos (por su alto precio, larga durabilidad o especialización) serán siempre \irs aunque se incorporen en un \sr ya que son infrecuentes por naturaleza. Sin embargo, los ítems nuevos deberían ser incorporados a los \srs hasta que alcancen la categoría de ítems frecuentes, momento en el cual ya pueden ser tratados mediante la búsqueda clásica de \ar.

\citet{HanFu-DiscoveryMultipleLevelARFromLargeDB-1995} proponen dividir \D mediante una jerarquía y buscar las \ars tanto entre ítems como entre las categorías en las que sitúan a cada uno de ellos. Con esta estrategia se puede reducir el número de ítems a vincular, cuando se trabaja con \irs se puede ver su relación con otros ítems mediante la categoría a la que pertenecen por lo que todos los ítems pueden estar representados en alguna \ar y se puede redirigir el análisis para estudiar mejor la relación de dichos \irs con el resto de ítems de la población. El procedimiento es correcto pero conlleva análisis específicos y la creación y mantenimiento de la jerarquía propuesta, lo que puede provocar una ralentización del análisis y falta de actualidad si aparecen nuevos ítems en la población y no se realiza en paralelo una correcta jerarquización de los mismos.







\citet{LiuHsuMa-ARMWithMultipleMS-1999} y \citet{PalshikarKaleApte-ARMUsingHeavyItemsets-2007} ponen el énfasis en la distribución de los ítems presentes en \D. La investigación precedente sobre \arm considera una distribución uniforme de los ítems de \D, lo que no es correcto en la mayoría de las ocasiones y produce una gran cantidad de reglas de asociación que no aportan información de calidad sobre las relaciones reales existentes entre los ítems de \D. Esa información de baja calidad presenta otro problema: no deja suficientes recursos para analizar otra información de mayor calidad pero menor presencia en la \db. Para resolver esto proponen el uso de \soportes mínimos múltiples en lugar del clásico \soporte mínimo único, de modo que los ítems con mayor \soporte en \D necesiten más evidencias para mostrar relaciones entre sí que los ítems con poca presencia, a los que los algoritmos clásicos simplemente ignoran considerándolos \irs.

 % \input{algoritmo/alg_MSApriori}
 %
 % Proponen el algoritmo \ac{MSApriori} (ver algoritmo~\ref{alg:MSApriori}) que es una modificación de Apriori en que se considera el \soporte particular de cada ítem ($MIS$). Para determinar el $MIS$ de cada ítem proponen su asignación por parte del analista o bien el uso de la distribución de frecuencias de los ítems presentes en \DD.






%TODO: Sección 2.3 (Abarcar más ítems raros prescindiendo de los frecuentes)
% \citet{GrothRoberston-DiscoveringFrequentItemsets-2001} inciden en la gran cantidad de reglas de asociación descubiertas por los algoritmos expuestos y la necesidad de aplicarles técnicas de \ARM para extraer conocimiento comprensible por los analistas. En vez de centrarse en la identificación de \itemsets no frecuentes para evitar su estudio proponen técnicas que, partiendo del algoritmo \apriori, excluyen del estudio los \itemsets que sabemos que serán frecuentes. Proponen el algoritmo \algoritmo{RangeApriori} que se diferencia de \apriori en la generación de 2-candidatos (ver algoritmo~\ref{alg:RangeApriori}) donde se obtienen todos los \itemsets con \soporte superior a $s_1$ pero inferior a $s_2$ al que denominan \textsl{nivel máximo de \soporte}. A partir de este nivel no modifican \apriori pues sólo proporciona \itemsets derivados de los \kitemsets[2] frecuentes encontrados.
%
% % % \afterpage{\clearpage
%            \lstinputlisting[label=alg:RangeApriori,
%                             caption={Algoritmo {RangeApriori} (generación de 2-candidatos), 2001},
%                             float=htb,
%                             basicstyle=\scriptsize]
%                             {./2_ARM/codigo/alg_RangeApriori}
% % % }
%  % \input{algoritmo/alg_RangeApriori}





%TODO: Sección 2.3
 % \citet{KourisMakrisTsakalidis-AnImprovedAlgorithmForMARUsingMSValues-2003} proponen el uso de múltiples \soportes mínimos mediante un algoritmo obtenido a partir de las ideas presentadas con los algoritmos \ac{DIC} y \ac{MSApriori}.






%TODO: Sección 2.3
 % \citet{Hu-AnEffAlgForDiscAndMaintFPWithMMS-2003} presenta una nueva estructura, MIS-Tree, para guardar la información más relevante sobre los \itemsets frecuentes que contenga \DD. Utilizando múltiples \soportes mínimos propone el algoritmo CFP-Growth que, según sus experimentos, es más eficiente que \ac{MSApriori}. Con otro algoritmo que propone para mantener el MIS-Tree se puede ajustar el \soporte mínimo de los ítems sin volver a generarlo.
 %






 % \citet{ZhangLuZhang-FuzzyLogicBased-2004} abordan el problema de determinar un \soporte mínimo para una \db de la que no tenemos información y proponen un algoritmo que resuelve esta común situación utilizando lógica difusa, de modo que el \soporte mínimo no lo decide el analista si no la propia distribución de \DD.




 % \citet{LeeHongLin-MiningAR-2005} presentan un algoritmo basado en Apriori que permite que cada ítem tenga su propio \soporte mínimo, en lugar de trabajar con un sopote mínimo común a todos los ítems.




 %TODO: Sección 2.3
 % \citet{HuChen-MiningAR-2006} presentan en un artículo el trabajo de \citet{Hu-AnEffAlgForDiscAndMaintFPWithMMS-2003}.








%Para abordar el problema del \ir~
\cite{LiuHsuMa-ARMWithMultipleMS-1999} proponen usar múltiples \soportes, de modo que la relación
entre dos ítems frecuentes sea considerada sólo si es una relación muy frecuente en \D. De este modo se alivia considerablemente la carga de memoria requerida por el algoritmo y permite abordar el estudio sobre un número mayor de ítems. La asignación de \soporte a cada ítem puede hacerse por parte del analista o bien teniendo en cuenta el propio \soporte de cada ítem en \D. En~\cite{KiranReddy-ImprovedMultipleMSBasedAppMineRareAR-2009} se propone una modificación interesante del algoritmo propuesto por~\cite{LiuHsuMa-ARMWithMultipleMS-1999}, incorporando al estudio medidas de tendencia de los ítems de \D.

Con la primera propuesta de~\cite{LiuHsuMa-ARMWithMultipleMS-1999} se pueden incorporar los ítems nuevos asignándoles un \soporte
mínimo muy bajo mientras sean nuevos, sin embargo el analista debe decidir en algún momento cuándo debe modificar su \soporte mínimo y qué nuevo \soporte asignarle.

En las restantes propuestas el \soporte mínimo se obtiene en función del \soporte real del ítem, con lo que muchos \irs se incorporan al sistema con facilidad, sin embargo siguen quedando muchos ítems sobre los que no obtenemos información pues si intentamos incorporarlos al estudio se detiene la ejecución del algoritmo por falta de recursos. Además las reglas que proporcionan siempre sugieren una relación entre un \ir como \antecedente y uno que no lo es como \consecuente, lo que en un \sr no es práctico porque se usaría únicamente cuando el usuario solicite un \ir, lo que ocurre en muy pocas ocasiones dada la naturaleza del ítem solicitado.















\borrar{Lo que queda de sección no está formateado}

candidatos
a considerar que haría abortar la ejecución
del algoritmo por falta de recursos.

Input: D, sm (\soporte mínimo) y om (oportunidad mínima)

Output: RO (\ROPs) y RA (\ARs) presentes en D

/* Obtener frecuencia de todos los ítems y 2-itemsets de \D */

foreach \transaccion Ti en D

foreach i1 en Ti {

Incrementa(FP1[i1]);

foreach (i1; i2) en Ti

Incrementa(FP1[i1]->FP2[i2]);

}

/* Extraemos las RO */

foreach i1 en FP1

if (FP1[i1] >= sm) then

foreach i2 en FP1[i1]->FP2

if (FP1[i2] < sm y FP2[i2]/FP1[i2]
>= om) then

Añadir RO(it1  it2);

Algoritmo 1: ORFind - Algoritmo de búsqueda de
Reglas de Oportunidad

Al aplicar el algoritmo clásico tras la ejecución del
algoritmo 1 a los \datasets T10I4D100K,
T40I10D100K y kosarak hemos obtenido información
de interés sobre todos los ítems de cada \dataset
en un tiempo inferior al empleado con la
implementación de las propuestas de [10] y [11].

Las reglas de asociación obtenidas entre ítems
frecuentes tienen el suficiente \soporte como para
ser utilizados como patrones de comportamiento
del colectivo estudiado. Las reglas de oportunidad
obtenidas vinculan los ítems infrecuentes a los
frecuentes, permitiendo a un sistema de recomendación
saber cuál es el momento más oportuno
para recomendar un ítem infrecuente de cuyo uso
no tenemos aún información suficiente.


\subsection{\ROPs}
\label{sec:arm:ROP}
% !TEX root = ../../../Lazcorreta.Tesis.tex
\ABIERTO

Las \ROPs son \ARs interpretadas en sentido inverso. Pueden ser útiles cuando aparece el \dilemaIR en nuestra colección de datos. El principal motivo de la aparición de este problema son los recursos de memoria utilizados, si tenemos que guardar mucha información sobre muchos datos podemos llegar a desbordar la memoria del sistema y perder todo el trabajo hecho hasta el momento. Hay muchas propuestas para aliviar este problema, todas ellas buscando perder cierta información que por algún motivo se va a considerar irrelevante a cambio de ganar espacio en memoria para poder analizar otros datos de la colección con información más relevante.
























En un Sistema de Recomendación Web se utiliza
información sobre taxonomías, contenido
semántico y uso del portal para hacer sugerencias
a sus usuarios. Sin embargo cuando un ítem es
nuevo no puede obtenerse información de su uso
hasta que no logra un cierto \soporte. Las reglas de
oportunidad permiten incorporar en el sistema la
información de uso de los ítems nuevos de modo
automático.

Las RO contienen los ítems frecuentes que son
visitados conjuntamente con los ítems sin \soporte
mínimo. Esto nos permite sugerir un \ir a
los usuarios que visitan alguna de las páginas que
más favorecen la presencia de dicho ítem, p.ej. la
página A.

Si la sugerencia es acertada es de esperar que el
\ir vaya aumentando su \soporte y se incorpore
de modo natural al proceso de estudio de
reglas de asociación, mostrando una asociación
cada vez más fuerte a la página A.

Si la sugerencia no es acertada el ítem seguirá
siendo raro y su asociación con la página A se irá
diluyendo a favor de una asociación con otras
páginas con las que realmente sea más afín. Esto
se debe a que el número de veces que aparece el
\ir en \D es pequeño y cualquier aparición
nueva del ítem modificará sustancialmente los
porcentajes en que se basan los algoritmos de
búsqueda de \ARs y \ROPs.


{\color{red}
Referencias

[4] Tseng , M.-C., Lin, W.-Y. Efficient mining of
generalized association rules with nonuniform
minimum support. Data \& Knowledge
Engineering 62 (1), pp. 41-64. 2007


[6] Kouris, I.N., Makris, C.H., Tsakalidis, A.K.
Using information retrieval techniques for
supporting data mining. Data \& Knowledge
Engineering 52 (3), pp. 353-383. 2005

}












\section{Publicaciones}
\label{sec:arm:publicaciones}
% !TEX root = ../../Lazcorreta.Tesis.tex
El trabajo expuesto en la sección~\ref{sec:2-2-1-EvaluacionDeImplementaciones} se presentó en Beijing (\urlConNotaAlPie{http://www.hci.international/index.php?module=conference&CF_op=view&CF_id=5}{HCII'07}). La investigación mostrada en la sección~\ref{sec:2-3-GeneracionDeAR} dio lugar a la publicación de un artículo en la revista \urlConNotaAlPie{http://www.journals.elsevier.com/expert-systems-with-applications/}{ESWA}, lo que posibilitó mucho más la difusión de nuestro trabajo. Por último presentamos en Barcelona (\urlConNotaAlPie{http://interaccion2009.aipo.es}{Interacción'09}) el trabajo expuesto en la sección~\ref{sec:2-4-ElItemRaro}.





\phantomsection
\subsection*{Selecting the Best Tailored Algorithm for Personalizing a Web Site, 2007}
\addcontentsline{toc}{subsection}{Actas de {HCII}'07}
\label{sec:nuestro-Selecting-2007}
En este artículo presentamos los resultados obtenidos al comparar diferentes implementaciones del algoritmo \apriori. Para seguir en la línea que estamos investigando hemos de crear nuestra propia implementación si queremos introducir los resultados obtenidos hasta la fecha y los que puedan ir surgiendo en el curso de la investigación. El código de las implementaciones puestas a prueba nos resultará útil para trabajar en nuestra propia implementación.

\begin{quote}
  Botella Beviá, F., Lazcorreta Puigmartí, E., Fernández-Caballero, A., González López, P., Gallud,
J.A. y Bia Platas, A. Selecting the Best Tailored Algorithm for Personalizing a Web Site. \emph{Proceedings of the 12th International Conference on Human-Computer Interaction ({HCII})}, Beijing (China), 2007. \leePDF{bib/nuestros/BotellaLazcorretaFCaballeroGonzalezGalludBia-SelectingTheBestTailoredAlgForPersonalAWebSite-2007.pdf}
\end{quote}

\begin{quotation}
	\noindent\textbf{Resumen}

\selectlanguage{english}
	\nopagebreak Automatic personalization for dynamic web systems has been carried out by several methods and techniques, like data mining or Web usage mining. The \apriori algorithm is one of the most used for web personalization. We have designed a web recommender system based on this algorithm to select the best user-tailored links. We needed an efficient algorithm to preserve updated our system in real time. We tested several implementations of \apriori algorithm but none of them looked to run properly with our data. We tested these implementations with different access log files of public domain (BMSWebView.dat, BMS-POS) and with our log data. We decided to develop a new implementation that better adapts to our data.
\selectlanguage{spanish}
\end{quotation}








\phantomsection
\subsection*{Towards personalized recommendation by two-step modified \apriori data mining algorithm, 2008}
\addcontentsline{toc}{subsection}{{ESWA}, vol. 35(3) 2008}
\label{sec:nuestro-Torwards-2008}

El artículo
\begin{quote}%\citet{Lazcorreta20081422}
  Lazcorreta Puigmartí, E., Botella Beviá, F. y Fernández-Caballero, A. Towards personalized recommendation by two-step modified \apriori data mining algorithm. {\em Expert Systems with Applications}, 35(3):1422--1429, 2008. \leePDF{bib/nuestros/LazcorretaBotellaFCaballero-TowardsPR-CIO-2008.pdf} \descargaCIO{http://cio.umh.es/files/2011/12/CIO_2007_26.pdf} \descarga{https://www.researchgate.net/publication/222690209_Towards_personalized_recommendation_by_two-step_modified_Apriori_data_mining_algorithm?ev=pub_cit}
\end{quote}

	\begin{quotation}
	\noindent\textbf{Resumen}

\selectlanguage{english}
	\nopagebreak In this paper a new method towards automatic personalized recommendation based on the behavior of a single user in accordance with all other users in web-based information systems is introduced. The proposal applies a modified version of the well-known \apriori data mining algorithm to the log files of a web site (primarily, an e-commerce or an e-learning site) to help the users to the selection of the best user-tailored links. The paper mainly analyzes the process of discovering association rules in this kind of big repositories and of transforming them into user-adapted recommendations by the two-step modified \apriori technique, which may be described as follows. A first pass of the modified \apriori algorithm verifies the existence of association rules in order to obtain a new repository of transactions that reflect the observed rules. A second pass of the proposed \apriori mechanism aims in discovering the rules that are really inter-associated. This way the behavior of a user is not determined by "what he does" but by "how he does". Furthermore, an efficient implementation has been performed to obtain results in real-time. As soon as a user closes his session in the web system, all data are recalculated to take the recent interaction into account for the next recommendations. Early results have shown that it is possible to run this model in web sites of medium size.
\selectlanguage{spanish}
	\end{quotation}

%ha sido citado en las siguientes publicaciones\footnote{Todos los enlaces han sido revisados en noviembre de 2014. Los que enlazan a \url{http://www.scopus.com} requieren identificación, puede verse un resumen de estas publicaciones en \scopus[7pt]{http://www.scopus.com/results/citedbyresults.url?sort=plf-f&cite=2-s2.0-44949263923&src=s&imp=t&sid=D17660FCC894527FCCD93BAF74F677F3.ZmAySxCHIBxxTXbnsoe5w\%3a1280&sot=cite&sdt=a&sl=0&origin=recordpage&txGid=D17660FCC894527FCCD93BAF74F677F3.ZmAySxCHIBxxTXbnsoe5w\%3a128} y en el fichero \texttt{bib/NosCitan/scopus.bib}, en formato \BibTeX, del \dvdAdjunto que complementa este informe. En \url{http://scholar.google.es/scholar?oi=bibs&hl=es&cites=15432134191557693855} nos atribuyen 45 citas.}.
%
%
%
%%\subsection*{Artículos que nos citan}
%
%\selectlanguage{english}
%\begin{enumerate}
%
%
%
%  %2008
%  
%  %~\nocite{RasteragiMdSap_DataMiningAndECommerce_2008}
%	\item Hamid Rastegari y Mohd.~Noor Md.~Sap. Data mining and e-commerce : methods, applications, and challenges. \emph{Jurnal Teknologi Maklumat}, 20:116--128, 2008. \leePDF{bib/NosCitan/RasteragiMdSap_DataMiningAndECommerce_2008.pdf} \descarga{http://eprints.utm.my/9427/1/MohdNoorMd2008_DataMiningAndE-Commerce.pdf}
%
%	\item Kittisak Kerdprasop y Nittaya Kerdprasop. A Rough Set Approach to Personalization in 
%Web-Based Learning Systems. In \emph{Proceedings of {IADIS} International Conference e-Learning}, pages 22--25, july 2008. \leePDF{bib/NosCitan/KerdprasopKerdprasop_ARoughSetApproachToPersonalizationInWebBasedSystems_2008.pdf} \descarga{https://www.researchgate.net/publication/220970028_A_Rough_Set_Approach_To_Personalization_In_Web-Based_Learning_Systems}
%
%
%
%
%  %2009
%
%  %~\nocite{LiXuWangChu_StrongestAssociationRulesMiningForPersonalizedRecommendation_2009}
%	\item J.~Li, Y.~Xu, Y.~Wang y C.~Chu. Strongest association rules mining for personalized recommendation. \emph{Xitong Gongcheng Lilun yu Shijian/System Engineering Theory and Practice}, 29(8):144--152, 2009.
%    %\leePDF{bib/NosCitan/LiXuWangChu_StrongestAssociationRulesMiningForPersonalizedRecommendation_2009.pdf}
%    \descarga{https://www.researchgate.net/publication/251709116_Strongest_Association_Rules_Mining_for_Personalized_Recommendation} \scopus{http://www.scopus.com/inward/record.url?eid=2-s2.0-70249125844&partnerID=40&md5=9b944e88ca9e796c4b539bda50d0f24e}
%
%  %~\nocite{KahramanliAllahverdi_NewMethodForComposingClassificationRulesARplusOPTBP_2009}
%	\item Humar Kahramanli y Novruz Allahverdi. A new method for composing classification rules: {AR+OPTBP}. In \emph{Proceedings of the 5th International Advanced Technologies Symposium (IATS'09)}, pages 1--6, may 2009. \leePDF{bib/NosCitan/KahramanliAllahverdi_NewMethodForComposingClassificationRulesARplusOPTBP_2009.pdf} \descarga{http://iats09.karabuk.edu.tr/press/bildiriler_pdf/IATS09_01-01_105.pdf}
%
%  %\nocite{ClewleyChenLiu_ApplicationsForDMTechniquesInCRM_2009}
%  \item N.~Clewley, S.Y.~Chen y X.~Liu \emph{Encyclopedia of Information Science and Technology 2nd edition}, 188--192, Idea Group Publishing, 2009. \leePDF{bib/NosCitan/ClewleyChenLiu_ApplicationsForDMTechniquesInCRM_2009.pdf} \descarga{http://www.irma-international.org/viewtitle/13571/} \visita{http://what-when-how.com/information-science-and-technology/applications-for-data-mining-techniques-in-customer-relationship-management-information-science/}
%  
%  \item L.~Jie, X.~Yong, Y.~Wang y C.~Chu \emph{Strongest association rules mining for personalized recommendation}. In \emph{Systems Engineering-Theory \& Practice}, vol. 29 issue 8, pgs. 144--152, aug 2009. \leePDF{bib/NosCitan/JieYongWangChu-StrongestARMforPersonalizedRecommendation-2009.pdf}
%  
%  
%  
%  
%  %2010
%
%  %~\nocite{Kirkos2010193}
%	\item E.~Kirkos. \emph{Modeling the auditors'{} opinions by using association rules}. Nova Science Publishers, Inc., 2010. \scopus{http://www.scopus.com/inward/record.url?eid=2-s2.0-84892052266&partnerID=40&md5=1f19c4a0e9edf4ffe238cedbaee5c05e}
%
%  %~\nocite{MarkowskaKwasnickaParadowski_IntelligentTechniquesInPersonalizationOfLearningInELearningSystems_2010}
%	\item U.~Markowska-Kaczmar, H.~Kwasnicka y M.~Paradowski. Intelligent techniques in personalization of learning in e-learning systems. \emph{Studies in Computational Intelligence}, 273:1--23, 2010. \leePDF{bib/NosCitan/MarkowskaKwasnickaParadowski_IntelligentTechniquesInPersonalizationOfLearningInELearningSystems_2010.pdf} \descarga{http://www.researchgate.net/publication/202144697_Intelligent_Techniques_in_Personalization_of_Learning_in_e-Learning_Systems/links/02bfe50e46c68dbb7f000000} \scopus{http://www.scopus.com/inward/record.url?eid=2-s2.0-77950283870&partnerID=40&md5=52470b95e2e46e28f0f33ccf2da79d1b}
%
%  %~\nocite{YuZhou_ParallelTIDBasedFrequentPatternMiningAlgorithmOnAPCClusterAndGridComputingSystem_2010}
%  \item K.-M.~Yu y J.~Zhou. Parallel tid-based frequent pattern mining algorithm on a pc cluster and grid computing system. \emph{Expert Systems with Applications}, 37(3):2486--2494, 2010.
%   % \leePDF{bib/NosCitan/YuZhou_ParallelTIDBasedFrequentPatternMiningAlgorithmOnAPCClusterAndGridComputingSystem_2010.pdf}
%   \scopus{http://www.scopus.com/inward/record.url?eid=2-s2.0-70449535621&partnerID=40&md5=271da43f8d3775ab73eab6e088791e11}
%
%  %~\nocite{ZafraVentura_WebUsageMiningForImprovingStudentsPerformanceInLearningManagementSystems_2010}
%  \item A.~Zafra y S.~Ventura. Web usage mining for improving students performance in learning
%  management systems. \emph{Lecture Notes in Computer Science (including subseries Lecture Notes in Artificial Intelligence and Lecture Notes in Bioinformatics)}, 6098(PART 3):439--449, 2010. \leePDF{bib/NosCitan/ZafraVentura_WebUsageMiningForImprovingStudentsPerformanceInLearningManagementSystems_2010.pdf} \descarga{http://sci2s.ugr.es/keel/pdf/keel/congreso/60980439.pdf} \scopus{http://www.scopus.com/inward/record.url?eid=2-s2.0-79551507673&partnerID=40&md5=dd539d3a93187ea49a7b9ff60f5257c1}
%
%  \item Reza~Samizadeh y Babak~Ghelichkhani. Use of Semantic Similarity and Web Usage Mining to Alleviate the Drawbacks of User-Based Collaborative Filtering Recommender Systems. \emph{International Journal of Industrial Engineering \& Production Research}, 23(3):137--146, sept 2010.
%\leePDF{bib/NosCitan/SamizadehGhelichkhani_UseOfSemanticSimilarityAndWUM_2010.pdf} \descarga{https://www.researchgate.net/publication/49606961_Use_of_Semantic_Similarity_and_Web_Usage_Mining_to_Alleviate_the_Drawbacks_of_User-Based_Collaborative_Filtering_Recommender_Systems}
%
%  %~\nocite{Zhou2010435}
%  \item J.~Zhou, K.-M.~Yu y B.-C.~Wu. Parallel frequent patterns mining algorithm on {GPU}. \emph{Conference Proceedings - IEEE International Conference on Systems, Man and Cybernetics}, 435--440, Istanbul, 2010. \scopus{http://www.scopus.com/inward/record.url?eid=2-s2.0-78751550624&partnerID=40&md5=bc8834c1550b6485fc0cae7a2bf31643}
%  
%  \item J.~Pinho Lucas \emph{Métodos de clasificación basados en asociación aplicados a sistemas de recomendación}, Tesis doctoral de la Universidad de Salamanca, octubre de 2010. \leePDF{bib/NosCitan/Lucas-MetodosDeClasificacion-2010-Tesis.pdf}
%  
%  
%  
%  
%  %2011
%
%  %~\nocite{Acharya:Modi:AlgForFindingFIBasedOnLatticeApproach:11}
%  \item Ajay Acharya y Shweta Modi. An algorithm for finding frequent \itemset based on lattice approach for lower cardinality dense and sparse dataset. \emph{International Journal on Computer Science and Engineering}, 3(1):371--378, jan 2011. \leePDF{bib/NosCitan/AcharyaModi_AlgForFindingFIBasedOnLatticeApproach_2011.pdf} \descarga{http://www.enggjournals.com/ijcse/doc/IJCSE11-03-01-121.pdf}
%
%  \item N.~Kerdprasop \emph{The Handbook of Emergent Technologies in Social Research}, chap. 18:412--434, Oxford University Press, 2011. \visita{https://books.google.es/books?id=Q9HlpMF7GgkC}
%
%  %~\nocite{Afify2011360}
%  \item A.A.~Afify. Discovery of association rules from manufacturing data. \emph{International Journal of Computer Aided Engineering and Technology}, 3(3-4):360--371, 2011. \visita{https://www.deepdyve.com/lp/inderscience-publishers/discovery-of-association-rules-from-manufacturing-data-C0PG4I07B4}  \scopus{http://www.scopus.com/inward/record.url?eid=2-s2.0-84869832476&partnerID=40&md5=e6dfc12e7feba592ac51d80cba44e46a}
%
%  %~\nocite{Cai2011117}
%  \item W.~Cai, D.~Vassalos, D.~Konovessis y G.~Mermiris. Safety (total risk) management for passenger ships - learning from the past, managing the future risk. \emph{Proceedings of the International Conference on Design and Operation of Passenger Ships}, 117--126, London, 2011. \scopus{http://www.scopus.com/inward/record.url?eid=2-s2.0-80052598465&partnerID=40&md5=e08c2909c56375417f7ee64c25b092cb}
%
%  %~\nocite{HuangLuDuan_MiningAssociationRulesToSupportResourceAllocationInBusinessProcessManagement_2011}
%  \item Z.~Huang, X.~Lu y H.~Duan. Mining association rules to support resource allocation in business process management. \emph{Expert Systems with Applications}, 38(8):9483--9490, 2011.
%   % \leePDF{bib/NosCitan/HuangLuDuan_MiningAssociationRulesToSupportResourceAllocationInBusinessProcessManagement_2011.pdf}
%   \scopus{http://www.scopus.com/inward/record.url?eid=2-s2.0-79953688532&partnerID=40&md5=d618a2f9133695ad3e419add6f4f5c0a}
%
%  %~\nocite{Li20111067}
%  \item M.~Li, Z.~Ge, Z.~Yu, Y.~Qi, K.~Liu y H.~Guo. Root-soil system domain knowledge constructing method based on relational analysis of single impact factor. \emph{Proceedings of the 2nd International Conference on Digital Manufacturing and Automation (ICDMA)}, pages 1067--1070, Zhangjiajie, Hunan, 2011. \scopus{http://www.scopus.com/inward/record.url?eid=2-s2.0-80855140462&partnerID=40&md5=94f65d480b6da285c1fa07f0b90f4b1a}
%
%  %~\nocite{LiLiuLi_AnApproachToExpertRecommendationBasedOnFuzzyLinguisticMethodAndFuzzyTextClassificationInKnowledgeManagementSystems_2011}
%  \item M.~Li, L.~Liu y C.-B. Li. An approach to expert recommendation based on fuzzy linguistic method and fuzzy text classification in knowledge management systems. \emph{Expert Systems with Applications}, 38(7):8586--8596, 2011.
%   % \leePDF{bib/NosCitan/LiLiuLi_AnApproachToExpertRecommendationBasedOnFuzzyLinguisticMethodAndFuzzyTextClassificationInKnowledgeManagementSystems_2011.pdf}
%   \scopus{http://www.scopus.com/inward/record.url?eid=2-s2.0-79952438361&partnerID=40&md5=06a98c8a9ad710ee80c45cc5248e655c}
%
%  %~\nocite{ZafraRomeroVentura_MultipleInstanceLearningForClassifyingStudentsInLearningManagementSystems_2011}
%  \item A.~Zafra, C.~Romero y S.~Ventura. Multiple instance learning for classifying students in learning management systems. \emph{Expert Systems with Applications}, 38(12):15020--15031, 2011. \leePDF{bib/NosCitan/ZafraRomeroVentura_MultipleInstanceLearningForClassifyingStudentsInLearningManagementSystems_2011.pdf} \descarga{http://sci2s.ugr.es/keel/pdf/keel/articulo/FinalVersionPublished.pdf} \scopus{http://www.scopus.com/inward/record.url?eid=2-s2.0-80052021793&partnerID=40&md5=2a24a577493e65ba172586aa59587354}
%
%  %~\nocite{Zhong_TheResearchAndApplicationOfWebLogMiningBasedOnThePlatformWeka_2011}
%  \item X.-Y. Zhong. The research and application of web log mining based on the platform weka. \emph{Proceedings of the International Conference on Advanced in Control Engineering and Information Science (CEIS), Procedia Engineering} vol.~15:4073--4078, Dali, Yunnam, 2011.
%   % \leePDF{bib/NosCitan/Zhong_TheResearchAndApplicationOfWebLogMiningBasedOnThePlatformWeka_2011.pdf} \descarga{http://ac.els-cdn.com/S187770581102265X/1-s2.0-S187770581102265X-main.pdf?_tid=d2a3dc4a-69cc-11e4-8509-00000aab0f6c&acdnat=1415729095_1d0d2963e8c7b9c11ae94f1a9a2ed3f3}
%   \scopus{http://www.scopus.com/inward/record.url?eid=2-s2.0-84055223551&partnerID=40&md5=a1ccccbe07b3047a9568d49a911807e1}
%
%  %~\nocite{Zuhtuogullari201139}
%  \item K.~Zuhtuogullari y N.~Allahverdi. An improved \itemset generation approach for mining medical databases. \emph{International Symposium on INnovations in Intelligent SysTems and Applications (INISTA)}, 39--43, Istanbul-Kadikoy, 2011. \scopus{http://www.scopus.com/inward/record.url?eid=2-s2.0-79961201396&partnerID=40&md5=3fc459a344a8da1d485871a753af351f}
%  
%  
%  
%  
%  %2012
%
%  %~\nocite{Carmona201211243}
%  \item C.J. Carmona, S.~Ram\'irez-Gallego, F.~Torres, E.~Bernal, M.J. Del~Jesus y
%  S.~García. Web usage mining to improve the design of an e-commerce website: Orolivesur.com. \emph{Expert Systems with Applications}, 39(12):11243--11249, 2012. \scopus{http://www.scopus.com/inward/record.url?eid=2-s2.0-84861193731&partnerID=40&md5=5ab85a7f28c8008bd2af1bf062088ab6}
%
%  %~\nocite{Carmona2012239}
%  \item C.J.~Carmona, S.~Ramírez Gallego, F.~Torres, E.~Bernal, M.J. Del~Jesús y S.~García. Subgroup discovery applied to the e-commerce website orolivesur.com. \emph{Proceedings of the 14th International Conference on Enterprise Information Systems (ICEIS)}, 2:239--244, Wroclaw, 2012. \scopus{http://www.scopus.com/inward/record.url?eid=2-s2.0-84865755007&partnerID=40&md5=d6827429fc96e281d5c1a465f9f11e59}
%
%  %~\nocite{Hirose2012246}
%  \item F.~Hirose, K.~Uenosono y S.~Komiya. A {CAI} system to identify weak parts of a learner: On the numbers of category layers and questions in a set of questions. \emph{Proceedings of the IASTED International Conference on Engineering and Applied Science (EAS)}, 246--253, Colombo, 2012. \scopus{http://www.scopus.com/inward/record.url?eid=2-s2.0-84883532949&partnerID=40&md5=1e5609f509211842894c897ed81fc856}
%
%  %~\nocite{PinhoLucas20121273}
%  \item J.~Pinho~Lucas, S.~Segrera y M.N.~Moreno. Making use of associative classifiers in order to alleviate typical drawbacks in recommender systems.\emph{Expert Systems with Applications}, 39(1):1273--1283, 2012. \scopus{http://www.scopus.com/inward/record.url?eid=2-s2.0-81855166958&partnerID=40&md5=a8c1bd24d8e1077f5638a860ac565b3c}
%
%  %~\nocite{Rollande2012108}
%  \item R.~Rollande y J.~Grundspenkis. Representation of study program as a part of graph based framework for tutoring module of intelligent tutoring system.\emph{Proceedings of the 2nd International Conference on Digital Information Processing and Communications (ICDIPC)}, 108--113, Klaipeda City, 2012. \scopus{http://www.scopus.com/inward/record.url?eid=2-s2.0-84866696670&partnerID=40&md5=ccaa94aa86887c2593b2c4926c3fe3e6}
%
%  %~\nocite{Zafra20122693}
%  \item A.~Zafra y S.~Ventura. Multi-instance genetic programming for predicting student performance in web based educational environments. \emph{Applied Soft Computing Journal}, 12(8):
%  2693--2706, 2012. \scopus{http://www.scopus.com/inward/record.url?eid=2-s2.0-84861901483&partnerID=40&md5=83ee9766d53a62bbdce1bd94bdb0819a}
%  
%  \item Rafi Ahmad Khan. \emph{KDD for Business Intelligence}. In \emph{Journal of Knowledge Management Practice}, Vol. 13, No. 2, June 2012. \visita{http://www.tlainc.com/articl304.htm}
%  
%  
%  
%  %2013
%
%  %~\nocite{Kim20132281}
%  \item S.~Kim y D.A.~Nembhard. Rule mining for scheduling cross training with a heterogeneous
%  workforce. \emph{International Journal of Production Research}, 51(8):2281--2300, 2013. \scopus{http://www.scopus.com/inward/record.url?eid=2-s2.0-84874621570&partnerID=40&md5=552edc1b424f7bc2c2b3e6c544f206ba}
%
%  %~\nocite{Rollande2013137}
%  \item R.~Rollande y J.~Grundspenkis. Graph based framework and its implemented prototype for personalized study planning. \emph{Proceedings of the 2nd International Conference on E-Learning and E-Technologies in Education (ICEEE)}, 137--142, Lodz, 2013. IEEE Computer Society. \scopus{http://www.scopus.com/inward/record.url?eid=2-s2.0-84898655380&partnerID=40&md5=b3e578daec28f927f82ec488671add0c}
%
%  %~\nocite{Wei2013302}
%  \item C.~Wei. Concept association mining based on \clustering and association rules. \emph{International Journal of Applied Mathematics and Statistics}, 47(17):302--310, 2013. \scopus{http://www.scopus.com/inward/record.url?eid=2-s2.0-84887507914&partnerID=40&md5=c9390f88ea34670cc9fe6ea0a06533fa}
%  
%  
%  
%  %2014
%
%  %~\nocite{Azyurt2014145}
%  \item O.~Ozyurt. The classification of the probability unit ability levels of the eleventh grade turkish students by cluster analysis. \emph{Turkish Online Journal of Distance Education}, 15(2):145--160, 2014. \leePDF{bib/NosCitan/Ozyurt_TheClassificationOfTheProbabilityUnitAbilityLevels_2014.pdf} \descarga{https://tojde.anadolu.edu.tr/tojde56/pdf/article_11.pdf} \scopus{http://www.scopus.com/inward/record.url?eid=2-s2.0-84898749605&partnerID=40&md5=dc839972abdc75761b5b358fbeb2a7b6}
%
%  %~\nocite{PenaAyala201465}
%  \item A.~Peña-Ayala y L.~Cárdenas. How educational data mining empowers state policies to reform education: the mexican case study. \emph{Studies in Computational Intelligence}, 524:65--101, 2014. \scopus{http://www.scopus.com/inward/record.url?eid=2-s2.0-84891845403&partnerID=40&md5=481db82aad7f8fd5b71e26b54614ef19}
%
%  %~\nocite{Ruiz20144}
%  \item P.A.~Potes Ruiz, B.~Kamsu-Foguem y B.~Grabot. Generating knowledge in maintenance from experience feedback. \emph{Knowledge-Based Systems}, 68:4--20, 2014. \leePDF{bib/NosCitan/PotesKamsuGrabot_GeneratingKnowledgeInMaintenanceFromExperienceFeedback_2014.pdf} \descarga{https://www.researchgate.net/profile/Bernard_Kamsu-Foguem/publication/257480419_Improving_Maintenance_Strategies_from_Experience_Feedback/links/542be4dd0cf27e39fa91ba82.pdf?origin=publication_detail}
%\scopus{http://www.scopus.com/inward/record.url?eid=2-s2.0-84906951482&partnerID=40&md5=53d86da9ece5b7afb59096c82d611678}
%
%  %~\nocite{Zhu20142137}
%  \item H.~Zhu y Y.~He. Analysis on tourist characteristics based on rough sets theory and \apriori algorithm. \emph{FALTA REVISTA}, volume 46 VOLUME 3:2137--2143, Wuhan, 2014. \scopus{http://www.scopus.com/inward/record.url?eid=2-s2.0-84887945257&partnerID=40&md5=fe619df980d5476702f7b66c38a27021}
%  
%  \item Shiva Asadianfam y Masoud Mohammadi. \emph{Identify Navigational Patterns of Web Users}. In \emph{International Journal of Computer-Aided Technologies (IJCAx)} Vol.1,No.1,April 2014. \leePDF{bib/NosCitan/AsadianfamMohammadi-IdentifyNavigationalPatternsOfWebUsers-2014.pdf}
%  
%  \item H.~Hashemi, M.~Bayanati, K.~Eisapour, M.A.~Alavi y A.~Alavi. Provide a Model for Web Content Strategy, Using Data Mining Techniques. \emph{Advances in Environmental Biology}, 8(7):3299--3304, may 2014. \leePDF{bib/NosCitan/HashemiBayanatiEisapourAlaviAlavi_ProvideAModelForWebContentStrategyUsingDataMiningTechniques_2014.pdf}
%  
%  \item Samuel  Nowakowski,  Ivana  Ognjanovic,  Monique  Grandbastien,  Jelena  Jovanovic,  Ramo Sendelj. \emph{Two  Recommending  Strategies  to  Enhance  Online  Presence  in  Personal  Learning Environments}. In \emph{Springer  Science+Business  Media}  New  York  2014. \leePDF{bib/NosCitan/NowakowskiOgnjanovicGrandbastienJovanovicSendelj-TwoRecommendingStrategies-2014.pdf}
%  
%  %TODO: Tengo dos más de sep 2014 y enero 2015
%
%\end{enumerate}
\selectlanguage{spanish}






\phantomsection
\subsection*{Recomendaciones en sistemas web mediante el estudio de \irs en \transacciones, 2010}
\addcontentsline{toc}{subsection}{Actas de Interacción'10}
\label{sec:nuestro-Recomendaciones-2010}

En Interacción'10 presentamos las \ROPs a través del siguiente artículo:
\begin{quote}
  Lazcorreta Puigmartí, E., Botella Beviá, F. y Fernández-Caballero, A. Recomendaciones en sistemas web mediante el estudio de ítems raros en transacciones. {\em Actas del XI Congreso Internacional de Interacción Persona-Ordenador}, 385--388, 2010. \leePDF{bib/nuestros/LazcorretaBotellaFCaballero-RecomendacionesEnSistemasWeb-2010.pdf}
\end{quote}

	\begin{quotation}
	\noindent\textbf{Resumen}

	\nopagebreak Las recomendaciones en \portalesWeb se enriquecen cuando incorporan información sobre el uso real del portal. Una de las herramientas que proporcionan dicha información es el \ARM (ARM) en las sesiones de navegación de los usuarios a través del portal. Si el número de páginas del portal es muy grande la \ARM se tropieza con el \dilemaIR, que postula que no se puede encontrar información sobre las páginas poco visitadas sin obtener una explosión de \ars. Este dilema ha sido estudiado desde una perspectiva que sobrecarga las tareas del algoritmo usado y de sus analistas y no aporta reglas de uso frecuente. En este artículo se introduce un nuevo enfoque al dilema que permite a los algoritmos de \ARM encontrar simultáneamente información de interés sobre los páginas del portal poco visitadas, con un consumo mínimo de recursos, sin intervención de los analistas y aportando conocimiento útil.
	\end{quotation}


%\begin{quote} %\citet{Lazcorreta:Botella:FCaballero:ReglasDeOportunidad:09}
  %Lazcorreta Puigmartí, E., Botella Beviá, F. y Fernández-Caballero, A. Reglas de Oportunidad: mejorando las recomendaciones web. {\em Actas del X Congreso Internacional de Interacción Persona-Ordenador}, 2009
%\end{quote}

