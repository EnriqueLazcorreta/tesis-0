% !TEX root = ../Lazcorreta.Tesis.tex
Para llevar a cabo las pruebas de rendimiento y aplicabilidad de nuestras propuestas se han usado datos propios y datos procedentes de diferentes \datasets públicos como UCI, KEEL, LUCS-KDD\ldots





\section{Sistemas de Recomendación Web}
En este capítulo usamos datos de un servidor propio con la intención de poder utilizar las recomendaciones sugeridas por nuestra metodología en el servidor del que se obtuvieron. Son los ficheros \texttt{TAL} y \texttt{CUAL} que no publicaremos por carecer de interés su contenido, ya que el servidor del que se obtuvieron ya no está disponible y no se podría dar ninguna interpretación a los resultados obtenidos\ldots

También usamos \texttt{TAL}\ldots




\section{Minería de Reglas de Asociación}
En este capítulo seguimos trabajando con los mismos datos que en el anterior e incorporamos\ldots




\section{Catálogos}
En este capítulo es en el que más opciones hemos tenido a la hora de seleccionar datos y probar la eficiencia y posibilidades de nuestros desarrollos. Existen muchos \datasets públicos bien documentados sobre el diseño y contenido de estos \datasets y es una información que enriquece mucho la investigación\ldots




