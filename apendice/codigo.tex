% !TEX root = ../Lazcorreta.Tesis.tex
Se ha desarrollado mucho código para poder comprobar todo lo que se afirma en esta tesis. Se ha trabajado del modo más estándar posible para conseguir un código eficiente y que pueda ser incorporado a otras investigaciones. Se publicará como código abierto bajo la licencia\ldots en\ldots




\section{Sistemas de Recomendación Web}
En este capítulo\ldots

%\begin{minted}{c++}
%
%#include <stdlib>
%#using std::cout;
%#using std::endl;
%
%int main()
%{
%   cout << "Hola, mundo" << endl;
%   
%   return 0;
%}
%\end{minted}


%\begin{lstlisting}%[firstline=2,lastline=5]%
%
%#include <stdlib>
%#using std::cout;
%#using std::endl;
%
%int main(int argc, char *argv[])
%{
%   cout << "Holal, mundo" << endl;
%   
%    return 0;
%}
%\end{lstlisting}
%Si consigo instalar minted parece que es un buen paquete para colorear código, ya veré qué hago si no quiero usar listings
%\begin{minted}[mathescape,
%               linenos,
%               numbersep=5pt,
%               gobble=2,
%               frame=lines,
%               framesep=2mm]{c++}
%int main(int argc, char *argv[])
%{
%   // Este comentario tiene símbolos matemáticos $\pi=\lim_{n\to\infty}\frac{P_n}{d}$
%    return 0;
%}
%\end{minted}
%  string title = "This is a Unicode BORRADO in the sky"
%  /*
%  Defined as $\pi=\lim_{n\to\infty}\frac{P_n}{d}$ where $P$ is the perimeter
%  of an $n$-sided regular polygon circumscribing a
%  circle of diameter $d$.
%  */
%  const double pi = 3.1415926535



\section{Minería de Reglas de Asociación}
En este capítulo\ldots




\section{Catálogos}
En este capítulo\ldots


{\tiny
\lstinputlisting[mathescape,frame=single,caption=Cabecera para lectura de ficheros KEEL,extendedchars=true,inputencoding=utf8]{borrar/codigo/ficheroKEEL-utf8.h}
}
%\inputminted[frame=lines]{cpp}{borrar/codigo/ficheroKEEL.h}



