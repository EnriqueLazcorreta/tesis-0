% !TEX root = ../Lazcorreta.Tesis.tex
La notación usada en este informe se ha intentado ajustar a la más utilizada en la bibliografía revisada a lo largo de estos años de investigación. Por este motivo no es uniforme en los tres capítulos de investigación en que se divide esta tesis.

\section{Sistemas de Recomendación Web}
\label{sec:notacion:srw}
En este capítulo\ldots




\section{Minería de Reglas de Asociación}
\label{sec:notacion:arm}
En este capítulo\ldots




\section{\Catalogos}
\label{sec:notacion:catalogos}

Si no se indica explícitamente se usarán los subíndices $i$, $j$ y $p\ldots$ para denotar \atributo, ejemplo y orden respectivamente, con lo que, en la tabla~\ref{tab:notacion:catalogos}, $1\leq i\leq N$ y $1\leq j\leq M$ y $p\ldots$ dependerá del subíndice al que represente.

\begin{longtable}[c]{p{.3\textwidth}p{.7\textwidth}}
   \caption{Notación sobre \Catalogos}\\
   Símbolo & Descripción \\\hline
\endfirsthead
   \multicolumn{2}{c}{\tablename\ \thetable\ -- \textit{\ldots continuación}} \\\hline
   Símbolo & Descripción \\\hline
\endhead
\hline
\multicolumn{2}{r}{\textit{Continúa en la página siguiente\ldots}} \\
\endfoot
\hline
\endlastfoot
  $\D = \{x_1,\ldots,x_M\}$                                      & \dataset  con todos los ejemplos que vamos a analizar, generalmente es una muestra de individuos de la población en estudio pero también puede contener otro tipo de observaciones \\
  $M=|\D|$                                                               & Número de ejemplos del \dataset \D \\
  $\A = \{\A_1,\ldots,\A_N\}$                                     & Conjunto de \atributos en estudio \\
  $N = |\A|$                                                              & Número de \atributos en estudio \\
  $X = \{X_1, \ldots, X_N\}$                                      & Variable $X$ \\
  $X_i$                                                                      & \Atributo $i$-ésimo \\
  $\alpha_i = \{\alpha_i^1, \ldots, \alpha_i^{m_i}\}$   & Alfabeto del \atributo $i$-ésimo \\
  $m_i = |\alpha_i|$ & Tamaño del alfabeto $\alpha_i$ \\
  $x_{ij} \in \alpha_i$                                                 & Valor que toma el individuo $j$ en $\A_i$ \\
  $x_j = \{x^1_j,\ldots,x_j^N\}$                                 & Realización de $X_j$, suceso $\left[(X_1 = x_j^1) \wedge\ \ldots \wedge (X_N = x_j^N)\right]$, ejemplo, individuo $j$-ésimo del \dataset, \registro $j$-ésimo, \ldots \\
  $X_i$                                                                    & Suceso atómico, evento primario \\
  $x_{ip} \simeq \left[X_i = \alpha_i^p\right]$            & Realización de suceso atómico \\
  $X^s \subset X$                                                     & Subconjunto de $X$ formado sólo con los \atributos del subconjunto $s \subset\{1\ldots N\}$, suceso compuesto, conjunto de sucesos atómicos $X^s = \{X_i\ |\ i\in s\}$ \\
  $x^s = \{x_i\ |\ i\in s\}$                                           & Realización del suceso compuesto $X^s$ \\
\label{tab:notacion:catalogos}
\end{longtable}




